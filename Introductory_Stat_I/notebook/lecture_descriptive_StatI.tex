\documentclass[]{article}
\usepackage{lmodern}
\usepackage{amssymb,amsmath}
\usepackage{ifxetex,ifluatex}
\usepackage{fixltx2e} % provides \textsubscript
\ifnum 0\ifxetex 1\fi\ifluatex 1\fi=0 % if pdftex
  \usepackage[T1]{fontenc}
  \usepackage[utf8]{inputenc}
\else % if luatex or xelatex
  \ifxetex
    \usepackage{mathspec}
  \else
    \usepackage{fontspec}
  \fi
  \defaultfontfeatures{Ligatures=TeX,Scale=MatchLowercase}
\fi
% use upquote if available, for straight quotes in verbatim environments
\IfFileExists{upquote.sty}{\usepackage{upquote}}{}
% use microtype if available
\IfFileExists{microtype.sty}{%
\usepackage{microtype}
\UseMicrotypeSet[protrusion]{basicmath} % disable protrusion for tt fonts
}{}
\usepackage[margin=1in]{geometry}
\usepackage{hyperref}
\hypersetup{unicode=true,
            pdftitle={Introdução e Estatística Descritiva},
            pdfauthor={Fernando B. Sabino da Silva},
            pdfborder={0 0 0},
            breaklinks=true}
\urlstyle{same}  % don't use monospace font for urls
\usepackage{color}
\usepackage{fancyvrb}
\newcommand{\VerbBar}{|}
\newcommand{\VERB}{\Verb[commandchars=\\\{\}]}
\DefineVerbatimEnvironment{Highlighting}{Verbatim}{commandchars=\\\{\}}
% Add ',fontsize=\small' for more characters per line
\usepackage{framed}
\definecolor{shadecolor}{RGB}{248,248,248}
\newenvironment{Shaded}{\begin{snugshade}}{\end{snugshade}}
\newcommand{\KeywordTok}[1]{\textcolor[rgb]{0.13,0.29,0.53}{\textbf{#1}}}
\newcommand{\DataTypeTok}[1]{\textcolor[rgb]{0.13,0.29,0.53}{#1}}
\newcommand{\DecValTok}[1]{\textcolor[rgb]{0.00,0.00,0.81}{#1}}
\newcommand{\BaseNTok}[1]{\textcolor[rgb]{0.00,0.00,0.81}{#1}}
\newcommand{\FloatTok}[1]{\textcolor[rgb]{0.00,0.00,0.81}{#1}}
\newcommand{\ConstantTok}[1]{\textcolor[rgb]{0.00,0.00,0.00}{#1}}
\newcommand{\CharTok}[1]{\textcolor[rgb]{0.31,0.60,0.02}{#1}}
\newcommand{\SpecialCharTok}[1]{\textcolor[rgb]{0.00,0.00,0.00}{#1}}
\newcommand{\StringTok}[1]{\textcolor[rgb]{0.31,0.60,0.02}{#1}}
\newcommand{\VerbatimStringTok}[1]{\textcolor[rgb]{0.31,0.60,0.02}{#1}}
\newcommand{\SpecialStringTok}[1]{\textcolor[rgb]{0.31,0.60,0.02}{#1}}
\newcommand{\ImportTok}[1]{#1}
\newcommand{\CommentTok}[1]{\textcolor[rgb]{0.56,0.35,0.01}{\textit{#1}}}
\newcommand{\DocumentationTok}[1]{\textcolor[rgb]{0.56,0.35,0.01}{\textbf{\textit{#1}}}}
\newcommand{\AnnotationTok}[1]{\textcolor[rgb]{0.56,0.35,0.01}{\textbf{\textit{#1}}}}
\newcommand{\CommentVarTok}[1]{\textcolor[rgb]{0.56,0.35,0.01}{\textbf{\textit{#1}}}}
\newcommand{\OtherTok}[1]{\textcolor[rgb]{0.56,0.35,0.01}{#1}}
\newcommand{\FunctionTok}[1]{\textcolor[rgb]{0.00,0.00,0.00}{#1}}
\newcommand{\VariableTok}[1]{\textcolor[rgb]{0.00,0.00,0.00}{#1}}
\newcommand{\ControlFlowTok}[1]{\textcolor[rgb]{0.13,0.29,0.53}{\textbf{#1}}}
\newcommand{\OperatorTok}[1]{\textcolor[rgb]{0.81,0.36,0.00}{\textbf{#1}}}
\newcommand{\BuiltInTok}[1]{#1}
\newcommand{\ExtensionTok}[1]{#1}
\newcommand{\PreprocessorTok}[1]{\textcolor[rgb]{0.56,0.35,0.01}{\textit{#1}}}
\newcommand{\AttributeTok}[1]{\textcolor[rgb]{0.77,0.63,0.00}{#1}}
\newcommand{\RegionMarkerTok}[1]{#1}
\newcommand{\InformationTok}[1]{\textcolor[rgb]{0.56,0.35,0.01}{\textbf{\textit{#1}}}}
\newcommand{\WarningTok}[1]{\textcolor[rgb]{0.56,0.35,0.01}{\textbf{\textit{#1}}}}
\newcommand{\AlertTok}[1]{\textcolor[rgb]{0.94,0.16,0.16}{#1}}
\newcommand{\ErrorTok}[1]{\textcolor[rgb]{0.64,0.00,0.00}{\textbf{#1}}}
\newcommand{\NormalTok}[1]{#1}
\usepackage{graphicx,grffile}
\makeatletter
\def\maxwidth{\ifdim\Gin@nat@width>\linewidth\linewidth\else\Gin@nat@width\fi}
\def\maxheight{\ifdim\Gin@nat@height>\textheight\textheight\else\Gin@nat@height\fi}
\makeatother
% Scale images if necessary, so that they will not overflow the page
% margins by default, and it is still possible to overwrite the defaults
% using explicit options in \includegraphics[width, height, ...]{}
\setkeys{Gin}{width=\maxwidth,height=\maxheight,keepaspectratio}
\IfFileExists{parskip.sty}{%
\usepackage{parskip}
}{% else
\setlength{\parindent}{0pt}
\setlength{\parskip}{6pt plus 2pt minus 1pt}
}
\setlength{\emergencystretch}{3em}  % prevent overfull lines
\providecommand{\tightlist}{%
  \setlength{\itemsep}{0pt}\setlength{\parskip}{0pt}}
\setcounter{secnumdepth}{5}
% Redefines (sub)paragraphs to behave more like sections
\ifx\paragraph\undefined\else
\let\oldparagraph\paragraph
\renewcommand{\paragraph}[1]{\oldparagraph{#1}\mbox{}}
\fi
\ifx\subparagraph\undefined\else
\let\oldsubparagraph\subparagraph
\renewcommand{\subparagraph}[1]{\oldsubparagraph{#1}\mbox{}}
\fi

%%% Use protect on footnotes to avoid problems with footnotes in titles
\let\rmarkdownfootnote\footnote%
\def\footnote{\protect\rmarkdownfootnote}

%%% Change title format to be more compact
\usepackage{titling}

% Create subtitle command for use in maketitle
\newcommand{\subtitle}[1]{
  \posttitle{
    \begin{center}\large#1\end{center}
    }
}

\setlength{\droptitle}{-2em}
  \title{Introdução e Estatística Descritiva}
  \pretitle{\vspace{\droptitle}\centering\huge}
  \posttitle{\par}
  \author{Fernando B. Sabino da Silva}
  \preauthor{\centering\large\emph}
  \postauthor{\par}
  \date{}
  \predate{}\postdate{}


\begin{document}
\maketitle

{
\setcounter{tocdepth}{2}
\tableofcontents
}
\subsection{\texorpdfstring{\textbf{Rstudio}}{Rstudio}}\label{rstudio}

\begin{itemize}
\tightlist
\item
  Faça uma pasta no seu computador onde você deseja manter os arquivos
  para usar no \textbf{Rstudio}.
\item
  Defina o diretório de trabalho nesta pasta:
  \texttt{Session\ -\textgreater{}\ Set\ Working\ Directory\ -\textgreater{}\ Choose\ Directory}
  (atalho: Ctrl+Shift+H).
\item
  Torne a alteração permanente definindo o diretório padrão em:
  \texttt{Tools\ -\textgreater{}\ Global\ Options\ -\textgreater{}\ Choose\ Directory}.
\end{itemize}

\subsection{\texorpdfstring{\textbf{R}
básico}{R básico}}\label{r-basico}

\begin{itemize}
\tightlist
\item
  Cálculos simples:
\end{itemize}

\begin{Shaded}
\begin{Highlighting}[]
\FloatTok{4.6} \OperatorTok{*}\StringTok{ }\NormalTok{(}\DecValTok{2} \OperatorTok{+}\StringTok{ }\DecValTok{3}\NormalTok{)}\OperatorTok{^}\DecValTok{4} 
\end{Highlighting}
\end{Shaded}

\begin{verbatim}
## [1] 2875
\end{verbatim}

\begin{itemize}
\tightlist
\item
  Defina um objeto (escalar) e o imprima:
\end{itemize}

\begin{Shaded}
\begin{Highlighting}[]
\NormalTok{a <-}\StringTok{ }\DecValTok{4} 
\NormalTok{a}
\end{Highlighting}
\end{Shaded}

\begin{verbatim}
## [1] 4
\end{verbatim}

\begin{itemize}
\tightlist
\item
  Defina um objeto (vetor) e o imprima:
\end{itemize}

\begin{Shaded}
\begin{Highlighting}[]
\NormalTok{b <-}\StringTok{ }\KeywordTok{c}\NormalTok{(}\DecValTok{2}\NormalTok{, }\DecValTok{5}\NormalTok{, }\DecValTok{7}\NormalTok{)}
\NormalTok{b}
\end{Highlighting}
\end{Shaded}

\begin{verbatim}
## [1] 2 5 7
\end{verbatim}

\begin{itemize}
\tightlist
\item
  Defina uma sequência de números e a imprima:
\end{itemize}

\begin{Shaded}
\begin{Highlighting}[]
\NormalTok{s <-}\StringTok{ }\DecValTok{1}\OperatorTok{:}\DecValTok{4}
\NormalTok{s}
\end{Highlighting}
\end{Shaded}

\begin{verbatim}
## [1] 1 2 3 4
\end{verbatim}

\begin{itemize}
\tightlist
\item
  Nota: Um comando mais flexível para sequências:
\end{itemize}

\begin{Shaded}
\begin{Highlighting}[]
\NormalTok{s <-}\StringTok{ }\KeywordTok{seq}\NormalTok{(}\DecValTok{1}\NormalTok{, }\DecValTok{4}\NormalTok{, }\DataTypeTok{by =} \DecValTok{1}\NormalTok{)}
\end{Highlighting}
\end{Shaded}

\begin{itemize}
\tightlist
\item
  \textbf{R} faz cálculos elemento a elemento:
\end{itemize}

\begin{Shaded}
\begin{Highlighting}[]
\NormalTok{a }\OperatorTok{*}\StringTok{ }\NormalTok{b}
\end{Highlighting}
\end{Shaded}

\begin{verbatim}
## [1]  8 20 28
\end{verbatim}

\begin{Shaded}
\begin{Highlighting}[]
\NormalTok{a }\OperatorTok{+}\StringTok{ }\NormalTok{b}
\end{Highlighting}
\end{Shaded}

\begin{verbatim}
## [1]  6  9 11
\end{verbatim}

\begin{Shaded}
\begin{Highlighting}[]
\NormalTok{b }\OperatorTok{^}\StringTok{ }\DecValTok{2}
\end{Highlighting}
\end{Shaded}

\begin{verbatim}
## [1]  4 25 49
\end{verbatim}

\begin{itemize}
\tightlist
\item
  Soma e produto de elementos:
\end{itemize}

\begin{Shaded}
\begin{Highlighting}[]
\KeywordTok{sum}\NormalTok{(b)}
\end{Highlighting}
\end{Shaded}

\begin{verbatim}
## [1] 14
\end{verbatim}

\begin{Shaded}
\begin{Highlighting}[]
\KeywordTok{prod}\NormalTok{(b)}
\end{Highlighting}
\end{Shaded}

\begin{verbatim}
## [1] 70
\end{verbatim}

\subsection{\texorpdfstring{Extensões do
\textbf{R}}{Extensões do R}}\label{extensoes-do-r}

\begin{itemize}
\tightlist
\item
  O \textbf{R} não precisa ser usado apenas como calculadora ou para
  atribuição de objetos simples. A sua funcionalidade pode ser estendida
  atráves de bibliotecas ou pacotes (muito similar a utilização de
  Plug-ins nos navegadores ou baixar aplicativos no google play). Alguns
  já vem instalados (automaticamente, by default) no \textbf{R} e você
  precisa apenas carregá-los (como fazemos depois que baixamos um
  aplicativo no celular e queremos usá-lo, por exemplo).
\item
  Para instalar um novo pacote no \textbf{Rstudio} você pode usar o
  menu: \texttt{Tools\ -\textgreater{}\ Install\ Packages}
\item
  Você precisa saber o nome do pacote que deseja instalar. Você também
  pode fazê-lo através do comando install.packages como abaixo:
\end{itemize}

\begin{Shaded}
\begin{Highlighting}[]
\KeywordTok{install.packages}\NormalTok{(}\StringTok{"mosaic"}\NormalTok{)}
\end{Highlighting}
\end{Shaded}

\begin{itemize}
\tightlist
\item
  Uma vez que o pacote esteja instalado, você pode carregá-lo através do
  comando \texttt{library} (ou \texttt{require}):
\end{itemize}

\begin{Shaded}
\begin{Highlighting}[]
\KeywordTok{library}\NormalTok{(mosaic)}
\end{Highlighting}
\end{Shaded}

\begin{verbatim}
## Warning: package 'mosaic' was built under R version 3.4.3
\end{verbatim}

\begin{verbatim}
## Warning: package 'dplyr' was built under R version 3.4.3
\end{verbatim}

\begin{verbatim}
## Warning: package 'ggformula' was built under R version 3.4.3
\end{verbatim}

\begin{verbatim}
## Warning: package 'ggplot2' was built under R version 3.4.3
\end{verbatim}

\begin{verbatim}
## Warning: package 'mosaicData' was built under R version 3.4.3
\end{verbatim}

\begin{itemize}
\tightlist
\item
  Isto carrega o pacote \texttt{mosaic} que possuí muitas funções
  convenientes para este curso (voltaremos a isso mais tarde). Ele
  também imprime muitas informações sobre as funções que foram alteradas
  pelo pacote \texttt{mosaic}, mas você pode ignorar isto com segurança.
\end{itemize}

\subsection{\texorpdfstring{Ajuda do
\textbf{R}}{Ajuda do R}}\label{ajuda-do-r}

\begin{itemize}
\tightlist
\item
  Você pode receber ajuda (help) via
  \texttt{?\textless{}command\textgreater{}}:
\end{itemize}

\begin{Shaded}
\begin{Highlighting}[]
\NormalTok{?sum}
\end{Highlighting}
\end{Shaded}

\begin{itemize}
\tightlist
\item
  Procurando por ajuda:
\end{itemize}

\begin{Shaded}
\begin{Highlighting}[]
\KeywordTok{help.search}\NormalTok{(}\StringTok{"plot"}\NormalTok{)}
\end{Highlighting}
\end{Shaded}

\begin{itemize}
\tightlist
\item
  Você pode encontrar um cheat sheet com funções do \textbf{R} que
  usaremos neste curso
  \href{https://drive.google.com/open?id=1678x0r9WNYQJlQPlovgIbwgp1hN5dt5c}{aqui}.
  Caso o arquivo não apareça, clique com o botão direito em cima do link
  e escolha \texttt{Open\ link\ in\ a\ new\ tab}.
\item
  Você pode salvar os comandos que você porventura tenha digitado em um
  arquivo para uso posterior:

  \begin{itemize}
  \tightlist
  \item
    Selecione o guia History no painel superior direito no
    \textbf{Rstudio} .
  \item
    Marque os comandos que você deseja salvar.
  \item
    Pressione o botão \texttt{To\ Source}.
  \end{itemize}
\item
  Pratique as suas habilidades básicas em:
  \url{http://tryr.codeschool.com}
\end{itemize}

\subsection{Dados: Exemplos}\label{dados-exemplos}

\begin{itemize}
\tightlist
\item
  Data:
  \href{http://lib.stat.cmu.edu/DASL/Datafiles/magadsdat.html}{Legibilidade
  de Anúncios em Revistas}
\item
  Trinta revistas foram classificadas pelo nível educacional de seus
  leitores.
\item
  Três revistas foram selecionadas \textbf{aleatoriamente} de cada um
  dos seguintes grupos:

  \begin{itemize}
  \tightlist
  \item
    Grupo 1: maior nível educacional
  \item
    Grupo 2: nível educacional médio
  \item
    Grupo 3: nível educacional mais baixo.
  \end{itemize}
\item
  Seis anúncios foram selecionados \textbf{aleatoriamente} de cada uma
  das nove revistas selecionadas:

  \begin{itemize}
  \tightlist
  \item
    Grupo 1: {[}1{]} Scientific American, {[}2{]} Fortune, {[}3{]} The
    New Yorker
  \item
    Grupo 2: {[}4{]} Sports Illustrated, {[}5{]} Newsweek, {[}6{]}
    People
  \item
    Grupo 3: {[}7{]} National Enquirer, {[}8{]} Grit, {[}9{]} True
    Confessions
  \end{itemize}
\item
  Logo, os dados contém informações sobre um total de 54 anúncios.
\end{itemize}

\subsection{Objetivos do Capítulo}\label{objetivos-do-capitulo}

\begin{itemize}
\tightlist
\item
  Identificar o tipo de variável (por exemplo, numérica ou categórica;
  discreta ou contínua; ordenada ou não)
\item
  Usar visualizações apropriadas para diferentes tipos de dados (por
  exemplo, histograma, gráfico de barras (barplot), gráfico de dispersão
  (scatterplot), boxplot, etc.)
\item
  Criar e interpretar tabelas de contingência e de distribuições de
  frequência (tabelas uni e bidirecionais - de uma e duas entradas)
\item
  Usar diferentes medidas de tendência central e dispersão e ser capaz
  de descrever a robustez de diferentes estatística (por exemplo, quando
  devemos usar cada uma e até que ponto elas podem ser usadas)
\item
  Descrever a forma das distribuições (usando também gráficos como o
  histograma e o boxplot)
\end{itemize}

\subsection{Exemplo (continuação) - variáveis e
formato}\label{exemplo-continuacao---variaveis-e-formato}

\begin{itemize}
\tightlist
\item
  Para cada anúncio (54 casos), os dados abaixo foram observados.
\item
  \textbf{Nome das variáveis}:

  \begin{itemize}
  \tightlist
  \item
    WDS = número de palavras na propaganda
  \item
    SEN = número de frases na propaganda
  \item
    3SYL = número de palavras com 3 ou mais sílabas no anúncio
  \item
    MAG = revista (1 a 9 como na página anterior)
  \item
    GROUP = nível educacional (1 a 3 como na página anterior)
  \end{itemize}
\item
  Dê uma olhada nos dados usando \textbf{Rstudio}:
\end{itemize}

\begin{Shaded}
\begin{Highlighting}[]
\NormalTok{magAds <-}\StringTok{ }\KeywordTok{read.delim}\NormalTok{(}\StringTok{"C:/Users/fsabino/Desktop/Codes/papers/Introductory_Stat_I/notebook/datasets_ads.txt"}\NormalTok{)}
\KeywordTok{head}\NormalTok{(magAds)}
\end{Highlighting}
\end{Shaded}

\begin{verbatim}
##   WDS SEN X3SYL MAG GROUP
## 1 205   9    34   1     1
## 2 203  20    21   1     1
## 3 229  18    37   1     1
## 4 208  16    31   1     1
## 5 146   9    10   1     1
## 6 230  16    24   1     1
\end{verbatim}

\begin{itemize}
\tightlist
\item
  Os nomes das variáveis estão na linha superior. Não é permitido
  começar o nome de uma variável com um dígito, então um \texttt{X} foi
  adicionado em \texttt{X3SYL}.
\end{itemize}

\subsection{Tipos de Dados}\label{tipos-de-dados}

\subsubsection{Variáveis Quantitativas}\label{variaveis-quantitativas}

\begin{itemize}
\tightlist
\item
  Medições contém valores numéricos.
\item
  Os dados quantitativos geralmente surgem das seguintes maneiras:

  \begin{itemize}
  \tightlist
  \item
    \textbf{Variáveis contínuas}: \texttt{medições} de, por exemplo,
    tempo de espera em uma fila, receitas, preços de ações, etc.
  \item
    \textbf{Variáveis discretas}: \texttt{contagens} de, por exemplo,
    palavras em um texto, acessos de um website, números de chegadas em
    uma fila em uma hora, etc.
  \end{itemize}
\item
  Medidas como esta têm um escala bem definida e no \textbf{R} elas são
  armazenadas como numéricas (\textbf{numeric}). 
\end{itemize}

\subsubsection{Variáveis
Categóricas/Qualitativas}\label{variaveis-categoricasqualitativas}

\begin{itemize}
\tightlist
\item
  A medida é um fator proveniente de um conjunto de determinadas
  categorias. Exemplos: sexo (masculino/feminino), classe social, escore
  de satisfação (baixo/médio/alto), etc.
\item
  A medida é normalmente armazenada (o que é altamente recomendável)
  como um fator (\textbf{factor}) no \textbf{R}. As categorias possíveis
  são chamadas de níveis (\textbf{levels}). Examplo: os níveis do fator
  ``sexo'' são masculino/feminino.
\item
  Fatores têm duas possíveis escalas:

  \begin{itemize}
  \tightlist
  \item
    \textbf{Escala Nominal}: Não há ordenação natural entre os níveis
    dos fatores. Exemplos: sexo e cor do cabelo.
  \item
    \textbf{Escala Ordinal}: Há uma ordenação natural entre os níveis
    dos fatores. Exemplos: classe social e escore de satisfação. Um
    fator no \textbf{R} pode ter um chamado atributo
    (\textbf{attribute}) atribuído, informando que a escala é ordinal
    (veja a função ordered()).
  \end{itemize}
\end{itemize}

\section{População e Amostra}\label{populacao-e-amostra}

\subsection{Objetivo da Estatística}\label{objetivo-da-estatistica}

\begin{itemize}
\tightlist
\item
  O objetivo da Estatística é ``dizer algo'' sobre a população.
\item
  Tipicamente, isso é feito utilizando as informações de uma amostra
  aleatória retirada da população de interesse.
\item
  Antes de retirar a amostra podemos ter alguma hipótese sobre a
  população. A amostra é então analisada como o objetivo de testar esta
  hipótese.
\item
  O processo de fazer conclusões para uma população com base em uma
  amostra é chamado de \textbf{inferência estatística}.
\end{itemize}

\subsection{\texorpdfstring{Seleção
\textbf{aleatória}}{Seleção aleatória}}\label{selecao-aleatoria}

\begin{itemize}
\tightlist
\item
  Exemplo: Para os dados das revistas:

  \begin{itemize}
  \tightlist
  \item
    Primeiro nós selecionamos \textbf{aleatoriamente} 3 revistas de cada
    grupo.
  \item
    Na sequência, nós selecionamos, \textbf{aleatoriamente}, 6 anúncios
    de cada revista.
  \item
    Um detalhe importante é que a seleção é feita de maneira
    completamente \textbf{aleatória}, i.e.

    \begin{itemize}
    \tightlist
    \item
      cada revista dentro de um grupo tem a mesma chance de ser
      escolhida e
    \item
      cada anúncio dentro de uma revista tem a mesma chance de ser
      escolhido.
    \end{itemize}
  \end{itemize}
\item
  No que veremos neste curso é fundamental que os dados coletados
  respeitem o princípio da aleatoriedade. Sempre que utilizarmos a
  palavra \textbf{amostra} daqui em diante, estaremos nos referindo a
  uma a.a. (amostra aleatória).
\item
  Mais geralmente:

  \begin{itemize}
  \tightlist
  \item
    Nós temos uma \textbf{população} de objetos.
  \item
    Nós escolhemos aleatoriamente \(n\) destes objetos, e do \(j\)-ésimo
    objeto nós obtemos a medição \(y_j\), \(j=1,2,\ldots,n\).
  \item
    As medições \(y_1, y_2, \ldots, y_n\) são então chamadas de
    \textbf{amostra}. Só uma amostra (que contém \(n\) elementos) e não
    várias amostras.
  \end{itemize}
\item
  Se nós, por exemplo, estivermos medindo a qualidade da água 4 vezes em
  um ano é uma má ideia coletarmos dados apenas com tempo bom. A
  amostragem escolhida ao longo do tempo não pode ser influenciada por
  algo que possa influenciar a medida em si.
\end{itemize}

\section{Tabelas de agrupamento e
frequência}\label{tabelas-de-agrupamento-e-frequencia}

\subsection{\texorpdfstring{Dividir toda a gama de valores em uma série
de intervalos:
``Binning''}{Dividir toda a gama de valores em uma série de intervalos: Binning}}\label{dividir-toda-a-gama-de-valores-em-uma-serie-de-intervalos-binning}

\begin{itemize}
\tightlist
\item
  A função \texttt{cut} irá dividir o intervalo de uma variável numérica
  em vários intervalos de tamanho igual e registrar a qual intervalo
  pertence cada observação. Por exemplo, para a variável \texttt{X3SYL}
  (o número de palavras com mais de 3 sílabas):
\end{itemize}

\begin{Shaded}
\begin{Highlighting}[]
\CommentTok{# Antes de 'cortar':}
\NormalTok{magAds}\OperatorTok{$}\NormalTok{X3SYL[}\DecValTok{1}\OperatorTok{:}\DecValTok{5}\NormalTok{]}
\end{Highlighting}
\end{Shaded}

\begin{verbatim}
## [1] 34 21 37 31 10
\end{verbatim}

\begin{Shaded}
\begin{Highlighting}[]
\CommentTok{# Após 'cortar' (dividir) em 4 intervalos:}
\NormalTok{syll <-}\StringTok{ }\KeywordTok{cut}\NormalTok{(magAds}\OperatorTok{$}\NormalTok{X3SYL, }\DecValTok{4}\NormalTok{)}
\NormalTok{syll[}\DecValTok{1}\OperatorTok{:}\DecValTok{5}\NormalTok{]}
\end{Highlighting}
\end{Shaded}

\begin{verbatim}
## [1] (32.2,43]     (10.8,21.5]   (32.2,43]     (21.5,32.2]   (-0.043,10.8]
## Levels: (-0.043,10.8] (10.8,21.5] (21.5,32.2] (32.2,43]
\end{verbatim}

\begin{itemize}
\tightlist
\item
  O resultado é um fator (\texttt{factor}) e os rótulos são os
  intervalos. Os itens personalizados podem ser atribuídos através do
  argumento \texttt{labels} (rótulos):
\end{itemize}

\begin{Shaded}
\begin{Highlighting}[]
\NormalTok{labs <-}\StringTok{ }\KeywordTok{c}\NormalTok{(}\StringTok{"poucas"}\NormalTok{, }\StringTok{"algumas"}\NormalTok{, }\StringTok{"muitas"}\NormalTok{, }\StringTok{"demais"}\NormalTok{)}
\NormalTok{syll <-}\StringTok{ }\KeywordTok{cut}\NormalTok{(magAds}\OperatorTok{$}\NormalTok{X3SYL, }\DecValTok{4}\NormalTok{, }\DataTypeTok{labels =}\NormalTok{ labs) }\CommentTok{# Nota: isso sobreescreverá a variável 'syll' definida acima}
\NormalTok{syll[}\DecValTok{1}\OperatorTok{:}\DecValTok{5}\NormalTok{]}
\end{Highlighting}
\end{Shaded}

\begin{verbatim}
## [1] demais  algumas demais  muitas  poucas 
## Levels: poucas algumas muitas demais
\end{verbatim}

\begin{Shaded}
\begin{Highlighting}[]
\NormalTok{magAds}\OperatorTok{$}\NormalTok{syll <-}\StringTok{ }\NormalTok{syll }\CommentTok{# Adicionando uma nova coluna ao conjunto de dados}
\end{Highlighting}
\end{Shaded}

\subsection{Tabelas}\label{tabelas}

\begin{itemize}
\tightlist
\item
  Para resumir os resultados nós podemos utilizar a função
  \texttt{tally} (contagem) do pacote\texttt{mosaic} (relembre que o
  pacote \textbf{deve ser carregado} escrevendo \texttt{library(mosaic)}
  se você ainda não o fez):
\end{itemize}

\begin{Shaded}
\begin{Highlighting}[]
\KeywordTok{tally}\NormalTok{( }\OperatorTok{~}\StringTok{ }\NormalTok{syll, }\DataTypeTok{data =}\NormalTok{ magAds)}
\end{Highlighting}
\end{Shaded}

\begin{verbatim}
## syll
##  poucas algumas  muitas  demais 
##      26      14      10       4
\end{verbatim}

\begin{itemize}
\tightlist
\item
  Em porcentagem:
\end{itemize}

\begin{Shaded}
\begin{Highlighting}[]
\KeywordTok{tally}\NormalTok{( }\OperatorTok{~}\StringTok{ }\NormalTok{syll, }\DataTypeTok{data =}\NormalTok{ magAds, }\DataTypeTok{format =} \StringTok{"percent"}\NormalTok{)}
\end{Highlighting}
\end{Shaded}

\begin{verbatim}
## syll
##  poucas algumas  muitas  demais 
##    48.1    25.9    18.5     7.4
\end{verbatim}

\begin{itemize}
\tightlist
\item
  Aqui nós usamos uma \texttt{fórmula} (caracterizada pelo til) para
  indicar que nós queremos a variável syll do conjunto de dados
  \texttt{magAds} (sem o til o \textbf{R} iria procurar por uma variável
  global chamada \texttt{syll} caso ela exista (se não existir dará
  certo) e a utilizaria ao invés da que queremos).
\end{itemize}

\subsection{2 fatores: Tabulação
Cruzada}\label{fatores-tabulacao-cruzada}

\begin{itemize}
\tightlist
\item
  Para fazer uma tabela da combinação de dois fatores nós utilizamos a
  função \texttt{tally} novamente:
\end{itemize}

\begin{Shaded}
\begin{Highlighting}[]
\KeywordTok{tally}\NormalTok{( }\OperatorTok{~}\StringTok{ }\NormalTok{syll }\OperatorTok{+}\StringTok{ }\NormalTok{GROUP, }\DataTypeTok{data =}\NormalTok{ magAds)}
\end{Highlighting}
\end{Shaded}

\begin{verbatim}
##          GROUP
## syll       1  2  3
##   poucas   8 11  7
##   algumas  4  2  8
##   muitas   3  5  2
##   demais   3  0  1
\end{verbatim}

\begin{itemize}
\tightlist
\item
  Frequências relativas (em porcentagem) por coluna:
\end{itemize}

\begin{Shaded}
\begin{Highlighting}[]
\KeywordTok{tally}\NormalTok{( }\OperatorTok{~}\StringTok{ }\NormalTok{syll }\OperatorTok{|}\StringTok{ }\NormalTok{GROUP, }\DataTypeTok{data =}\NormalTok{ magAds, }\DataTypeTok{format =} \StringTok{"percent"}\NormalTok{)}
\end{Highlighting}
\end{Shaded}

\begin{verbatim}
##          GROUP
## syll         1    2    3
##   poucas  44.4 61.1 38.9
##   algumas 22.2 11.1 44.4
##   muitas  16.7 27.8 11.1
##   demais  16.7  0.0  5.6
\end{verbatim}

\begin{itemize}
\tightlist
\item
  A tabela acima mostra, por exemplo, qual a porcentagem de anúncios no
  grupo 1 que tem `poucas', `algumas', `muitas' ou `demais' com mais de
  3 sílabas.
\end{itemize}

\section{Gráficos}\label{graficos}

\subsection{Gráfico de barras}\label{grafico-de-barras}

\begin{itemize}
\tightlist
\item
  Para criar um gráfico de barras com os dados da tabela nós usamos a
  função \texttt{gf\_bar} do pacote \texttt{mosaic}. Para cada nível do
  fator uma caixa é desenhada com a altura proporcional a frequência
  (contagem) daquele nível.
\end{itemize}

\begin{Shaded}
\begin{Highlighting}[]
\KeywordTok{gf_bar}\NormalTok{( }\OperatorTok{~}\StringTok{ }\NormalTok{syll, }\DataTypeTok{data =}\NormalTok{ magAds)}
\end{Highlighting}
\end{Shaded}

\includegraphics{lecture_descriptive_StatI_files/figure-latex/bargraph-1.pdf}

\begin{itemize}
\tightlist
\item
  O gráfico de barras também pode ser dividido por grupo:
\end{itemize}

\begin{Shaded}
\begin{Highlighting}[]
\KeywordTok{gf_bar}\NormalTok{( }\OperatorTok{~}\StringTok{ }\NormalTok{syll }\OperatorTok{|}\StringTok{ }\NormalTok{GROUP, }\DataTypeTok{data =}\NormalTok{ magAds)}
\end{Highlighting}
\end{Shaded}

\includegraphics{lecture_descriptive_StatI_files/figure-latex/bargraph_grouped-1.pdf}

\subsection{Os dados de Ericksen}\label{os-dados-de-ericksen}

\begin{itemize}
\tightlist
\item
  Descrição dos dados:
  \href{http://www.rdocumentation.org/packages/car/functions/Ericksen}{Ericksen
  1980 U.S. Census Undercount}.
\item
  Este conjunto de dados contém as seguintes variáveis:

  \begin{itemize}
  \tightlist
  \item
    \texttt{minority}: Percentual de negros ou hispânicos.
  \item
    \texttt{crime}: Taxa de crimes graves por 1000 indivíduos na
    população.
  \item
    \texttt{poverty}: Percentual de pobres.
  \item
    \texttt{language}: Percentual com dificuldade em falar ou escrever
    Inglês.
  \item
    \texttt{highschool}: Percentual com idade igual ou superior a 25
    anos que não terminou o ensino médio.
  \item
    \texttt{housing}: Percentual de habitação em pequenos edifícios de
    unidades múltiplas.
  \item
    \texttt{city}: Um fator com níveis: \texttt{city} (cidade principal)
    ou \texttt{state} (estado or estado-resto).
  \item
    \texttt{conventional}: Percentual de domicílios contados por
    enumeração pessoal convencional.
  \item
    \texttt{undercount}: Estimativa preliminar de subentendimento
    percentual.
  \end{itemize}
\item
  Os dados de Ericksen têm 66 linhas/observações e 9 colunas/variáveis.
\item
  As observações são medidas em 16 grandes cidades, as partes restantes
  dos estados em que essas cidades estão localizadas, e os outros
  estados dos EUA.
\end{itemize}

\begin{Shaded}
\begin{Highlighting}[]
\NormalTok{Ericksen <-}\StringTok{ }\KeywordTok{read.delim}\NormalTok{(}\StringTok{"C:/Users/fsabino/Desktop/Codes/papers/Introductory_Stat_I/notebook/datasets_Ericksen.txt"}\NormalTok{)}
\KeywordTok{head}\NormalTok{(Ericksen)}
\end{Highlighting}
\end{Shaded}

\begin{verbatim}
##   minority crime poverty language highschool housing  city conventional
## 1     26.1    49      19      0.2         44     7.6 state            0
## 2      5.7    62      11      1.7         18    23.6 state          100
## 3     18.9    81      13      3.2         28     8.1 state           18
## 4     16.9    38      19      0.2         44     7.0 state            0
## 5     24.3    73      10      5.0         26    11.8 state            4
## 6     15.2    73      10      1.2         21     9.2 state           19
##   undercount
## 1      -0.04
## 2       3.35
## 3       2.48
## 4      -0.74
## 5       3.60
## 6       1.34
\end{verbatim}

\begin{itemize}
\tightlist
\item
  Quer fazer um histograma para a taxa de criminalidade - como?
\end{itemize}

\subsection{Histograma (usado para variáveis
quantitativas)}\label{histograma-usado-para-variaveis-quantitativas}

\begin{itemize}
\tightlist
\item
  Como fazer um histograma para alguma variável \texttt{x}:

  \begin{itemize}
  \tightlist
  \item
    Divida o intervalo do valor mínimo de \texttt{x} para o valor máximo
    de\texttt{x} em um número apropriado de sub-intervalos de tamanho
    igual.
  \item
    Desenhe uma caixa em cada sub-intervalo, sendo a altura proporcional
    ao número de observações no subintervalo.
  \end{itemize}
\item
  Histograma de taxas de criminalidade para os dados de Ericksen
\end{itemize}

\begin{Shaded}
\begin{Highlighting}[]
\KeywordTok{gf_histogram}\NormalTok{( }\OperatorTok{~}\StringTok{ }\NormalTok{crime, }\DataTypeTok{data =}\NormalTok{ Ericksen)}
\end{Highlighting}
\end{Shaded}

\includegraphics{lecture_descriptive_StatI_files/figure-latex/hist-1.pdf}

Questão: Explique como o histograma é construído.

\section{Resumo de Variáveis
Quantitativas}\label{resumo-de-variaveis-quantitativas}

\subsection{Medidas de centro dos dados (tendência central/posição):
Média, Mediana e
Moda}\label{medidas-de-centro-dos-dados-tendencia-centralposicao-media-mediana-e-moda}

\begin{itemize}
\tightlist
\item
  Retornemos ao exemplo de anúncios da revista (\texttt{WDS} = número de
  palavras no anúncio). Uma série de resumos numéricos para \texttt{WDS}
  pode ser encontrada usando a função \texttt{favstats}:
\end{itemize}

\begin{Shaded}
\begin{Highlighting}[]
\KeywordTok{favstats}\NormalTok{( }\OperatorTok{~}\StringTok{ }\NormalTok{WDS, }\DataTypeTok{data =}\NormalTok{ magAds)}
\end{Highlighting}
\end{Shaded}

\begin{verbatim}
##  min Q1 median  Q3 max mean sd  n missing
##   31 69     96 202 230  123 66 54       0
\end{verbatim}

\begin{itemize}
\tightlist
\item
  Os valores observados da variável \texttt{WDS} são \(y_1=205\),
  \(y_2=203,\ldots,y_n=208\), onde existe um total de \(n=54\) valores.
  Conforme definido anteriormente, isso constitui uma \textbf{amostra}.
\item
  123 é a \textbf{média} da amostra, que é calculada por \[
    \bar{y}=\frac{1}{n}\sum_{i=1}^n y_i
  \] Nós chamamos \(\bar{y}\) de \textbf{média amostral}.
\end{itemize}

\begin{Shaded}
\begin{Highlighting}[]
\KeywordTok{mean}\NormalTok{(y)}
\end{Highlighting}
\end{Shaded}

\begin{verbatim}
## [1] 123
\end{verbatim}

\begin{itemize}
\item
  A média é o ponto de equilíbro dos dados.
\item
  A média é sensível a valores extremos.
\item
  \textbf{mediana} = 96 é o percentil 50, i.e.~o valor que divide a
  amostra em 2 grupos de igual tamanho.
\end{itemize}

\begin{Shaded}
\begin{Highlighting}[]
\KeywordTok{median}\NormalTok{(y)}
\end{Highlighting}
\end{Shaded}

\begin{verbatim}
## [1] 96
\end{verbatim}

\begin{verbatim}
* Veja como calcular a mediana (e qualquer outro percentil) na página 75 do livro "Estatística Aplicada à Administração e Economia" (está na página do curso no moodle). Resolva os exemplos das páginas 75-77.
\end{verbatim}

\begin{itemize}
\item
  A \textbf{mediana} é robusta a valores extremos.
\item
  É uma medida mais apropriada quando trabalhamos com dados
  assimétricos.
\item
  Uma propriedade importante da \textbf{média} e da \textbf{mediana} é
  que elas têm a mesma unidade de medida que as observações.
\item
  \textbf{moda} = 208 é o valor mais frequente do banco de dados.
\end{itemize}

\begin{Shaded}
\begin{Highlighting}[]
\KeywordTok{names}\NormalTok{(}\KeywordTok{sort}\NormalTok{(}\OperatorTok{-}\KeywordTok{table}\NormalTok{(y)))[}\DecValTok{1}\NormalTok{]}
\end{Highlighting}
\end{Shaded}

\begin{verbatim}
## [1] "208"
\end{verbatim}

\begin{itemize}
\tightlist
\item
  Exercício: Faça o exercício 1 da página 24 do livro Estatística (Costa
  Neto). A resposta está na página 258.
\end{itemize}

\subsection{Medidas de variabilidade: amplitude, amplitude
interquartílica, variância, desvio padrão, e coeficiente de
variação}\label{medidas-de-variabilidade-amplitude-amplitude-interquartilica-variancia-desvio-padrao-e-coeficiente-de-variacao}

\begin{itemize}
\tightlist
\item
  Nós queremos saber ``Quanto as observações estão desviadas do seu
  valor central?''

  \begin{itemize}
  \tightlist
  \item
    Ao olhar os dados e gráficos podemos ter uma sensação disto.
  \item
    Porém, é comum estarmos interessados em um número para que possamos
    comparar as distribuições amostrais.
  \end{itemize}
\item
  \textbf{Amplitude} é a diferença entre o maior (máximo) e o menor
  (mínimo) valor.

  \begin{itemize}
  \tightlist
  \item
    Ela só usa dois valores para o seu cálculo, isto é, não leva todos
    em consideração.
  \item
    Como trabalhamos com uma amostra, a amplitude que encontraremos será
    a amostral, isto é, em geral, temos uma subestimativa da verdadeira
    amplitude.
  \end{itemize}
\item
  A \textbf{amplitude interquartílica} é a diferença entre os valores do
  terceiro quartil e do primeiro quartil, isto é, \(Q_{3} - Q_{1}\).

  \begin{itemize}
  \tightlist
  \item
    Ela utiliza 50\% dos valores para o seu cálculo.
  \end{itemize}
\item
  The \textbf{variância (empírica)} é a média dos desvios quadrados em
  relação à média: \[
      s^2=\frac{1}{n-1}\sum_{i=1}^n (y_i-\bar{y})^2.
    \]
\item
  \textbf{sd} \(=\) \textbf{desvio padrão} \(= s=\sqrt{s^2}\).
\item
  Nota:~Por exemplo, se as observações são medidas em metros,~a unidade
  de medida da \textbf{variância} será \(\text{metro}^2\) o que
  usualmente dificulta a interpretação. Por outro lado, o \textbf{desvio
  padrão} tem a mesma unidade de medida das observações. 
\item
  O \textbf{coeficiente de variação} (CV) é uma medida adimensional que
  serve para comparar a variabilidade de variáveis medidas em diferentes
  unidade de medida ou cujas médias e desvios-padrão sejam muito
  diferentes (mesma unidade de medida e grandezas muito diferentes).
  Portanto, o \textbf{CV} é uma medida de variabilidade relativa, ao
  contrário das demais vistas que são medidas de variabilidade
  absolutas. Define-se o CV como a razão entre o desvio-padrão e a média
  (frequentemente é multiplicado por 100\% para ser representado como
  uma variação percentual).
\end{itemize}

\[ CV = \frac{s}{\bar{x}} \] * Exercício: Faça o exercício 4 da página
32 do livro Estatística (Costa Neto). O objetivo do exercício é treinar
o cálculo. A resposta está na página 258.

\subsection{Cálculo da média, mediana, amplitude interquartílica e
desvio-padrão usando a função favstats do pacote
mosaic}\label{calculo-da-media-mediana-amplitude-interquartilica-e-desvio-padrao-usando-a-funcao-favstats-do-pacote-mosaic}

\begin{itemize}
\tightlist
\item
  Medidas Resumo de \texttt{WDS}:
\end{itemize}

\begin{Shaded}
\begin{Highlighting}[]
\KeywordTok{favstats}\NormalTok{( }\OperatorTok{~}\StringTok{ }\NormalTok{WDS, }\DataTypeTok{data =}\NormalTok{ magAds)}
\end{Highlighting}
\end{Shaded}

\begin{verbatim}
##  min Q1 median  Q3 max mean sd  n missing
##   31 69     96 202 230  123 66 54       0
\end{verbatim}

\textbf{Exercício}: Interprete os resultados acima.

\subsection{Uma palavra sobre
terminologia}\label{uma-palavra-sobre-terminologia}

\begin{itemize}
\tightlist
\item
  \textbf{Desvio padrão}:~uma medida de variabilidade de uma variável na
  amostra (ou população).
\item
  \textbf{Erro padrão}:~uma medida de variabilidade de uma estimativa
  (um particular valor de uma função da amostra). Por exemplo, uma
  medida de variabilidade da média amostral.
\end{itemize}

\subsection{Uma regra empírica (veremos detalhes mais à
frente)}\label{uma-regra-empirica-veremos-detalhes-mais-a-frente}

\includegraphics{lecture_descriptive_StatI_files/figure-latex/empRule-1.pdf}

Se o histograma com base na amostra parece uma função em forma de sino,
então

\begin{itemize}
\tightlist
\item
  cerca de 68\% das observações estão entre \(\bar{y}-s\) e
  \(\bar{y}+s\).
\item
  acerca de 95\% das observações estão entre \(\bar{y}-2s\) e
  \(\bar{y}+2s\).
\item
  Todas ou quase todas as observações (99.7\%) estão entre
  \(\bar{y}-3s\) e \(\bar{y}+3s\).
\end{itemize}

\subsection{Percentis}\label{percentis}

\begin{itemize}
\item
  \textbf{O \(p\)-ésimo percentil} é um valor tal que pelo menos \(p\)\%
  das observações são menores ou iguais a esse valor e pelo menos.
\item
  Veja como calcular os percentis nas páginas 75-77 do livro texto.
\end{itemize}

\subsection{Mediana, quartis e amplitude
interquartílica}\label{mediana-quartis-e-amplitude-interquartilica}

Recordando

\begin{Shaded}
\begin{Highlighting}[]
\KeywordTok{favstats}\NormalTok{( }\OperatorTok{~}\StringTok{ }\NormalTok{WDS, }\DataTypeTok{data =}\NormalTok{ magAds)}
\end{Highlighting}
\end{Shaded}

\begin{verbatim}
##  min Q1 median  Q3 max mean sd  n missing
##   31 69     96 202 230  123 66 54       0
\end{verbatim}

\begin{itemize}
\tightlist
\item
  50-percentil = 96 é a \textbf{mediana} e é uma medida de tendência
  central/posição (centro dos dados).
\item
  0-percentil = 31 é o valor \textbf{mínimo}.
\item
  25-percentil = 69 é o \textbf{primeiro quartil} ou \textbf{quartil
  inferior} (Q1).~Mediana dos 50\% menores valores.
\item
  75-percentil = 201.5 é o \textbf{terceiro quartil} ou \textbf{quartil
  superior} (Q3).~Mediana dos 50\% maiores valores.
\item
  100-percentil = 230 é o valor \textbf{máximo}.
\item
  \textbf{Amplitude Interquartílica (IQR)}:~uma medida de variabilidade
  dada pela diferença entre o quartil superior e o quartil inferior:
  201.5 - 69 = 132.5.
\end{itemize}

\section{Mais gráficos}\label{mais-graficos}

\subsection{Box plots}\label{box-plots}

Como desenhar um box plot:

\begin{itemize}
\tightlist
\item
  Box:

  \begin{itemize}
  \tightlist
  \item
    Calcule a mediana, e os quartis inferior e superior.
  \item
    Trace uma linha na mediana e desenhe uma caixa entre os quartis
    superior e inferior.
  \item
    Calcule a amplitude interquartílica e a chame de IQR.
  \item
    Calcule os seguintes valores:

    \begin{itemize}
    \tightlist
    \item
      L = quartil inferior - 1.5*IQR
    \item
      U = quartil superior + 1.5*IQR
    \end{itemize}
  \item
    Desenhe uma linha ligando o quartil inferior até a menor medida que
    seja maior do que \emph{L}.
  \item
    Similarmente, desenhe uma linha ligando o quartil superior até a
    maior medida que seja inferior a \emph{U}.
  \item
    Regras de decisão
  \item
    Max(\(X_\left[1 \right]\), L)
  \item
    Min(\(X_\left[n \right]\), U)
  \end{itemize}
\item
  Outliers: Observações com valor menor do que \emph{L} ou maior do que
  \emph{U} são desenhadas como círculos.
\end{itemize}

\emph{Nota: As caixas são fechadas (em inglês, as extremidades são
chamadas de ``Whiskers'') no mínimo e no máximo das observações que não
são consideradas outliers.}

\begin{center}\rule{0.5\linewidth}{\linethickness}\end{center}

\subsubsection{Boxplot para os dados de
Ericksen}\label{boxplot-para-os-dados-de-ericksen}

Boxplot das taxas de pobreza separadamente para cidados e estados
(variável \texttt{city}):

\begin{Shaded}
\begin{Highlighting}[]
\KeywordTok{gf_boxplot}\NormalTok{(poverty }\OperatorTok{~}\StringTok{ }\NormalTok{city, }\DataTypeTok{data =}\NormalTok{ Ericksen)}
\end{Highlighting}
\end{Shaded}

\includegraphics{lecture_descriptive_StatI_files/figure-latex/boxplot-1.pdf}

\begin{itemize}
\tightlist
\item
  Parece haver mais pobreza nas cidades.
\item
  Um único estado difere notoriamente dos outros com alta taxa de
  pobreza.
\end{itemize}

\subsection{\texorpdfstring{2 variáveis quantitativas variables: Gráfico
de dispersão (``Scatter
plot'')}{2 variáveis quantitativas variables: Gráfico de dispersão (Scatter plot)}}\label{variaveis-quantitativas-variables-grafico-de-dispersao-scatter-plot}

Para duas variáveis quantitativas, um gráfico frequentemente utilizado é
o de dispersão:

\begin{Shaded}
\begin{Highlighting}[]
\KeywordTok{gf_point}\NormalTok{(poverty }\OperatorTok{~}\StringTok{ }\NormalTok{highschool, }\DataTypeTok{data =}\NormalTok{ Ericksen)}
\end{Highlighting}
\end{Shaded}

\includegraphics{lecture_descriptive_StatI_files/figure-latex/scatter0-1.pdf}

Isto pode ser colorido ou dividido de acordo com o valor de
\texttt{city}:

\begin{Shaded}
\begin{Highlighting}[]
\KeywordTok{gf_point}\NormalTok{(poverty }\OperatorTok{~}\StringTok{ }\NormalTok{highschool }\OperatorTok{|}\StringTok{ }\NormalTok{city, }\DataTypeTok{data =}\NormalTok{ Ericksen)}
\end{Highlighting}
\end{Shaded}

\includegraphics{lecture_descriptive_StatI_files/figure-latex/unnamed-chunk-29-1.pdf}

\begin{Shaded}
\begin{Highlighting}[]
\KeywordTok{gf_point}\NormalTok{(poverty }\OperatorTok{~}\StringTok{ }\NormalTok{highschool, }\DataTypeTok{col =} \OperatorTok{~}\NormalTok{city, }\DataTypeTok{data =}\NormalTok{ Ericksen)}
\end{Highlighting}
\end{Shaded}

\includegraphics{lecture_descriptive_StatI_files/figure-latex/unnamed-chunk-29-2.pdf}

Se nos quisermos adicionar uma linha de regressão (uma equação da reta
neste caso) nós podemos usar as funções abaixo:

\begin{Shaded}
\begin{Highlighting}[]
\KeywordTok{gf_point}\NormalTok{(poverty }\OperatorTok{~}\StringTok{ }\NormalTok{highschool, }\DataTypeTok{col =} \OperatorTok{~}\NormalTok{city, }\DataTypeTok{data =}\NormalTok{ Ericksen) }\OperatorTok\StringTok{ }\KeywordTok{gf_lm}\NormalTok{()}
\end{Highlighting}
\end{Shaded}

\includegraphics{lecture_descriptive_StatI_files/figure-latex/scatter05-1.pdf}

\subsection{Assimetria e Curtose}\label{assimetria-e-curtose}

\begin{itemize}
\tightlist
\item
  O conteúdo pode ser estudado nas páginas 30-31 do livro
  ``Estatística'' (Costa Neto).
\end{itemize}

\subsection{Covariância e Correlação}\label{covariancia-e-correlacao}

\begin{verbatim}
* O conteúdo pode ser estudado nas páginas 98-104 do livro "Estatística Aplicada à Administração e Economia" (edição que está no moodle).
\end{verbatim}

\section{Apêndice}\label{apendice}

\subsection{Recodificando variáveis}\label{recodificando-variaveis}

\begin{itemize}
\tightlist
\item
  A função \texttt{factor} converterá diretamente um vetor em uma
  variável qualitativa (escala nominal).~Por exemplo:
\end{itemize}

\begin{Shaded}
\begin{Highlighting}[]
\KeywordTok{head}\NormalTok{(magAds}\OperatorTok{$}\NormalTok{GROUP)}
\end{Highlighting}
\end{Shaded}

\begin{verbatim}
## [1] 1 1 1 1 1 1
\end{verbatim}

\begin{Shaded}
\begin{Highlighting}[]
\KeywordTok{class}\NormalTok{(magAds}\OperatorTok{$}\NormalTok{GROUP)}
\end{Highlighting}
\end{Shaded}

\begin{verbatim}
## [1] "integer"
\end{verbatim}

\begin{Shaded}
\begin{Highlighting}[]
\NormalTok{f <-}\StringTok{ }\KeywordTok{factor}\NormalTok{(magAds}\OperatorTok{$}\NormalTok{GROUP)}
\KeywordTok{class}\NormalTok{(f)}
\end{Highlighting}
\end{Shaded}

\begin{verbatim}
## [1] "factor"
\end{verbatim}

\begin{Shaded}
\begin{Highlighting}[]
\CommentTok{# magAds$GROUP <- f}
\CommentTok{# head(magAds$GROUP)}
\end{Highlighting}
\end{Shaded}

\begin{itemize}
\tightlist
\item
  Desta forma, os números são substituídos por rótulos mais informativos
  descrevendo o nível educacional.
\end{itemize}

\section{Apontar e clicar no gráfico}\label{apontar-e-clicar-no-grafico}

\subsection{\texorpdfstring{\texttt{mplot}}{mplot}}\label{mplot}

\begin{itemize}
\tightlist
\item
  Se os pacotes \texttt{mosaic} e \texttt{manipulate} forem instalados e
  estiverem carregados, nós podemos construir gráficos usando a função
  \texttt{mplot} simplesmente apontando e clicando.
\item
  Usando \texttt{mplot} você pode fazer alterações pressionando o botão
  de configurações (uma roda dentada) no canto superior esquerdo da
  janela gráfica.
\end{itemize}

\begin{Shaded}
\begin{Highlighting}[]
\KeywordTok{mplot}\NormalTok{(Ericksen)}
\end{Highlighting}
\end{Shaded}

\begin{itemize}
\tightlist
\item
  No final, você pode pressionar ``Mostrar expressão'' (Show expression)
  para obter o código.
\end{itemize}


\end{document}
