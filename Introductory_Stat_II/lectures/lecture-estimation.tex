\documentclass[]{article}
\usepackage{lmodern}
\usepackage{amssymb,amsmath}
\usepackage{ifxetex,ifluatex}
\usepackage{fixltx2e} % provides \textsubscript
\ifnum 0\ifxetex 1\fi\ifluatex 1\fi=0 % if pdftex
  \usepackage[T1]{fontenc}
  \usepackage[utf8]{inputenc}
\else % if luatex or xelatex
  \ifxetex
    \usepackage{mathspec}
  \else
    \usepackage{fontspec}
  \fi
  \defaultfontfeatures{Ligatures=TeX,Scale=MatchLowercase}
\fi
% use upquote if available, for straight quotes in verbatim environments
\IfFileExists{upquote.sty}{\usepackage{upquote}}{}
% use microtype if available
\IfFileExists{microtype.sty}{%
\usepackage{microtype}
\UseMicrotypeSet[protrusion]{basicmath} % disable protrusion for tt fonts
}{}
\usepackage[margin=1in]{geometry}
\usepackage{hyperref}
\hypersetup{unicode=true,
            pdftitle={Estimação},
            pdfauthor={Fernando B. Sabino da Silva},
            pdfborder={0 0 0},
            breaklinks=true}
\urlstyle{same}  % don't use monospace font for urls
\usepackage{color}
\usepackage{fancyvrb}
\newcommand{\VerbBar}{|}
\newcommand{\VERB}{\Verb[commandchars=\\\{\}]}
\DefineVerbatimEnvironment{Highlighting}{Verbatim}{commandchars=\\\{\}}
% Add ',fontsize=\small' for more characters per line
\usepackage{framed}
\definecolor{shadecolor}{RGB}{248,248,248}
\newenvironment{Shaded}{\begin{snugshade}}{\end{snugshade}}
\newcommand{\KeywordTok}[1]{\textcolor[rgb]{0.13,0.29,0.53}{\textbf{#1}}}
\newcommand{\DataTypeTok}[1]{\textcolor[rgb]{0.13,0.29,0.53}{#1}}
\newcommand{\DecValTok}[1]{\textcolor[rgb]{0.00,0.00,0.81}{#1}}
\newcommand{\BaseNTok}[1]{\textcolor[rgb]{0.00,0.00,0.81}{#1}}
\newcommand{\FloatTok}[1]{\textcolor[rgb]{0.00,0.00,0.81}{#1}}
\newcommand{\ConstantTok}[1]{\textcolor[rgb]{0.00,0.00,0.00}{#1}}
\newcommand{\CharTok}[1]{\textcolor[rgb]{0.31,0.60,0.02}{#1}}
\newcommand{\SpecialCharTok}[1]{\textcolor[rgb]{0.00,0.00,0.00}{#1}}
\newcommand{\StringTok}[1]{\textcolor[rgb]{0.31,0.60,0.02}{#1}}
\newcommand{\VerbatimStringTok}[1]{\textcolor[rgb]{0.31,0.60,0.02}{#1}}
\newcommand{\SpecialStringTok}[1]{\textcolor[rgb]{0.31,0.60,0.02}{#1}}
\newcommand{\ImportTok}[1]{#1}
\newcommand{\CommentTok}[1]{\textcolor[rgb]{0.56,0.35,0.01}{\textit{#1}}}
\newcommand{\DocumentationTok}[1]{\textcolor[rgb]{0.56,0.35,0.01}{\textbf{\textit{#1}}}}
\newcommand{\AnnotationTok}[1]{\textcolor[rgb]{0.56,0.35,0.01}{\textbf{\textit{#1}}}}
\newcommand{\CommentVarTok}[1]{\textcolor[rgb]{0.56,0.35,0.01}{\textbf{\textit{#1}}}}
\newcommand{\OtherTok}[1]{\textcolor[rgb]{0.56,0.35,0.01}{#1}}
\newcommand{\FunctionTok}[1]{\textcolor[rgb]{0.00,0.00,0.00}{#1}}
\newcommand{\VariableTok}[1]{\textcolor[rgb]{0.00,0.00,0.00}{#1}}
\newcommand{\ControlFlowTok}[1]{\textcolor[rgb]{0.13,0.29,0.53}{\textbf{#1}}}
\newcommand{\OperatorTok}[1]{\textcolor[rgb]{0.81,0.36,0.00}{\textbf{#1}}}
\newcommand{\BuiltInTok}[1]{#1}
\newcommand{\ExtensionTok}[1]{#1}
\newcommand{\PreprocessorTok}[1]{\textcolor[rgb]{0.56,0.35,0.01}{\textit{#1}}}
\newcommand{\AttributeTok}[1]{\textcolor[rgb]{0.77,0.63,0.00}{#1}}
\newcommand{\RegionMarkerTok}[1]{#1}
\newcommand{\InformationTok}[1]{\textcolor[rgb]{0.56,0.35,0.01}{\textbf{\textit{#1}}}}
\newcommand{\WarningTok}[1]{\textcolor[rgb]{0.56,0.35,0.01}{\textbf{\textit{#1}}}}
\newcommand{\AlertTok}[1]{\textcolor[rgb]{0.94,0.16,0.16}{#1}}
\newcommand{\ErrorTok}[1]{\textcolor[rgb]{0.64,0.00,0.00}{\textbf{#1}}}
\newcommand{\NormalTok}[1]{#1}
\usepackage{graphicx,grffile}
\makeatletter
\def\maxwidth{\ifdim\Gin@nat@width>\linewidth\linewidth\else\Gin@nat@width\fi}
\def\maxheight{\ifdim\Gin@nat@height>\textheight\textheight\else\Gin@nat@height\fi}
\makeatother
% Scale images if necessary, so that they will not overflow the page
% margins by default, and it is still possible to overwrite the defaults
% using explicit options in \includegraphics[width, height, ...]{}
\setkeys{Gin}{width=\maxwidth,height=\maxheight,keepaspectratio}
\IfFileExists{parskip.sty}{%
\usepackage{parskip}
}{% else
\setlength{\parindent}{0pt}
\setlength{\parskip}{6pt plus 2pt minus 1pt}
}
\setlength{\emergencystretch}{3em}  % prevent overfull lines
\providecommand{\tightlist}{%
  \setlength{\itemsep}{0pt}\setlength{\parskip}{0pt}}
\setcounter{secnumdepth}{5}
% Redefines (sub)paragraphs to behave more like sections
\ifx\paragraph\undefined\else
\let\oldparagraph\paragraph
\renewcommand{\paragraph}[1]{\oldparagraph{#1}\mbox{}}
\fi
\ifx\subparagraph\undefined\else
\let\oldsubparagraph\subparagraph
\renewcommand{\subparagraph}[1]{\oldsubparagraph{#1}\mbox{}}
\fi

%%% Use protect on footnotes to avoid problems with footnotes in titles
\let\rmarkdownfootnote\footnote%
\def\footnote{\protect\rmarkdownfootnote}

%%% Change title format to be more compact
\usepackage{titling}

% Create subtitle command for use in maketitle
\newcommand{\subtitle}[1]{
  \posttitle{
    \begin{center}\large#1\end{center}
    }
}

\setlength{\droptitle}{-2em}
  \title{Estimação}
  \pretitle{\vspace{\droptitle}\centering\huge}
  \posttitle{\par}
  \author{Fernando B. Sabino da Silva}
  \preauthor{\centering\large\emph}
  \postauthor{\par}
  \date{}
  \predate{}\postdate{}


\begin{document}
\maketitle

{
\setcounter{tocdepth}{2}
\tableofcontents
}
\section{Estimação por ponto e por
intervalo}\label{estimaaao-por-ponto-e-por-intervalo}

\subsection{Estimação por ponto e por
intervalo}\label{estimaaao-por-ponto-e-por-intervalo-1}

\begin{itemize}
\item
  Nós queremos investigar hipóteses sobre parâmetros (constantes
  populacionais). Exemplo: a média \(\mu\) e o desvio-padrão
  \(\sigma\).

  \begin{itemize}
  \tightlist
  \item
    Se \(\mu\) for o tempo médio de espera em uma fila, então pode ser
    relevante estudar se o tempo médio excede, por exemplo, a 2
    minutos.
  \end{itemize}
\item
  Com base em uma amostra, nós calculamos uma \textbf{estimativa
  pontual} que de maneira coloquial é o nosso melhor ``chute'' para o
  valor do parâmetro.

  \begin{itemize}
  \tightlist
  \item
    Por exemplo, nós usamos \(\bar{y}\) como uma estimativa para
    \(\mu\) e \(s\) como uma estimativa para \(\sigma\).
  \end{itemize}
\item
  Em geral, nós estamos também interessados em calcular uma
  \textbf{estimativa por intervalo} (também chamada de
  \textbf{intervalo de confiança}). Este intervalo é construído em
  torno da estimativa por ponto.
\item
  A estimativa do parâmetro (e o intervalo de confiança) pode ser
  utilizado para investigar a hipótese.
\end{itemize}

\subsection{Estimadores por ponto:
Viés}\label{estimadores-por-ponto-vias}

\begin{itemize}
\tightlist
\item
  Se queremos estimar a média da população \(\mu\), nós temos
  várias possibilidades (a princípio). Exemplos:

  \begin{itemize}
  \tightlist
  \item
    a média amostral \(\bar{y}\)
  \item
    a média amostral \(y_T\) dos quartis superior (\(Q_{3}\)) e
    inferior (\(Q_{1}\)).
  \end{itemize}
\item
  Vantagem de \(y_T\): Pouca influência de outliers (observações com
  valor muito alto/baixo), i.e.~não teremos praticamente nenhum efeito
  se houver alguns erros no banco de dados.
\item
  Desvantagem de \(y_T\): Se a distribuição da população for
  assimétrica, então \(y_T\) será um estimador \textbf{viesado}, o
  que significa que no longo prazo este estimador sistematicamente será
  acima ou abaixo do verdadeiro valor de \(\mu\).
\item
  Geralmente, nós preferimos que um estimador seja \textbf{não
  viesado}, isto é, que a sua distriuição seja centrada em volta do
  verdadeiro valor do parâmetro.
\item
  Relembre que para uma população com média \(\mu\), a média
  amostral \(\bar{y}\) também tem média \(\mu\), i.e., \(\bar{y}\) é
  um estimador não viesado da média populacional \(\mu\).
\end{itemize}

\subsection{Estimadores por ponto:
Eficiência}\label{estimadores-por-ponto-eficiancia}

\begin{itemize}
\tightlist
\item
  Nós já sabemos (mostramos em aula - verifique se você sabe
  \textbf{provar} isto) que o erro padrão de \(\bar{y}\) é
  \(\frac{\sigma}{\sqrt{n}}\), i.e.~o erro padrão converge para zero
  quando o tamanho da amostra aumenta (você saberia explicar a
  intuição disto?).
\item
  É difícil expressar o erro padrão de \(y_T\), mas é possível
  provar que ele será maior do que o erro padrão de \(\bar{y}\). Este
  é um bom motivo para que, usualmente, \(\bar{y}\) seja um estimador
  preferível.
\item
  Em geral, nós preferimos que um estimador seja \textbf{eficiente}. De
  maneira coloquial, isto significa que o erro padrão converge para
  zero quando o tamanho da amostra aumenta. O estimador \textbf{mais
  eficiente} será aquele que converge para zero a uma velocidade maior.
  No nosso exemplo, para a maioria das populações, \(y_T\) é
  ineficiente.
\end{itemize}

\subsection{Notação}\label{notaaao}

\begin{itemize}
\tightlist
\item
  O símbolo \(\hat{\ }\) acima de um parâmetro é frequentemente
  utilizado para denotar uma estimativa (pontual) de um parâmetro.
  Exemplos:

  \begin{itemize}
  \tightlist
  \item
    estimativa da média populacional:~\(\hat{\mu}=\bar{y}\)
  \item
    estimativa do desvio padrão:~\(\hat{\sigma}=s\)
  \end{itemize}
\item
  Quando observamos uma variável binária (\(0/1\)), o que, por
  exemplo, é utilizada para denotar sim/não ou masculino/feminino,
  então nós usamos a notação \[p=P(Y=1)\] para a proporção da
  população com a característica \(Y=1\).
\item
  A estimativa \(\hat{p}=(y_1+y_2+\ldots+y_n)/n\) é a frequência
  relativa amostral de \(Y=1\).
\end{itemize}

\subsection{Intervalo de Confiança}\label{intervalo-de-confianaa}

\begin{itemize}
\tightlist
\item
  A definição geral de um intervalo de confiança para um parâmetro
  populacional é a seguinte:

  \begin{itemize}
  \tightlist
  \item
    Um \textbf{intervalo de confiança} para um parâmetro é um
    intervalo construído com base na amostra, isto é, ele é
    aleatório (depende da amostra em particular). Esperamos que este
    intervalo contenha o verdadeiro valor do parâmetro (que não é
    aleatório, é uma constante populacional).
  \item
    A ``probabilidade'' de que esta construção produza um intervalo
    que inclua o verdadeiro valor do parâmetro é chamada de
    \textbf{nível de confiança} (ou de cobertura). Tipicamente, o
    nível de confiança escolhido é de \(95\%\).
  \item
    (1-nível de confiança) é chamado de \textbf{nível de
    significância} (neste exemplo, \(1-0.95=0.05\),~i.e.~\(5\%\)).
  \end{itemize}
\item
  Frequentemente, o intervalo é construído como um intervalo
  simétrico em torno de uma estimativa por ponto:

  \begin{itemize}
  \tightlist
  \item
    \textbf{estimativa por ponto \(\pm\)margem de erro}
  \item
    Regra de bolso: Com uma margem de erro de aproximadamente 1.96 vezes
    o erro padrão você obtém um intervalo de confiança de
    aproximadamente \(95\%\).
  \item
    i.e: \textbf{estimativa por ponto} \(\pm\) \textbf{1.96 x erro
    padrão} tem nível de confiança de aproximadamente \(95\%\).
  \end{itemize}
\end{itemize}

\subsection{Intervalo de Confiança para a
Proporção}\label{intervalo-de-confianaa-para-a-proporaao}

\begin{itemize}
\tightlist
\item
  Considere uma população com uma distribuição onde a probabilidade
  de ter uma determinada característica seja \(p\) e a probabilidade de
  não ter seja \(1-p\).
\item
  Quando as categorias \(não/sim\) são denotadas por \(0/1\), i.e.
  \(y\) é \(0\) ou \(1\), a distribuição de \(y\) tem um desvio
  padrão de : \[
    \quad \sigma=\sqrt{p(1-p)}.
    \] Isto é, o desvio padrão não é um parâmetro ``livre'' para
  uma variável \(0/1\), pois o seu valor está diretamente ligado a
  probabilidade \(p\).
\item
  Com uma amostra de tamanho \(n\) o erro padrão de \(\hat{p}\) será
  (dado que \(\hat{p} = \frac{\sum_{i=1}^n y_i}{n}\)): \[
    \quad \sigma_{\hat{p}}=
    \frac{\sigma}{\sqrt{n}} =\sqrt{\frac{p(1-p)}{n}}.
    \]
\item
  Nós não sabemos o valor de \(p\), mas se inserirmos a estimativa
  iremos obter o \textbf{erro padrão estimado} de \(\hat{p}\): \[
    ep=\sqrt{\frac{\hat{p}(1-\hat{p})}{n}}.
    \]
\item
  A regra de bolso nos diz que o intervalo
  \[\hat{p}\pm 1.96\sqrt{\frac{\hat{p}(1-\hat{p})}{n}}\] tem nível de
  confiança de aproximadamente \(95\%\). i.e., antes que os dados sejam
  conhecidos, o intervalo aleatório dado pela fórmula acima tem
  aproximadamente 95\% de ``probabilidade'' de conter o verdadeiro valor
  de \(p\).
\end{itemize}

\begin{center}\rule{0.5\linewidth}{\linethickness}\end{center}

\subsubsection{Exemplo: Estimativa por ponto e por intervalo para a
proporção}\label{exemplo-estimativa-por-ponto-e-por-intervalo-para-a-proporaao}

\begin{itemize}
\tightlist
\item
  Vamos dar uma olhada em dados de uma pesquisa nacional conduzida no
  Chile entre Abril e Maio de 1988. Informações sobre os dados podem
  ser encontradas
  \href{https://www.rdocumentation.org/packages/car/versions/2.1-6/topics/Chile}{aqui}.
\end{itemize}

\begin{Shaded}
\begin{Highlighting}[]
\NormalTok{Chile <-}\StringTok{ }\KeywordTok{read.delim}\NormalTok{(}\StringTok{"C:/Users/fsabino/Desktop/Codes/papers/Introductory_Stat_II/notebook/Chile.txt"}\NormalTok{)}
\end{Highlighting}
\end{Shaded}

\begin{itemize}
\tightlist
\item
  Concentremo-nos na variável \texttt{sex},~i.e.~a distribuição de
  gênero na amostra.
\end{itemize}

\begin{Shaded}
\begin{Highlighting}[]
\KeywordTok{library}\NormalTok{(mosaic)}
\end{Highlighting}
\end{Shaded}

\begin{verbatim}
## Warning: package 'mosaic' was built under R version 3.4.3
\end{verbatim}

\begin{verbatim}
## Warning: package 'dplyr' was built under R version 3.4.3
\end{verbatim}

\begin{verbatim}
## Warning: package 'ggformula' was built under R version 3.4.3
\end{verbatim}

\begin{verbatim}
## Warning: package 'ggplot2' was built under R version 3.4.3
\end{verbatim}

\begin{verbatim}
## Warning: package 'mosaicData' was built under R version 3.4.3
\end{verbatim}

\begin{verbatim}
## Warning: package 'Matrix' was built under R version 3.4.4
\end{verbatim}

\begin{Shaded}
\begin{Highlighting}[]
\KeywordTok{tally}\NormalTok{( }\OperatorTok{~}\StringTok{ }\NormalTok{sex, }\DataTypeTok{data =}\NormalTok{ Chile)}
\end{Highlighting}
\end{Shaded}

\begin{verbatim}
## sex
##    F    M 
## 1379 1321
\end{verbatim}

\begin{Shaded}
\begin{Highlighting}[]
\KeywordTok{tally}\NormalTok{( }\OperatorTok{~}\StringTok{ }\NormalTok{sex, }\DataTypeTok{data =}\NormalTok{ Chile, }\DataTypeTok{format =} \StringTok{"prop"}\NormalTok{)}
\end{Highlighting}
\end{Shaded}

\begin{verbatim}
## sex
##         F         M 
## 0.5107407 0.4892593
\end{verbatim}

\begin{itemize}
\tightlist
\item
  Proporção da população (desconhecida) de mulheres (F),~\(p\).
\item
  Estimativa de \(p\):
  \(\quad \hat{p} = \frac{1379}{1379+1321} = 0.5107\)
\item
  Regra de bolso :
  \(\quad \hat{p} \pm 1.96 \times ep = 0.5107 \pm 2 \sqrt{\frac{0.5107(1-0.5107)}{1379 + 1321}} = (0.4919, 0.5296)\)
  é um intervalo de confiança aproximado de 95\% para \(p\).
\end{itemize}

\begin{center}\rule{0.5\linewidth}{\linethickness}\end{center}

\subsubsection{\texorpdfstring{Exemple: Intervalos de confiança para a
proporção no
\textbf{R}}{Exemple: Intervalos de confiança para a proporção no R}}\label{exemple-intervalos-de-confianaa-para-a-proporaao-no-r}

\begin{itemize}
\tightlist
\item
  \textbf{R} automaticamente calcula o intervalo de confiança para a
  proporção de pessoas do sexo feminino quando nós fazemos um teste
  de hipóteses (voltaremos a isso mais adiante):
\end{itemize}

\begin{Shaded}
\begin{Highlighting}[]
\KeywordTok{prop.test}\NormalTok{( }\OperatorTok{~}\StringTok{ }\NormalTok{sex, }\DataTypeTok{data =}\NormalTok{ Chile, }\DataTypeTok{correct =} \OtherTok{FALSE}\NormalTok{)}
\end{Highlighting}
\end{Shaded}

\begin{verbatim}
## 
##  1-sample proportions test without continuity correction
## 
## data:  Chile$sex  [with success = F]
## X-squared = 1.2459, df = 1, p-value = 0.2643
## alternative hypothesis: true p is not equal to 0.5
## 95 percent confidence interval:
##  0.4918835 0.5295675
## sample estimates:
##         p 
## 0.5107407
\end{verbatim}

\begin{itemize}
\tightlist
\item
  O argumento \texttt{correct\ =\ FALSE} é necessário para pedir ao
  \textbf{R} para fazer uma aproximação para a distribuição normal
  como feito nestas notas. Quando \texttt{correct\ =\ TRUE} (o default)
  uma correção matemática que você não aprendeu se aplica e os
  resultados serão ligeiramente diferentes.
\end{itemize}

\begin{center}\rule{0.5\linewidth}{\linethickness}\end{center}

\subsection{Intervalos de confiança aproximados para a
proporção}\label{intervalos-de-confianaa-aproximados-para-a-proporaao}

\begin{itemize}
\tightlist
\item
  Com base no teorema central do limite (CLT), nós temos :
  \[\hat{p}\approx N \left (p,{\frac{p(1-p)}{n}} \right)\] \textbf{ep}
  \(n\hat{p}\) e \(n(1-\hat{p})\) são grandes o suficiente para que a
  aproximação seja válida (maiores do que \(15\), por exemplo).
\item
  Para construir um intervalo de confiança com nível de confiança
  (aproximado) \(1-\alpha\):

  \begin{enumerate}
  \def\labelenumi{\arabic{enumi})}
  \tightlist
  \item
    Encontre o valor crítico \(z_{crit}\) para o qual a probabilidade
    na cauda superior da distribuição normal seja \(\alpha/2\). 
  \item
    Calcule \(ep=\sqrt{\frac{\hat{p}(1-\hat{p})}{n}}\)
  \item
    Então \(\hat{p}\pm z_{crit}\times ep\) é um intervalo de
    confiança com nível de confiança \(1-\alpha\).
  \end{enumerate}
\end{itemize}

\begin{center}\rule{0.5\linewidth}{\linethickness}\end{center}

\subsubsection{Exemplo: Dados do Chile}\label{exemplo-dados-do-chile}

Para os dados do \texttt{Chile} calcule os intervalos de confiança 99\%
e 95\% para a probabilidade de que uma pessoa seja do sexo feminino:

\begin{itemize}
\tightlist
\item
  Para um nível de confiança de \(99\%\), temos \(\alpha=1\%\) e

  \begin{enumerate}
  \def\labelenumi{\arabic{enumi})}
  \tightlist
  \item
    \(z_{crit}\)=\texttt{qdist("norm",\ 1\ -\ 0.01/2)}=2.576.
  \item
    Sabemos que \(\hat{p}=0.5107\) e \(n=2700\), então
    \(ep = \sqrt{\frac{\hat{p}(1-\hat{p})}{n}} = 0.0096\).
  \item
    Assim, um intervalo de confiança de 99\% é:
    \(\hat{p}\pm z_{crit}\times ep=(0.4859, 0.5355)\).
  \end{enumerate}
\item
  Para um nível de confiança de \(95\%\), temos \(\alpha=5\%\) e

  \begin{enumerate}
  \def\labelenumi{\arabic{enumi})}
  \tightlist
  \item
    \(z_{crit}\)=\texttt{qdist("norm",\ 1\ -\ 0.05/2)}=1.96.
  \item
    Novamente, \(\hat{p}=0.5107\) and \(n=2700\) e assim \(ep=0.0096\).
  \item
    Deste modo, nós encontramos um intervalo de confiança de 95\%:
    \(\hat{p}\pm z_{crit}\times ep=(0.4918, 0.5295)\) (como resultado de
    \texttt{prop.test}).
  \end{enumerate}
\end{itemize}

\subsection{Intervalo de confiança para a média - amostra retirada de
uma população com distribuição
normal}\label{intervalo-de-confianaa-para-a-madia---amostra-retirada-de-uma-populaaao-com-distribuiaao-normal}

\begin{itemize}
\tightlist
\item
  Quando é razoável supor que a distribuição da população é
  normal, nós temos o resultado \textbf{exato} \[
    \bar{y}\sim \texttt{N}\bigg(\mu,\frac{\sigma^{2}}{{n}}\bigg),
    \] i.e.~\(\bar{y}\pm z_{crit}\times \frac{\sigma}{\sqrt{n}}\) não
  é mais apenas um intervalo de confiança aproximado (como no caso da
  proporção - por que é aproximado neste caso?), mas sim um intervalo
  de confiança exato para a média populacional, \(\mu\).
\item
  Na prática, porém, \textbf{nós não conhecemos } \(\sigma\) e ao
  invés disso, nós somos obrigados a utilizar o desvio padrão da
  amostra \(s\) para encontrar o \textbf{erro padrão estimado}
  \(ep=\frac{s}{\sqrt{n}}.\)
\item
  Esta incerteza extra, no entanto, implica que um intervalo de
  confiança exato para a média populacional \(\mu\) não pode ser
  construído usando o escore-\(z\).
\item
  Um intervalo exato ainda pode ser construído usando o chamado
  \textbf{escore-\(t\)}, que além do nível de confiança depende dos
  \textbf{graus de liberdade} (degrees of freedom = df), que neste caso
  são \(df = n-1\). Isto é, o intervalo de confiança toma agora a
  forma \[
    \bar{y}\pm t_{crit}\times ep.
    \]
\item
  Nota: É importante aprender as relações entre as distribuições
  normal, t de Student, qui-quadrado e F. Veja mais detalhes em Costa
  Neto (Estatística) e Casella and Berger (Statistical Inference,
  traduzido para português). Mais detalhes serão vistos em sala de
  aula.
\end{itemize}

\subsection{\texorpdfstring{A distribuição \(t\) e o escore
\(t\)}{A distribuição t e o escore t}}\label{a-distribuiaao-t-e-o-escore-t}

\begin{itemize}
\tightlist
\item
  O cálculo do escore \(t\) é baseado na \textbf{distribuição
  \(t\)}, que é semelhante a distribuição normal padrão \(z\):

  \begin{itemize}
  \tightlist
  \item
    ela é simétrica em torno de zero e é uma função em forma de
    ``sino'', mas
  \item
    como vimos tem caudas mais ``pesadas'' e portanto
  \item
    um desvio padrão maior do que o desvio padrão da distribuição
    normal padrão.
  \item
    Note que o desvio padrão da distribuição \(t\) decaí em função
    de seus \textbf{graus de liberdade} (que denotamos por \(df\)).
  \item
    e quando \(df\) cresce a distribuição \(t\) se aproxima da
    distribuição normal padrão.
  \end{itemize}
\end{itemize}

A expressão da função densidade não será indicada aqui (pode ser
encontrada nos livros sugeridos ou no google). Ao invés disso, a
distribução \(t\) é representada abaixo para \(df =1,2,10\) e
\(\infty\).

\includegraphics{lecture-estimation_files/figure-latex/unnamed-chunk-5-1.pdf}

\begin{center}\rule{0.5\linewidth}{\linethickness}\end{center}

\subsubsection{\texorpdfstring{Cálculo do escore \(t\) no
\textbf{R}}{Cálculo do escore t no R}}\label{calculo-do-escore-t-no-r}

\begin{Shaded}
\begin{Highlighting}[]
\KeywordTok{qdist}\NormalTok{(}\StringTok{"t"}\NormalTok{, }\DataTypeTok{p =} \DecValTok{1} \OperatorTok{-}\StringTok{ }\FloatTok{0.025}\NormalTok{, }\DataTypeTok{df =} \DecValTok{4}\NormalTok{)}
\end{Highlighting}
\end{Shaded}

\includegraphics{lecture-estimation_files/figure-latex/unnamed-chunk-6-1.pdf}

\begin{verbatim}
## [1] 2.776445
\end{verbatim}

\begin{itemize}
\tightlist
\item
  Um escore \(t\) é o quantil (i.e.~o valor no eixo x) para o qual
  temos uma dada probabilidade na \textbf{cauda direita}.
\item
  Para obter, por exemplo, um escore \(t\) correspondente a uma
  probabilidade na cauda direita de 2.5 \% nós temos que procurar o
  quantil 97.5 \% usando \texttt{qdist} with \texttt{p\ =\ 1\ -\ 0.025},
  pois \texttt{qdist} olha a área para o \textbf{lado esquerdo}.
\item
  Os graus de liberdade são determinados pelo tamanho da amostra. No
  exemplo anterior usamos df = 4 para ilustração.
\item
  Como um escore \(t\) para uma probabilidade na cauda direita de 2.5 \%
  é 2.776 e a distribuição \(t\) é simétrica em torno do 0, nós
  temos que uma observação tem probabilidade de 1 - 2 \(\cdot\) 0.025
  = 95 \% de estar entre -2.776 and 2.776 para uma distribuição \(t\)
  com 4 graus de liberdade.
\end{itemize}

\subsection{Exemplo: Intervalo de Confiança para a
média}\label{exemplo-intervalo-de-confianaa-para-a-madia}

\begin{itemize}
\tightlist
\item
  Em estatística I usamos o conjunto de dados de \texttt{Ericksen}.
  Queremos agora construir um intervalo de confiança de \(95\%\) para a
  média da população mean \(\mu\) da variável \texttt{crime}.
\end{itemize}

\begin{Shaded}
\begin{Highlighting}[]
\NormalTok{Ericksen <-}\StringTok{ }\KeywordTok{read.delim}\NormalTok{(}\StringTok{"C:/Users/fsabino/Desktop/Codes/papers/Introductory_Stat_I/notebook/datasets_Ericksen.txt"}\NormalTok{)}
\NormalTok{stats <-}\StringTok{ }\KeywordTok{favstats}\NormalTok{( }\OperatorTok{~}\StringTok{ }\NormalTok{crime, }\DataTypeTok{data =}\NormalTok{ Ericksen)}
\NormalTok{stats}
\end{Highlighting}
\end{Shaded}

\begin{verbatim}
##  min Q1 median Q3 max     mean       sd  n missing
##   25 48     55 73 143 63.06061 24.89107 66       0
\end{verbatim}

\begin{Shaded}
\begin{Highlighting}[]
\KeywordTok{qdist}\NormalTok{(}\StringTok{"t"}\NormalTok{, }\DecValTok{1} \OperatorTok{-}\StringTok{ }\FloatTok{0.025}\NormalTok{, }\DataTypeTok{df =} \DecValTok{66} \OperatorTok{-}\StringTok{ }\DecValTok{1}\NormalTok{, }\DataTypeTok{plot =} \OtherTok{FALSE}\NormalTok{)}
\end{Highlighting}
\end{Shaded}

\begin{verbatim}
## [1] 1.997138
\end{verbatim}

\begin{itemize}
\tightlist
\item
  i.e., nós temos

  \begin{itemize}
  \tightlist
  \item
    \(\bar{y} = 63.061\)
  \item
    \(s = 24.891\)
  \item
    \(n = 66\)
  \item
    \(df = n-1 = 65\)
  \item
    \(t_{crit} = 1.997\).
  \end{itemize}
\item
  O intervalo de confiança é
  \(\bar{y}\pm t_{crit}\frac{s}{\sqrt{n}} =  (56.94 ,  69.18)\)
\item
  Todos estes cálculos podem ser feitos automaticamente no \textbf{R}:
\end{itemize}

\begin{Shaded}
\begin{Highlighting}[]
\KeywordTok{t.test}\NormalTok{( }\OperatorTok{~}\StringTok{ }\NormalTok{crime, }\DataTypeTok{data =}\NormalTok{ Ericksen, }\DataTypeTok{conf.level =} \FloatTok{0.95}\NormalTok{)}
\end{Highlighting}
\end{Shaded}

\begin{verbatim}
## 
##  One Sample t-test
## 
## data:  crime
## t = 20.582, df = 65, p-value < 2.2e-16
## alternative hypothesis: true mean is not equal to 0
## 95 percent confidence interval:
##  56.94162 69.17960
## sample estimates:
## mean of x 
##  63.06061
\end{verbatim}

\subsection{\texorpdfstring{Exemplo: Fazendo vários intervalos de
confiança no
\textbf{R}}{Exemplo: Fazendo vários intervalos de confiança no R}}\label{exemplo-fazendo-varios-intervalos-de-confianaa-no-r}

\begin{itemize}
\tightlist
\item
  Vamos olhar o conjunto de dados \texttt{chickwts} que já vem
  integrado no \textbf{R}.
\item
  \texttt{?chickwts} produz uma página com a seguinte informação
\end{itemize}

\begin{quote}
Um experimento foi conduzido para medir e comparar a eficácia de
vários suplementos alimentares sobre a taxa de crescimento de galinhas
Os filhotes recém-nascidos foram alocados aleatoriamente em seis
grupos, e cada grupo recebeu um suplemento alimeantar diferente. Seus
pesos em gramas após seis semanas são dados juntamente com os tipos de
alimentação.
\end{quote}

\begin{itemize}
\tightlist
\item
  \texttt{chickwts} é um data frame com 71 observações e 2
  variáveis:

  \begin{itemize}
  \tightlist
  \item
    \texttt{weight}:~é uma variável numérica que representa o peso do
    filhote.
  \item
    \texttt{feed}:~um fator (variável qualitativa/categórica) que
    representa o tipo de alimentação.
  \end{itemize}
\item
  Calcule um intervalo de confiança para o peso médio de cada
  alimentação separadamente; o intervalo de confiança é de
  inferior(\texttt{lower}) para superior (\texttt{upper}) dado por
  média \(\pm\) escore t * erro padrão
  (\texttt{mean}\(\pm\)\texttt{tscore\ *\ ep}):
\end{itemize}

\begin{Shaded}
\begin{Highlighting}[]
\NormalTok{cwei <-}\StringTok{ }\KeywordTok{favstats}\NormalTok{( weight }\OperatorTok{~}\StringTok{ }\NormalTok{feed, }\DataTypeTok{data =}\NormalTok{ chickwts)}
\NormalTok{ep <-}\StringTok{ }\NormalTok{cwei}\OperatorTok{$}\NormalTok{sd }\OperatorTok{/}\StringTok{ }\KeywordTok{sqrt}\NormalTok{(cwei}\OperatorTok{$}\NormalTok{n) }\CommentTok{# Erros padrão}
\NormalTok{tscore <-}\StringTok{ }\KeywordTok{qdist}\NormalTok{(}\StringTok{"t"}\NormalTok{, }\DataTypeTok{p =}\NormalTok{ .}\DecValTok{975}\NormalTok{, }\DataTypeTok{df =}\NormalTok{ cwei}\OperatorTok{$}\NormalTok{n }\OperatorTok{-}\StringTok{ }\DecValTok{1}\NormalTok{, }\DataTypeTok{plot =} \OtherTok{FALSE}\NormalTok{) }\CommentTok{# Escores t para uma probabilidade de cada à direita de 2.5%}
\NormalTok{cwei}\OperatorTok{$}\NormalTok{lower <-}\StringTok{ }\NormalTok{cwei}\OperatorTok{$}\NormalTok{mean }\OperatorTok{-}\StringTok{ }\NormalTok{tscore }\OperatorTok{*}\StringTok{ }\NormalTok{ep}
\NormalTok{cwei}\OperatorTok{$}\NormalTok{upper <-}\StringTok{ }\NormalTok{cwei}\OperatorTok{$}\NormalTok{mean }\OperatorTok{+}\StringTok{ }\NormalTok{tscore }\OperatorTok{*}\StringTok{ }\NormalTok{ep}
\NormalTok{cwei[, }\KeywordTok{c}\NormalTok{(}\StringTok{"feed"}\NormalTok{, }\StringTok{"mean"}\NormalTok{, }\StringTok{"lower"}\NormalTok{, }\StringTok{"upper"}\NormalTok{)]}
\end{Highlighting}
\end{Shaded}

\begin{verbatim}
##        feed     mean    lower    upper
## 1    casein 323.5833 282.6440 364.5226
## 2 horsebean 160.2000 132.5687 187.8313
## 3   linseed 218.7500 185.5610 251.9390
## 4  meatmeal 276.9091 233.3083 320.5099
## 5   soybean 246.4286 215.1754 277.6818
## 6 sunflower 328.9167 297.8875 359.9458
\end{verbatim}

\begin{itemize}
\tightlist
\item
  Nós podemos traçar os intervalos de confinaça como segmentos de
  linhas horizontais usando a função \texttt{gf\_errorbarh}:
\end{itemize}

\begin{Shaded}
\begin{Highlighting}[]
\KeywordTok{gf_errorbarh}\NormalTok{(feed }\OperatorTok{~}\StringTok{ }\NormalTok{mean }\OperatorTok{+}\StringTok{ }\NormalTok{lower }\OperatorTok{+}\StringTok{ }\NormalTok{upper, }\DataTypeTok{data =}\NormalTok{ cwei) }\OperatorTok\StringTok{ }
\StringTok{  }\KeywordTok{gf_point}\NormalTok{(feed }\OperatorTok{~}\StringTok{ }\NormalTok{mean)}
\end{Highlighting}
\end{Shaded}

\includegraphics{lecture-estimation_files/figure-latex/unnamed-chunk-10-1.pdf}

\section{Determinando o tamanho da
amostra}\label{determinando-o-tamanho-da-amostra}

\subsection{Tamanho da amostra para a
proporção}\label{tamanho-da-amostra-para-a-proporaao}

\begin{itemize}
\tightlist
\item
  O intervalo de confiança é da forma estimativa por ponto \(\pm\)
  margem de erro estimada.
\item
  Quando nós estimamos uma proporção, a margem de erro é \[
  e=z_{crit}\sqrt{\frac{p(1-p)}{n}},
  \] onde o escore \(z\) crítico, \(z_{crit}\), é determinado pelo
  nível de confiança especificado.
\item
  Imagine que nós queremos planejar um experimento, onde
  \textbf{desejamos obter uma certa margem de erro \(e\)} (e portanto
  uma largura específica do intervalo de confiança associado).
\item
  Se nós resolvermos a equação acima, é possível notar:

  \begin{itemize}
  \tightlist
  \item
    Se \(n=p(1-p)(\frac{z_{crit}}{e})^2\), nós obtemos uma estimativa
    de \(p\) com margem de erro \(e\).
  \end{itemize}
\item
  Se não tivermos um bom palpite para o valor de \(p\), nós podemos
  usar o valor do ``pior'' caso, isto é, \(p=50\%\) (\(1/4\) é o maior
  valor possível que pode assumir a função \(p(1-p)\)). O tamanho
  amostral correspondente \(n=(\frac{z_{crit}}{2e})^2\) garante que nós
  iremos obter uma estimativa com uma margem de erro que é no
  \emph{máximo} \(e\).
\end{itemize}

\begin{center}\rule{0.5\linewidth}{\linethickness}\end{center}

\subsubsection{Exemplo}\label{exemplo}

\begin{itemize}
\tightlist
\item
  Vamos escolher \(z_{crit}=1.96\), i.e~o nível de confiança é de
  95\%.
\item
  Qual o número de eleitores que devemos entrevistar para obter uma
  margem de erro igual a \(1\%\)?
\item
  O pior caso é \(p=0.5\), o que nos leva a: \[
    n=p(1-p)\left(\frac{z_{crit}}{e}\right)^2=\frac{1}{4}\left(\frac{1.96}{0.01}\right)^2 = 9604.
    \]
\item
  Se nós estivermos interessados na proporção de votos para o
  candidato de um partido cujo um bom palpite é de no máximo
  \(p=0.23\), teríamos \[
    n=p(1-p)\left(\frac{z_{crit}}{e}\right)^2=0.23(1-0.23)\left(\frac{1.96}{0.01}\right)^2 = 6804.
    \]
\item
  Se nós estivermos interessados na proporção de votos para o
  candidato de um partido cujo um bom palpite é de no máximo
  \(p=0.05\), teríamos \[ 
    n=p(1-p)\left(\frac{z_{crit}}{e}\right)^2 = 0.05(1-0.05)\left(\frac{1.96}{0.01}\right)^2 = 1825.
    \]
\end{itemize}

\subsection{Tamanho da amostra para a
média}\label{tamanho-da-amostra-para-a-madia}

\begin{itemize}
\tightlist
\item
  O intervalo de confiança é da forma estimativa por ponto \(\pm\)
  margem de erro estimada.
\item
  Quando estimamos uma média com desvio-padrão conhecido a margem de
  erro é \[
    e=z_{crit}\frac{\sigma}{\sqrt{n}},
    \] onde um escore \(z\) crítico, \(z_{crit}\), é determinado pelo
  nível especificado de confiança.
\item
  Imagine que nós queremos planejar um experimento, onde
  \textbf{desejamos obter uma certa margem de erro \(e\)}.
\item
  Se nós resolvermos a equação acima, é possível notar:

  \begin{itemize}
  \tightlist
  \item
    Se \(n=(\frac{z_{crit}\sigma}{e})^2\), nós obtemos uma estimativa
    com margem de erro \(e\).
  \end{itemize}
\item
  Problema: Nós usualmente não sabemos \(\sigma\). Possíveis
  soluções:

  \begin{itemize}
  \tightlist
  \item
    Com base em estudos similares realizados anteriormente, nós fazemos
    um palpite ``educado'' sobre o valor de \(\sigma\).
  \item
    Baseado em um estudo piloto estimamos o valor de \(\sigma\).
  \end{itemize}
\end{itemize}

\section{Exercícios e Leituras
Recomendadadas}\label{exercacios-e-leituras-recomendadadas}

\(1\). Qual deve ser o valor do desvio padrão para que uma
distribuição normal com média 9 cubra o intervalo 0-18 com
probabilidade de 99.73\%? Experimente com diferentes valores de \(n\).

\(2\). Leia a seção 3.4 de Costa Neto (doravante CN) e verifique o que
foi dado em aula sobre o assunto.

\(3\). Leia as seções 4.4 e 4.5 de CN e refaça os exemplos (tem
solução).

\(4\). Faça os exercícios 1, 3, 5, 7, 14, 16, 17, 18 e 21 da seção
4.6 (Exercícios propostos) de CN.

\section{Apêndice}\label{apandice}

\begin{itemize}
\item
  Instale o pacote TeachingDemos: install.packages(``TeachingDemos'').
  Após a instalação carregue o pacote para poder usá-lo:
  library(TeachingDemos).
\item
  Use a função ci.examp() para visualizar 50 intervalos de confiança.
  Você pode visualizar os argumentos usados (by default) na função
  pedindo ajuda: ?ci.examp
\item
  Tente alterar alguns dos argumentos da função e utilize method =
  ``t''. Experimente, por exemplo, gerar 40 amostras aleatórias (reps =
  40) e ``depois peça os gráficos correspondentes com''plote" os
  correspondentes 40 intervalos de confiança usando um nível de
  confiança de 90\% (conf.level = 0.90).
\item
  Convença-se de que esperamos que 4 destes intervalos de confiança
  não contenham a verdadeira média populacional. Como isso se encaixa
  com o que você está vendo?
\item
  Visualize melhor usando o seguinte aplicativo:
  \url{http://shiny.calvin.edu/rpruim/CIs/}
\item
  Experimente também a função clt.examp(). Visualize os argumentos
  pedindo ajuda: ?clt.examp
\item
  Use a função windows() para abrir um dispositivo gráfico.

  \begin{itemize}
  \tightlist
  \item
    clt.examp()
  \item
    clt.examp(5)
  \item
    clt.examp(30)
  \item
    clt.examp(50)
  \end{itemize}
\end{itemize}


\end{document}
