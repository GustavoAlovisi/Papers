\documentclass[review]{elsarticle}

\usepackage{lineno,hyperref}
\usepackage{color}
\modulolinenumbers[5]

\journal{Emerging Markets Review}

%%%%%%%%%%%%%%%%%%%%%%%
%% Elsevier bibliography styles
%%%%%%%%%%%%%%%%%%%%%%%
%% To change the style, put a % in front of the second line of the current style and
%% remove the % from the second line of the style you would like to use.
%%%%%%%%%%%%%%%%%%%%%%%

%% Numbered
%\bibliographystyle{model1-num-names}

%% Numbered without titles
%\bibliographystyle{model1a-num-names}

%% Harvard
%\bibliographystyle{model2-names.bst}\biboptions{authoryear}

%% Vancouver numbered
%\usepackage{numcompress}\bibliographystyle{model3-num-names}

%% Vancouver name/year
%\usepackage{numcompress}\bibliographystyle{model4-names}\biboptions{authoryear}

%% APA style
%\bibliographystyle{model5-names}\biboptions{authoryear}

%% AMA style
%\usepackage{numcompress}\bibliographystyle{model6-num-names}

%% `Elsevier LaTeX' style
\bibliographystyle{elsarticle-num}
%%%%%%%%%%%%%%%%%%%%%%%

\begin{document}

\begin{frontmatter}

\title{High Frequency Pairs Trading at the Brazilian Financial Market: A Mixed Copula-HEAVY approach}
%\tnotetext[mytitlenote]{Fully documented templates are available in the elsarticle package on \href{http://www.ctan.org/tex-archive/macros/latex/contrib/elsarticle}{CTAN}.}

%% Group authors per affiliation:
%\author{Elsevier\fnref{myfootnote}}
%\address{Radarweg 29, Amsterdam}
%\fntext[myfootnote]{Since 1880.}

%% or include affiliations in footnotes:

\author[address1]{Fernando Augusto Boeira Sabino da Silva\corref{mycorrespondingauthor}}
\cortext[mycorrespondingauthor]{Corresponding author}
\ead{fsabino@ufrgs.br}

\author[address2]{Tainan de Bacco Freitas Boff}
\ead{tainan.freitas@ufrgs.br}

\author[address1]{Flavio Augusto Ziegelmann}
\ead{flavioz@ufrgs.br}

\address[address1]{Department of Statistics, Institute of Mathematics and Statistics, Federal University of Rio Grande do Sul, 9500 Bento Gonçalves Av., 43-111, Porto Alegre, RS, 91509-900, Brazil}
\address[address2]{Graduate Program in Economics, Federal University of Rio Grande do Sul, Porto Alegre, Brazil}

\begin{abstract}
\textcolor{red}{Inserir abstract.}
\end{abstract}

\begin{keyword}
Pairs Trading \sep Copula \sep Distance \sep Statistical Arbitrage \sep High-Frequency \sep Realized Variance % Stationary Bootstrap. Principal Component Analysis. Trimmed Scores Regression 
% A revista pede para inserir no máximo 6 palavras-chave.
% C51, C58 # Acho que não devem ser usadas neste trabalho, já que estudos que utilizam métodos econométricos bem estabelecidos não devem ser classificados na categoria C.
\JEL  G11 \sep G12 \sep G14 
\end{keyword}

\end{frontmatter}

\linenumbers

\section{Introduction}

\section{Methodology}

\subsection{Pairs trading} % 3

\subsection{Distance Framework} % 4

\subsection{Copula Framework} % 5

\subsection{HEAVY model} % 1
% Realized variance

\subsection{Performance assessment} 

\section{Data and empirical results}

\subsection{Data} % 2

\subsection{Results and discussion} 

\section{Conclusions}

\section*{References}

\bibliography{mypapers}

\end{document}