	\section{Concluding Remarks}
	
Pairs trading fall under the class of statistical arbitrage strategies. It involves a portfolio consisting of long one stock and short the other betting on the empirical fact that the spread among stocks which have strong co-movements tend to return to their historical level.

The main goal of this paper is to verify if a strategy composed of a mixture of copulas is able to generate higher and more robust returns than the distance methodology. We are also interested in understanding better the factors that affect their profitability. 

Using a long-term comprehensive data set spanning 25 years, our empirical analysis suggests that the mixed copula strategy has a superior performance than the distance approach 
when the number of trades is comparable, which occurs for the case of Top 5 pairs.

 
The main findings when the number of trading signals is equiparable are summarized below.
	
	\begin{enumerate}
				
		
	\item The mixed copula strategy is able to generate a higher mean excess return and a Sharpe ratio over twice as much as what we get from investing in the traditional distance method after trading costs. 
	
		
		\vspace{0.3cm}
		
		\item 	The mixed copula approach delivers economically larger alphas than the distance method for both weighting schemes (10 and 58 bps per month on committed and fully invested capital, respectively) after transaction costs, suggesting the importance of the proposed method. It should also be noted that the alphas provided by mixed copula and distance strategies are significant at 1\% and 10\%, respectively, after accounting for several asset pricing factors such as momentum, liquidity, profitability and investment. Thus, the results show that the profits are not fully explained by the other factors.
		
		\vspace{0.3cm}
		
		\item As it can be observed, the right-hand-side tail (of positive outcomes) of the density of the five-year Sharpe ratio is longer for the mixed copula strategy, implying that the copula-based strategy has a better risk-adjusted performance than the distance approach.
		
		\vspace{0.3cm}
		
		\item The share of days with negative excess returns is smaller for the mixed copula approach (41.79\%) than for the distance strategy (46.98\%) and the market performance (47.45\%). 
		
	    \vspace{0.3cm}
		
		
		\item	Neither strategy consistently shows superiority over all subperiods, at least on committed capital. Overall, the mixed copula strategy shows a superior out-of-sample performance relative to the distance approach in the second and third subperiods (1996-2000 and 2001-2005) and after the subprime mortgage crisis (2011-2015), while the distance method delivers a significant better performance in the first (1991-1995) and fourth subperiods (2006-2010) on committed capital. 
		
	\end{enumerate}

\vspace{0.3cm}

We found that the average number of pairs traded per six-month period is only comparable among the strategies for Top 5 pairs in this study. This suggests that a constant two standard deviation threshold \citep{ggr06} is less conservative than the opening trigger point suggested by \citet*{rf15} using the cumulative mispricing indexes $M_{1,t}$ and $M_{2,t}$. Further studies in the application of copulas in pairs trading should investigate the optimal points of entry and exit to make the comparisons more meaningful.


	\newpage