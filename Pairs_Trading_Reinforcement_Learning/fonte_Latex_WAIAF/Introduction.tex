	\section{Introduction}
	
	Pairs trading is one type of statistical arbitrage strategy that involves the simultaneous long/short of a pair of securities whose prices tend to move together. When the spread between them widens, we short the more expensive stock (winner) and buy an equivalent dollar amount of the cheap stock (loser). Thus, pairs trading is an example of a contrarian investment strategy which bets on price reversals. At convergence, the lower stock is sold and short position is covered. The difference in the cash flows on the convergence is the arbitrageur profit. The return on the long position less the return on the short position is the excess return of this pair. This convergence of stock prices related to the mean reversion has been documented by \citet*{de1985,de1987} and \citet*{jegade1993}. The strategy was developed by Nunzio Tartaglia and his quantitative group of former academics at Morgan Stanley in the 1980s. They report that the black box strategy made over $50$ million profit for the firm in 1987. Pairs trading has since become an increasingly popular market netrual investment strategy used by hedge funds, proprietary trading desks, and individual traders. 
	
	\citet*{jt95} argue that most of behind potential contrarian profits are due to excessive overreaction to company-specific factors. \citet*{da2013closer} label this sentiment-based explanation. \citet*{shiller1984,black1986,stiglitz1989,summers1989} , and \citet*{subra2005}, among others, have suggested that short-term reversal profits are evidence that market prices may reflect investor overreaction to information, or fads, or simply cognitive errors. Another possible explanation is labeled liquidity-based explanation by \citet*{da2013closer}. In the model of \citet*{campbell1993} uninformed traders lead to a temporary price concession that, when absorbed by liquidity providers, results in a reversal in price that serves as compensation for those who provide liquidity. Consistent with such mechanism, \citet*{avramov2006} ) find that the reversal profits mainly derive prom positions in small, high turnover, and illiquid stocks. \citet*{pastor2003} suggest directly measuring the degree of illiquidity by the occurrence of an initial price change and subsequent reversal. 
%		rationale)to)support)the)results Engelberg,)Gao)and)Jagannathan)(2008) also)discuss)the)rationale)behind)potential)profits)in)
%	pairs)trading)and)study)how)idiosyncratic)news,)common)news)and)liquidity)are)affecting)the)
%	pairs’)performance.) )there)are)few)explanations)or)theoretical)arguments)that)can)rationalize)the)results.)
%	theoretical)reasons)for)why)pairs)trading)might be)profitable.

The explanation for profitability of contrarian strategies lies in the propensity of noise traders to make decisional errors, and informed investors to have a preference for prior winner stocks
	
	he rationale behind the mean-reverting process is that there exists a long-term equilibrium (mean) for the spread. The investor may bet on the reversion of the current spread to its historical mean by selling and buying an appropriate amount of the pair of the stocks.
	
	The rationale of the pairs trading is to
	make a profit and avoid market risk. 
	
	There are several reasons for the popularity of pairs-trading. First, the procedure is simple to
	understand and execute. Second, valuation models, which are subjected to wide error margins, are not
	required since pairs-trading is based on relative valuation and the position is often near market-neutral.
	Third, it is sufficiently flexible to accommodate various investment styles. Lastly, it normally does not
	evoke frequent intraday re-balancing, such that actual trading can be automated and is feasibly
	profitable
	
	The pairs trading strategy is a popular method for trading financial assets. One of the reasons for such popularity is that the result of this type of operation depends solely on the relationship between the price of two assets, and not on the overall market condition. The possibility of spotting inefficiencies in assets pricing is what allows the investor to make consistent profits using a systematic method for trading financial contracts.
	
	However, it became popular through the study carried out by \citet*{ggr06}, named distance method.
	
	Currently, there are three main approaches for pairs trading: distance, cointegration and copula. The traditional distance method has been widely researched and tested throughout the pairs trading literature. However, this approach only captures dependencies well in the case of elliptically distributed random variables. This assumption is generally not met in practice, motivating the utilization of copula-based models to address the univariate and multivariate stylized facts for multivariate financial return stocks. Nevertheless, the use of copulas in this context is still recent and needs more comprehensive and profound studies.
	
	The performance of the distance method has been measured thoroughly using different data sets and financial markets \citep{ggr06,p09,df10,df12,bv12,cm13,rf15}. Different approaches are provided in many articles and books (see, among others, \citet{v04,elliott2005,do2006,avellaneda2010,bogomolov2013,stubinger2016,liu2017,stubinger2018}). In an efficient market, strategies based on mean-reversion concepts should not generate consistent profits. However, \citet*{ggr06} find that pairs trading generates consistent statistical arbitrage profits in the U.S. equity market during 1962-2002 using CSRP data, although the profitability declines over the period. They obtain a mean excess return above 11\% a year during the reported period. The authors attribute the abnormal returns to a non-identified systematic risk factor. They support their view showing that there is a high degree of correlation between the excess returns of no overlapping top pairs even after accounting for risk factors from an augmented version of \citet*{ff93}'s three factors. \citet*{df10} extend their work expanding the data sample and also find a declining trend - 33 basis points (bps) mean excess return per month for 2003-09 versus 124 basis points mean excess return per month for 1962-88. \citet*{df12} show that the distance method is unprofitable after 2002 when trading costs are considered. \citet*{bv12} test the profitability of pairs trading under different weighting structures and trade initiation conditions using data from the Finnish stock market. They also find that their proposed strategy is profitable even after initiating the positions one day after the signal. \citet*{rf15} evaluates distance, cointegration and copula methods using a long-term comprehensive data set spanning over five decades. They find that the copula method has a weaker performance than the distance and cointegration methods in terms of excess returns and various risk-adjusted metrics.
	
The distance strategy \citep{ggr06} uses the distance between normalized security prices to capture the degree of mispricing between stocks. According to \citet*{xie14} the distance method has a multivariate normal nature since it assumes a symmetric distribution of the spread between the normalized prices of the stocks within a pair and it uses a single distance measure, which can be seen as an alternative measurement of the linear association, to describe the relationship between two stocks. We know that if the series have joint normal distribution, then the linear correlation fully describes the dependence between securities. However, it is well known that the dependence between two securities are rarely jointly normal and thus the traditional hypothesis of (multivariate) gaussianity is completely inadequate\footnote{A main feature of joint distributions characterized by tail dependence is the presence of heavy and possibly asymmetric tails.} \citep{campbell97,cont01,ane03,mcneil15}.  Therefore, a single distance measure may fail to catch the dynamics of the spread between a pair of securities, and thus initiate and close the trades at non-optimal positions, restricting alpha-generation.
	
Due to the complex dependence patterns of financial markets, a high-dimensional multivariate approach to tail dependence analysis is surely more insightful than assuming multivariate normal returns. Due to its flexibility, copulas are able to model better the empirically verified regularities normally attributed to multivariate financial returns: (1) asymmetric conditional variance with higher volatility for negative returns than for positive returns \citep{h98}; (2) conditional skewness \citep{ait01,chen01,patton01}; (3) Leptokurdicity \citep{t01,andreou01}; and (4) nonlinear temporal dependence \citep{cont01,campbell97}. Thus, to address these issues, \citet*{lw2013} propose a pairs trading strategy based on two-dimensional copulas. However, they evaluate its performance using only three pre-selected pairs over a period of less than three years. \citet*{xie14} employ a similar methodology over a ten-year period with 89 stocks. Both studies find that the performance of copula strategy is superior to the distance strategy. \citet*{xw13} set out the distance and cointegration approaches as special cases of copulas under certain regularity conditions. The authors also recommend further research on how to incorporate copulas in pairs selection. It is suggested there is a possibility of larger profits in terms of returns since copulas deal better with non-linear dependence structures. The approach may sound plausible but it may not lead to a viable standalone trading quantitative strategy due to overfitting issues, hence not justifying the marginal performance improvement given by a more complex model. As cited above \citet*{rf15} use a more comprehensive data set consisting of all stocks in the US market from 1962 to 2014. Meanwhile, they find an opposite result. Particularly, the distance, cointegration and Copula-GARCH strategies show a mean monthly excess return of 36, 33, and 5 bps after transaction costs and 88, 83, and 43 bps before transaction costs.

In this paper, we will conduct an empirical investigation to offer some evidence of the behavior of the distance and mixed copula strategies. We propose, differently from \citet*{rf15} and \citet*{xie14}, a mixed copula-based model to capture linear and nonlinear associations and at the same time cover a wider range of possible dependence structures. We aim to assess whether building a more sophisticated strategy can take advantage of any market frictions or anomalies uncovering relationships and pattern, bearing a potential of higher returns compared to the traditional approach. We find that the mixture copula strategy is able to generate a higher mean excess return than the distance method when the number of trading signals is equiparable. We also want to investigate the sensitivity of the copula method to different opening thresholds and how trading costs affect the profitability of these strategies.

 Our strategy consists in fitting, initially, the daily returns of the formation period using an ARMA(p,q)-GARCH(1,1) to model the marginals. For each pair, we test the following elliptical and Archimedean copula functions: Gaussian, t, Clayton, Frank, Gumbel, one Archimedean mixture copula consisting of the optimal linear combination of Clayton, Frank and Gumbel copulas and one mixture copula consisting of the optimal linear combination of Clayton, t and Gumbel copulas. Following \citet*{ggr06} we calculate returns using two weighting schemes: the return on committed capital and on fully invested capital. The former commits\footnote{We assume zero return for non-open pairs, although in practice one could earn returns on idle capital.} equal amounts of capital to each one of the pairs even if the pair has not been traded\footnote{The commited capital is considered more realistic as it takes into account the opportunity cost of the capital that has been allocated for trading.}, whereas the latter divides all capital among the pairs that are open.
	
We compare the performance out-of-sample of the strategies using a variety of criteria, all of which are computed using a rolling period procedure similar to that used by \citet*{ggr06} with the exception that the time horizon of formation and trading periods are rolled forward by six months as in \citet*{bv12}. The main criteria we focus on are: (1) mean and cumulative excess return, (2) risk-adjusted metrics as Sharpe and Sortino ratios, (3) percentage of negative trades, (4) t-values for various risk factors, and (5) maximum drawdown between two consecutive days and between two days within a maximum period of six months.

	In order to evaluate if pairs trading profitability is associated to exposure to different systematic risk factors\footnote{The single-factor capital asset pricing model (CAPM) of \citet*{s64} and \citet*{l65}, as well as its consumption based version \citep{b79}, among other extensions, has been empirically tested and rejected by numerous studies, which show that the cross-sectional variation in expected equity returns cannot be explained by the market beta alone, providing evidence that investors demand compensation for not being able to diversify firm-specific characteristics.}, we regress daily excess returns on seven factors: daily \citet*{ff15}'s five research factors \footnote{\citet*{ff15} found evidences that the three factor model was not sufficient to explain a lot of the variation in average returns related to profitability and investment.} plus momentum and long-term reversal. We find that the intercept is statistically greater than zero for all regressions at 1\% level when considering the mixed copula strategy, showing that our results are robust to the augmented \citet*{ff15}'s risk adjustment factors. In addition, the share of observations with negative excess returns is smaller for the mixed copula than for the distance strategy.
	
	To test for differences in returns and Sharpe ratios we use the stationary bootstrap of \citet*{pr94} with the automatic block-length selection of \citet*{pw04} and 10,000 bootstrap resamples. To compute the bootstrap p-values we employ the methodology proposed by \citet*{lw08}. We aim to compare the results on a statistical basis to mitigate potential data snooping issues.
	
	The remainder of the paper is organized as follows. A general review of the distance and copula models as well as the trading strategies we perform are discussed in Section 2. Section 3 summarizes the data and empirical results of the analysis. Finally, Section 4 provides a brief conclusion. Additional results are reported in the Appendix.
	
	\vspace{0.6cm}
