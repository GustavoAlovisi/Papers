\documentclass[a4paper]{article}
%\documentclass[a4paper,12pt]{article}
\usepackage[utf8]{inputenc}
\usepackage{indentfirst}
\usepackage{times}
\usepackage[T1]{fontenc}
\usepackage[affil-it]{authblk}
\usepackage{authblk}
\usepackage{amssymb,amsmath,amsthm,ragged2e}
\usepackage{setspace,lipsum}
\usepackage[pdftex]{color,graphicx}
\usepackage{adjustbox}
\usepackage{lmodern,caption}
\usepackage{booktabs,array}
\usepackage{epstopdf}
\DeclareGraphicsRule{.tif}{png}{.png}{`convert #1 `dirname #1`/`basename #1 .tif`.png}
\usepackage{array,longtable}
\usepackage[top=3cm, bottom=2cm, left=3cm, right=2cm]{geometry}
\newcommand{\otoprule}{\midrule[\heavyrulewidth]}
\newcolumntype{Z}{>{\centering\arraybackslash}X}
\usepackage{latexsym,hhline}
\usepackage[round]{natbib} % use author-year citation style
\usepackage[colorlinks=true,allcolors=blue]{hyperref} % simplify color scheme ...
%\usepackage{cite}
\usepackage{setspace}
\onehalfspacing
\DeclareMathOperator{\diag}{diag}
\DeclareMathOperator{\Ima}{Im}
\DeclareCaptionLabelSeparator{horse}{:\quad} % change according to your needs
\captionsetup{
	labelsep = horse,
	figureposition = bottom % proper spacing between figure and caption
}
\usepackage{tikz,color,listings,float,booktabs,enumerate,multirow,multicol,subcaption}
\usepackage{parskip}
\setlength{\parindent}{15pt}
\usepackage{bigstrut,hyperref,tabularx,ctable,array,longtable,tikz}
\hypersetup{
	pdftitle={TITLE},
	pdfauthor={AUTHOR},
	pdfsubject={SUBJECT},
	pdfkeywords={KEYWORD} {KEYWORD} {KEYWORD},
	colorlinks=true,
	linkcolor=blue,
	citecolor=blue,
	filecolor=magenta,
	urlcolor=blue
}
\newtheorem{theorem}{Theorem}
\newtheorem{proposition}{Proposition}
\newtheorem{lemma}{Lemma}
\newtheorem{definition}{Definition}
\newtheorem{corollary}{Corollary}[theorem]
\usepackage{rotating}

% Real Number
\newcommand{\R}{\mathbb R}
% natural numbers
\newcommand{\Nat}{\mathbb N}
% complex numbers
\newcommand{\C}{\mathbb C}
\newcommand{\Pstar}{\mathbb{P}^{\ast}}
\newcommand{\E}{\mathbb{E}}
\newcommand{\Var}{\mathrm{Var}}
\newcommand{\Cov}{\mathrm{Cov}}
\newcommand{\Expect}{{\rm I\kern-.3em E}}
\usepackage[listings]{tcolorbox}
\tcbuselibrary{listings,theorems}
\usepackage{fancyheadings,xspace,amsmath,tikz-qtree}
\usetikzlibrary{matrix}

\usepackage[labelfont=bf]{caption} %% for the figure and Tables Caption%%
%\usepackage[textfont=bf]{caption}
\usepackage{threeparttable}
\usepackage{standalone}

\begin{document}
	
	
	\makeatletter
	\def\@maketitle{%
		\newpage
		\null
		\vskip 2em%
		\begin{center}%
			\let \footnote \thanks
			{\Large\bfseries \@title \par}%
			\vskip 1.5em%
			{\normalsize
				\lineskip .5em%
				\begin{tabular}[t]{c}%
					\@author
				\end{tabular}\par}%
			\vskip 1em%
			{\normalsize \@date}%
		\end{center}%
		\par
		\vskip 1.5em}
	\makeatother
	
	\title{Performance of Pairs Trading on the S\&P 500 using Distance and Mixed Copula models}
	\author[]{ Fernando A. Boeira Sabino da Silva\thanks{\texttt{Department of Statistics, Federal University of Rio Grande do Sul, Porto Alegre, RS 91509-900, Brazil, e-mail: fsabino@ufrgs.br}; Corresponding author.}}
	\author[]{Flavio A. Ziegelmann\thanks{\texttt{Department of Statistics, Federal University of Rio Grande do Sul, Porto Alegre, RS 91509-900, Brazil, e-mail: flavioz@ufrgs.br}}}
	\author[]{João F. Caldeira \thanks{\texttt{Department of Economics, Federal University of Rio Grande do Sul, Porto Alegre, RS 90040-000, Brazil, e-mail: joao.caldeira@ufrgs.br}}\thanks{We thank Cristina Tessari for her suggestions and for helping us obtaining the data we use.}}
	\affil[]{}
	%\affil[2]{}
	\date{\today}
	\maketitle
	
	
	\begin{abstract}
		We carry out a study to evaluate and compare the relative performance of the distance and mixed copula strategies. Using data from the S\&P500 shares from 1990 to 2015, we find that the mixed copula strategy is able to generate a higher mean excess return than the traditional distance method under different weighting structures when the number of tradeable signals is equiparable. Particularly, the mixed copula and distance methods show a mean annualized value-weighted excess returns after costs on committed and fully invested capital as high as 3.98\% and 3.14\% and 12.73\% and 6.12\%, respectively, with annual Sharpe ratios up to 0.88. The mixed copula strategy shows positive and significant alphas during the sample period after accounting for various risk-factors.
		
		\smallskip
		
		\noindent \textbf{Keywords:} Distance; Mixed Copula; Two-Dimensional Pairs Trading; S\&P 500; Statistical Arbitrage.
		
		\noindent \textbf{JEL Codes:} C51, C58, G11.
	\end{abstract}
	
	% Add various lists on new pages.
	%\pagebreak
	%\tableofcontents
	
	%\pagebreak
	%\listoffigures
	
	%\pagebreak
	%\listoftables
	
	% Start the paper on a new page.
	
	\section{Introduction}

	Pairs trading is a statistical arbitrage strategy that involves the simultaneous long/short of two relatively mispriced stocks which have strong historical co-movements. The performance of the strategies has been recently discussed in several studies with much interest in empirical finance, since the strategy has potential to generate sustained alpha through relatively low-risk positions. In addition, the strategy is claimed to be market neutral, which means that the investors are not exposed to market risk. The strategy was pioneered by Gerry Bamberger and later led by Nunzio Tartaglia's quantitative group at Morgan Stanley in the 1980s. However, it became popular through the study carried out by \citet*{ggr06}, named distance method.
	
	Currently, there are three main strategies for pairs trading: distance, cointegration and copula. In this paper, we will conduct an empirical investigation to offer some evidence of the behavior of the distance and mixed copula strategies under different investment scenarios. 
	
	The performance of the distance method has been measured thoroughly using different data sets and financial markets \citep{ggr06,p09,df10,df12,bv12,cm13,rf15}. In an efficient market, strategies based on mean-reversion concepts should not generate consistent profits. However, \citet*{ggr06} find that pairs trading generates consistent statistical arbitrage profits in the U.S. equity market during 1962-2002 using CSRP data, although the profitability declines over the period. They obtain a mean excess return above 11\% a year during the reported period. The authors attribute the abnormal returns to a non-identified systematic risk factor. They support their view showing that there is a high degree of correlation between the excess returns of no overlapping top pairs even after accounting for risk factors from an augmented version of \citet*{ff93}'s three factors. \citet*{df10} extend their work expanding the data sample and also find a declining trend - 33 basis points (bps) mean excess return per month for 2003-09 versus 124 basis points mean excess return per month for 1962-88. \citet*{df12} show that the distance method is unprofitable after 2002 when trading costs are considered. \citet*{bv12} test the profitability of pairs trading under different weighting structures and trade initiation conditions using data from the Finnish stock market. They also find that their proposed strategy is profitable even after initiating the positions one day after the signal. \citet*{rf15} evaluates distance, cointegration and copula methods using a long-term comprehensive data-set spanning over five decades. They find that the copula method has a weaker performance than the distance and cointegration methods in terms of excess returns and various risk-adjusted metrics.
	
The distance strategy \citep{ggr06} uses the distance between normalized security prices to capture the degree of mispricing between stocks. According to \citet*{xie14} the distance method has a multivariate normal nature since it assumes a symmetric distribution of the spread between the normalized prices of the stocks within a pair and it uses a single distance measure, which can be seen as an alternative measurement of the linear association, to describe the relationship between two stocks. We know that if the series have joint normal distribution, then the linear correlation fully describes the dependence between securities. However, a main feature of joint distributions characterized by tail dependence is the presence of heavy and possibly asymmetric tails. it is well known that the dependence between two securities are rarely jointly normal and thus the traditional hypothesis of (multivariate) gaussianity is completely inadequate  \citep{campbell97,cont01,ane03,mcneil15}.  Therefore, a single distance measure may fail to catch the dynamics of the spread between a pair of securities, and thus initiate and close the trades at non-optimal positions.
	
Due to the complex dependence patterns of financial markets, a high-dimensional multivariate approach to tail dependence analysis is surely more insightful than assuming multivariate normal returns. Since its flexibility, copulas are able to model better the empirically verified regularities normally attributed to multivariate financial returns: (1) asymmetric conditional variance with higher volatility for large negative returns and smaller volatility for positive returns \citep{h98}; (2) conditional skewness \citep{ait01,chen01,patton01}; (3) Leptokurdicity \citep{t01,andreou01}; and (4) nonlinear temporal dependence \citep{cont01,campbell97}. Thus, to address these issues, \citet*{lw2013} propose a pairs trading strategy based based on two-dimensional copulas to overcome the limitations of the distance method. However, they evaluate its performance using only three pre-selected pairs over a period of less than three years. \citet*{xie14} employ a similar methodology over a ten-year period with 89 stocks. Both studies find that the performance of copula strategy is superior to the distance strategy. \citet*{xw13} set out the distance and cointegration approaches as special cases of copulas under certain regularity conditions. The authors also recommend further research on how to incorporate copulas in pairs selection. It is suggested there is a possibility of larger profits in terms of returns since copulas deals better with non-linear dependency structures. The approach may sound plausible but it may not lead to a viable standalone trading quantitative strategy due to issues as overfitting, hence do not justifying the marginal performance improvement given by a more complex model. As cited above \citet*{rf15} use a more comprehensive data set consisting of all the shares in the US market from 1962 to 2014. Meanwhile, they find an opposite result. Particularly, the distance, cointegration and Copula-GARCH strategies show a mean monthly excess return of 36, 33, and 5 bps after transaction costs and 88, 83, and 43 bps before transaction costs.

The main novel contribution of this paper is to employ a mixed copula model in order to capture linear/nonlinear associations and at the same time cover a wider range of possible dependencies structures. We aim to assess if build a more sophisticated model can take advantage of any market frictions/anomalies uncovering relationships and patterns. We find that the mixture copulas are selected for more than 90\% of the pairs in the scenarios investigated and that the strategy is able 
to generate a higher mean excess return than the distance method when the 
number of trading signals is equiparable. We also want to investigate the sensitivity of copula method to different opening thresholds and how trading costs and different measures for pairs selection affect the profitability of these strategies.

 Our strategy consists in fitting, initially, the daily returns of the formation period using an ARMA(1,1)-GARCH(1,1) to model the marginals. For each pair, we test the following elliptical and Archimedean copula functions: Gaussian, t, Clayton, Frank, Gumbel, one Archimedean mixture copula consisting of the optimal linear combination of Clayton, Frank and Gumbel copulas and one mixture copula consisting of the optimal linear combination of Clayton, t and Gumbel copulas. Following \citet*{ggr06} we calculate returns using two weighting schemes: the return on committed capital and on fully invested capital. The former commits\footnote{We assume zero return for non-open pairs, although in practice one could earn returns on idle capital.}equal amounts of capital to each one of the pairs even if the pair has not been traded\footnote{The commited capital is considered more realistic as it takes into account the opportunity cost of the capital that has been allocated for trading.}, whereas the latter divides all capital between the pairs that are open.
	
We compare the performance out-of-sample of the strategies using a variety of criteria, all of which are computed using a rolling period procedure similar to that used by \citet*{ggr06} with the exception that the time horizon of formation and trading periods are rolled forward by six months as in \citet*{bv12}. The main criteria we focus are: (1) mean and cumulative excess return, (2) risk-adjusted metrics as Sharpe and Sortino ratios, (3) \% of negative trades, (4) t-values for various risk factors, (5) maximum drawdown between two consecutive days and between two days within a period of maximum six month and (6) total number of pairs opened.
	
	In order to find if pairs trading profitability is associated to exposure to different systematic risk factors\footnote{The single-factor capital asset pricing model (CAPM) of \citet*{s64} and \citet*{l65}, as well as its consumption based version (\citet*{b79}), among other extensions, has been empirically tested and rejected by numerous studies, which show that the cross-sectional variation in expected equity returns cannot be explained by the market beta alone, providing evidence that investors demand compensation for not being able to diversify firm-specific characteristics.}, we regress daily excess returns on seven factors: daily \citet*{ff15}'s five research factors \footnote{\citet*{ff15} found evidences that the three factor model was not sufficient to explain a lot of the variation in average returns related to profitability and investment.} plus momentum and long-term reversal. We find that the intercept is statistically greater than zero for all regressions at 1\% level when considering the mixed copula strategy, showing that our results are robust to the augmented \citet*{ff15}'s risk adjustment factors. In addition, the share of observations with negative excess returns is smaller for the mixed copula than distance strategy.
	
	To test the statistical significance of the returns, standard deviations and Sharpe ratio we use the stationary bootstrap of \citet*{pr94} using the automatic block-length selection of \citet*{pw04} and 10,000 bootstrap resamples. To compute the bootstrap p-values we use the methodology proposed by \citet*{lw08}. We aim to compare the results on a statistical basis to mitigate potential data snooping problems.
	
	The remainder of the paper is organized as follows. The data, a general review of the distance and copula models and the trading strategies we have used are discussed in Section 2. Section 3 summarizes the empirical results of the analysis. Finally, Section 4 provides a brief conclusion. Additional results are reported in the Appendix.
	
	\vspace{0.6cm}
	
	\section{Data and Methodology}
	
	Our data set consists of daily data of adjusted closing prices of all shares that belongs to the S\&P500 market index from July 2nd, 1990 to December 31st, 2015. We obtain adjusted closing prices from Bloomberg and returns on the Fama and French factors from French's website\footnote{\url{http://mba.tuck.dartmouth.edu/pages/faculty/ken.french/Data_Library/det_st_rev_factor_daily.html}}. The data set sample period is made up of 6,426 days and includes a total of 1100 stocks over all periods. Only stocks that are listed during the formation period are included in the analysis, \emph{i.e.}, around 500 stocks in each trading period.
	

	Using data from the Center for Research in Security Prices (CRSP) from 1980 to 2006, \citet*{french2008} estimates that the cost of active investing, including total commissions, bid-ask spreads, and other cost investors pay for trading services, has dropped from 146 basis points in 1980 to 11 basis points in 2006. Considering the US stock live trades on the Nyse-Amex between August 1998 and September 2013 for a large institucional investor, \citet*{fim15} estimate that the average trading costs for market impact (MI) and implementation shortfall methodology (IS) are 8.81 and 9.13 basis points, respectively, while the median trading costs are 6.24 and 7.63 basis points, respectively. For our data we assume trading costs on the order of 10 and 20 basis points.
	
	\subsection{Distance Framework}
	
	Our implementation of the distance strategy is similar to \citet*{ggr06} and \citet*{bv12}. We calculate the spread between the normalized daily closing prices (known as distance) of all combinations of stocks pairs during the next 12 months, named formation period, adjusting them by dividends, stock splits and other corporate actions. Specifically, the pairs are formed using data from January to December or from July to June. Prices are scaled to \$1 at the beginning of each formation period and then evolve using the return series\footnote{%
		Missing values have been interpolated.}.  We then select the top 5, 10,...,35 of those combinations that have the smallest sum of squared spreads, allowing re-selection of a specific pair, during the formation period. These pairs are then traded in the next six months (trading period).
	

In \citet*{ggr06}, when the spread diverges by two or more standard deviations (which is calculated in the formation period) from the mean, the stocks are assumed to be mispriced in terms of their relative value to each other and thus we open a short position in the outperforming stock and a long in the underperforming one. 

The price divergence is expected to be temporary, i.e., the prices are expected to converge to its long-term mean value of 0 (mean-reverting behavior). Hence, the positions are closed once the normalized prices cross. The pair is then monitored for another divergence and thus a pair can complete multiple round-trip trades. Trades that do not converge can result in a loss if they are still open at the end of the trading period when they are automatically closed. This results in fat left tails. Therefore, since the conditional variance is empirically higher for large negative returns and smaller for positive returns, it may be inappropriate to use constant trigger points because the volatility differs at different price levels.

To calculate the daily percentage returns for a pair, we compute%
	\begin{equation}
	\begin{aligned}
	r_{pt}=w_{1t}r_{t}^{L}-w_{2t}r_{t}^{S},
	\end{aligned}
	\label{eq:eq34}
	\end{equation}
	where $L$ and $S$ stands for long and short, respectively. Returns $r_{pt}$ can be interpreted as excess returns since in \eqref{eq:eq34} the riskless rate is canceled out when one calculates the long and short excess returns. The weights $w_{1t}$ and $w_{2t}$ are initially assumed to be one. After that, they change according to the changes in the value of the stocks, \emph{i.e.}, $w_{it}=w_{it-1}(1+r_{it-1})$.
	
	\vspace{0.6cm}
	
	\subsection{Copula Framework}
	
	The notion of copula was first introduced by \citet*{sklar1959}. Sklar's theorem states that any multivariate joint distribution can be written in terms of their univariate marginal distribution function and the dependence structure (represented in $C$) between the variables. We state the Sklar's Theorem below.
	
	\vspace{0.6cm}
	
    \begin{theorem}
		(Sklar's Theorem) Let $X_{1},...,X_{d}$ be random variables with
		distribution functions $F_{1},...,F_{d}$, respectively. Then, there exists
		an d-copula C such that,
		\begin{equation}
		F\left( x_{1},...,x_{d}\right) =C\left( F_{1}\left( x_{1}\right)
		,...,F_{d}\left( x_{d}\right) \right) ,  \label{21}
		\end{equation}
	\end{theorem}for all $\mathbf{x}=\left( x_{1},...,x_{d}\right) \in
	%BeginExpansion
	\mathbb{R}
	%EndExpansion
	^{d}$. Conversely, if $C$ is an $d$-copula and $F_{1},...,F_{d}$ are
	distribution functions, then the function $F$ defined by $\left( \ref{21}\right) $ is a $d-$%
	dimensional distribution function with margins $F_{1},...,F_{d}$. Furthermore, if the marginals $F_{1},...,F_{d}$ are all continuous, $C$ is
	unique. Otherwise $C$ is uniquely determined on $\Ima F_{1}\times ...\times \Ima %
	F_{d}$.
	
	Using the Sklar's theorem we can derive an important
	relation between the marginal distributions and a copula. Let $f$ be a
	joint density function (of the $d-$dimensional distribution function $F$)
	and $f_{1},...,f_{d}$ univariate density functions of the margins $%
	F_{1},...,F_{d}$. Assuming that $F\left( \cdot \right) $ and $C\left( \cdot
	\right) $ are differentiable, by $\left( \ref{21}\right)$ we have
	
	\begin{eqnarray}
	\frac{\partial ^{d}F\left( x_{1},...,x_{d}\right) }{\partial
		x_{1}...\partial x_{d}} &\equiv &f\left( x_{1},...,x_{d}\right) =\frac{
		\partial ^{d}C\left( F_{1}\left( x_{1}\right) ,...,F_{d}\left( x_{d}\right)
		\right) }{\partial x_{1}...\partial x_{d}} \\
	&=&c\left( u_{1},...,u_{d}\right) \prod_{i=1}^{d}f_{i}\left( x_{i}\right),
	\label{23}
	\end{eqnarray}%
	where $u_{i}$=$F_{i}\left( x_{i}\right) $, $i=1,...,d$. Thus, copulas are functions that connect a multivariate distribution function and their marginal distributions. Thereafter, copulas accommodate various forms of dependence through suitable choice of the copula ``correlation matrix'' since they conveniently separate marginals from dependence component. These carry on all relevant information about the dependence structure between random variables and allow a greater flexibility in modeling multivariate distributions and their margins. The methodology allows one to derive joint distributions from marginals, even when these are not normally distributed. In fact, copulas allow the marginal distributions to be modeled independently from each other, and no assumption on the joint behavior of the marginals is required, which provides a great deal of flexibility in modeling joint distributions.  From a modelling perspective, Sklar’s Theorem allow us to estimate the multivariate distribution in two parts: (i) finding the marginal distributions; (ii) finding the dependency between the filtered data from (i).
	
	The choice of the copula function is also not dependent on the marginal distributions. Thus, by using copulas, the linearity restriction that applies to the dependence structure of multivariate random variables in a traditional dependence setting is relaxed. Therefore, depending on the chosen copulas, different dependence structures can be modeled to allow for any asymmetries\footnote{Copulas measures lower and upper tail dependencies and nonlinear and linear relationships in a rich set for describing dependencies between pairs. Copula is also invariant under strictly monotonic transformations \citep{cherubini04,nelsen06} and hence the same copula is obtained if we use price or return series, for example.}.
	
	A further important property of copulas concerns the partial derivatives of a copula with respect to its variables. Let now $H$ be a bivariate function with marginal distribution functions $F$ and $G$. According to \citet*{sklar1959} then there exists a copula $C:\left[ 0,1\right] ^{2}\rightarrow \left[ 0,1\right] $ such that $H(x_{1},x_{2})=C(F(x_{1}),G(x_{2}))$ for all for all $x_{1},x_{2}$ $\in\mathbf{R^{2}}$. If $F$ and $G$ are continuous, then $C$ is unique; otherwise, $C$ is uniquely determined in $\Ima F\times \Ima G$. Conversely, if $C$ is a copula and $F$ and $G$ are distribution functions, then the function $H$ is a joint distribution function with marginals $F$ and $G$ and we can write
	\begin{equation}
	\begin{aligned}
	C(u_{1},u_{2})=H(F^{-1}(u_{1}),G^{-1}(u_{2})),
	\end{aligned}
	\label{eq:eq07}
	\end{equation}
	where $u_{1}=F(x_{1})$ $\Rightarrow $ $x_{1}=F^{-1}(u_{1})$, $u_{2}=G(x_{2}))\Rightarrow $ $x_{2}=G^{-1}(u_{2})$ and $F^{-1}$ and $G^{-1}$ are the quasi-inverses of $F$ and $G$, respectively. For any copula $C$, $\frac{\partial C\left( u_{1},u_{2}\right) }{\partial u_{1}}$ and $\frac{\partial C\left( u_{1},u_{2}\right) }{\partial u_{2}}$ exist almost everywhere. The proposition below states that the partial derivatives of a copula function corresponds to the conditional probabilities of the random variables \citep[see][]{cherubini04,nelsen06}.
	
	\vspace{0.6cm}
	
	\begin{proposition}
		Let $U_{1}$ and $U_{2}$ be two random variables with distribution $U(0,1)$. Then,
	\end{proposition}
	\begin{eqnarray*}
		P\left( U_{1}\leq u_{1}\left\vert U_{2}=u_{2}\right. \right)  &=&\frac{%
			\partial C\left( u_{1},u_{2}\right) }{\partial u_{2}}=P\left( X_{1}\leq
		x_{1}\left\vert X_{2}=x_{2}\right. \right) , \\
		P\left( U_{2}\leq u_{2}\left\vert U_{2}=u_{1}\right. \right)  &=&\frac{%
			\partial C\left( u_{1},u_{2}\right) }{\partial u_{1}}=P\left( X_{2}\leq
		x_{2}\left\vert X_{1}=x_{1}\right. \right) 
	\end{eqnarray*}
	where
	\begin{equation}
	\begin{aligned}
	\frac{\partial C\left( u_{1},u_{2}\right) }{\partial u_{2}}=P\left( U_{1}\leq u_{1}\left\vert
	U_{2}=u_{2}\right. \right) =\underset{h\rightarrow 0}{lim}P\left( U_{1}\leq u_{1}\left\vert
	u_{2}\leq U_{2}\leq u_{2}+h\right. \right)
	\end{aligned}
	\label{eq:eq08}
	\end{equation}
	and
	\begin{equation}
	\begin{aligned}
	\frac{\partial C\left( u_{1},u_{2}\right) }{\partial u_{1}}=P\left( U_{2}\leq u_{2}\left\vert
	U_{1}=u_{1}\right. \right) =\underset{h\rightarrow 0}{lim}P\left( U_{2}\leq u_{2}\left\vert
	u_{1}\leq U_{1}\leq u_{1}+h\right. \right).
	\end{aligned}
	\label{eq:eq09}
	\end{equation}
	
	By using the fact that the partial derivative of the copula function gives the conditional distribution function, \citet*{xie14} define a measure to denote the degree of mispricing:
	
	\vspace{0.6cm}
	
	\begin{definition}
		(Mispricing Index) Let $R_{t}^{X}$ and $R_{t}^{Y}$ represent the random variables of the daily returns of stocks $X$ and $Y$ on time $t$, and
		the realizations of those returns on time t are $r_{t}^{X}$ and $r_{t}^{Y}$, we have
		\begin{eqnarray*}
			MI_{X\mid Y}^{t} & = & P(R_{t}^{X}<r_{t}^{X}\mid R_{t}^{Y}=r_{t}^{Y}) \\
			& \text{and}  & \\
			MI_{Y\mid X}^{t} & = & P(R_{t}^{Y}<r_{t}^{Y}\mid R_{t}^{X}=r_{t}^{X}),
		\end{eqnarray*}
		where $MI_{X|Y}$ and $MI_{Y|X}$ are named the mispricing indexes.
	\end{definition}
	
	Therefore, the conditional probabilities $MI_{X\mid Y}^{t}$ and $MI_{Y\mid X}^{t}$ indicate whether the return of $X$ is considered high or low at time $t$, given the information on the return of $Y$ on the time $t$ and the historical relation between the two stocks' returns, and vice-versa. For example, if the value of $MI_{X\mid Y}^{t}$ is equal to $0.5$, $r_{t}^{X}$ is neither too high nor too low given $r_{t}^{Y}$ and their historical relation. In other words, the historical data indicate that, on average, there is an equal number of observations of the return of $X$ being larger or smaller than $r_{t}^{X}$ if the return of stock $Y$ is equal to $r_{t}^{Y}$ and therefore, a conditional value of 0.5 means that the two underlying stocks are considered fairly-valued. In this case, we can say that stock $X$ is fairly priced relative to stock $Y $ on day $t$. 
	
	Given current realizations $r_{t}^{X}$ and $r_{t}^{Y}$, if $F_{X}$ and $F_{Y} $ are the marginal distribution functions of $R_{t}^{X}$ and $R_{t}^{Y} $ and C is the copula connecting $F_{X}$ and $F_{Y}$ , we define $u_{1}=F_{X}\left( r_{t}^{X}\right) $ and $u_{2}=F_{Y}\left( r_{t}^{Y}\right) $, and have
	
	\begin{equation}
	\begin{aligned}
	MI_{X\mid Y}^{t}& = &\frac{\partial C(u_{1},u_{2})}{\partial u_{2}} & = & P(R_{t}^{X}<r_{t}^{X}\mid R_{t}^{Y}=r_{t}^{Y}) \\
	& \text{and} & \\
	MI_{Y\mid X}^{t}& = &\frac{\partial C(u_{1},u_{2})}{\partial u_{1}}& = & P(R_{t}^{X}<r_{t}^{X}\mid R_{t}^{Y}=r_{t}^{Y}).
	\end{aligned}
	\label{eq:eq31}
	\end{equation}
	\vspace{0.6cm}
	
	Note that the conditional probabilities, $M_{t}^{X\left\vert Y\right. }$ and $%
		M_{t}^{Y\left\vert X\right. }$ , only measure the degrees of relative
		mispricing for a single day. To determine an overall degree of relative
		mispricing we follow \citet*{rf15}. Initially, let $m_{1,t}$ and $m_{2,t}$ be the
		overall mispricing indexes of stocks $X_{1}$ and $X_{2}$, defined by $\left(
		M_{t}^{X\left\vert Y\right. }-0.5\right) $ and $\left( M_{t}^{Y\left\vert
			X\right. }-0.5\right) $, respectively. At beggining of each
		trading period two cumulative mispriced indexed $M_{1}$ and $M_{2}$ are set
		to zero and then evolve for each day through%

	
	\begin{eqnarray*}
		M_{1,t} &=&M_{1,t-1}+m_{1,t} \\
		M_{2,t} &=&M_{2,t-1}+m_{2,t}
	\end{eqnarray*}
	
	Positive (negative) $M_{1}$ and negative (positive) $M_{2}$ are interpreted
		as stock 1 (stock 2) being overvalued relative to stock 2 (stock 1).
		
	We
		perform a sensitivity analysis to open a long-short position once one of the
		cumulative indexes is above 0.05, 0.10,..., 0.55 and the other one is below
		-0.05, -0.10,....,-0.55 at the same time for Top 5, 10,....,35 pairs. The
		positions are unwound when both cumulative mispriced indexes return to zero.
		The pairs are then monitored for other possible trades throughout the
		remainder of the trading period.
		
		%	For our copula approach, we find that 0.75 is a good combination for the mispricing indexes during our backtesting analysis\footnote{We search over a grid from 0.55 to 0.95 with a step of size 0.05.}. This would be similar to the 0.75$\sigma$ trigger point if data is normally distributed.
	
	\citet*{rf15} propose the following steps to obtain $M_{1,t}$ and $M_{2,t}$ using copulas: (1) First, we calculate daily returns for each stock during the formation period and
estimate the marginal distributions of these returns separately by fitting an appropriate ARMA(p,q)-GARCH(1,1) model\footnote{We look for the best ARMA(p,q) model up to order (1,1).} to each univariate time series by obtaining the estimates $\widehat{\mu }_{i}$ and $\widehat{\sigma }_{i}$ of the conditional mean and variance of these processes, respectively. Moreover, using the estimated parametric models, we construct the standardized residuals vectors given, for each $i=1,...,t$, by
	\begin{equation}
	\begin{aligned}
	\widehat{\varepsilon }_{i}=\frac{x_{i}-\widehat{\mu }_{i}}{\widehat{\sigma }_{i}}.
	\end{aligned}
	\label{eq:eq32}
	\end{equation}
	
	The standardized residuals vectors are then converted to the pseudo-observations $z_{i}=\frac{n}{n+1}F_{i}\left( \widehat{\varepsilon }_{i}\right) $, where $F_{i}$ is estimated by using their empirical distribution function; 
	
	\vspace{0.3cm}
	\vspace{0.3cm}
	
	(2) After obtaining the estimated marginal distributions from the previous step
	, we estimate the two-dimensional copula model to data that has been transformed to [0,1] margins to connect the joint distributions with the marginals $F_{X}$ and $F_{Y}$, i.e.,
		\[
		H\left( r_{t}^{X},r_{t}^{Y}\right) =C\left(F_{X}\left( r_{t}^{X}\right)
		,F_{Y}\left( r_{t}^{Y}\right) \right) , 
		\]%
		where $H$ is the joint distribution, $r_{t}^{X}$ e $r_{t}^{Y}$ are stock
		returns and $C$ is the copula that best fits the uniform marginals and estimate its parameter(s). Copulas that are tested in this step are Gaussian, t, Clayton, Frank, Gumbel, one Archimedean mixture copula consisting of the optimal linear combination of Clayton, Frank and Gumbel copulas and one mixture copula consisting of the optimal linear combination of Clayton, t and Gumbel copulas.\footnote{Archimedean copulas contain different tail dependence characteristics. Clayton and Gumbel are nonsymmetric copulas that describe more accurately lower and upper tail dependence, respectively. The Frank copula is the only bivariate reflection symmetric Archimedean family but it has different properties when compared to the bivariate gaussian and bivariate t copulas. Hence, by using mixture copulas we cover a wider range of possible dependencies within a single model.}.
	
	Specifically, a mixture of Clayton, Frank and Gumbel copulas and a mixture of  Clayton, t and Gumbel copulas can be written, respectively, as
	\begin{equation}
	\mathcal{C}_{\theta}^{CFG}\left(u_{1},u_{2}\right)=\pi_{1}\mathcal{C}_{\alpha}^{C}\left(u_{1},u_{2}\right)+\pi_{2}\mathcal{C}_{\beta}^{F}\left(u_{1},u_{2}\right)+\left(1-\pi_{1}-\pi_{2}\right)\mathcal{C}_{\delta}^{G}\left(u_{1},u_{2}\right),
	\label{eq:eq24}
	\end{equation}
	
	and
	
	\begin{equation}
	\mathcal{C}_{\xi}^{CFG}\left(u_{1},u_{2}\right)=\pi_{1}\mathcal{C}_{\alpha}^{C}\left(u_{1},u_{2}\right)+\pi_{2}\mathcal{C}_{\Sigma,\nu}^{t}\left(u_{1},u_{2}\right)+\left(1-\pi_{1}-\pi_{2}\right)\mathcal{C}_{\delta}^{G}\left(u_{1},u_{2}\right),
	\label{eq:eq25}
	\end{equation}
	where $\theta=\left(\alpha,\beta,\delta\right)'$ are the Clayton, Frank and Gumbel copula (dependence) parameters and $\xi=\left(\alpha,(\Sigma,\nu),\delta\right)'$ are the Clayton, t and Gumbel copula parameters, respectively, and $\pi_{1}$, $\pi_{2} \in [0,1]$. The estimates are obtained by the minimization of the negative log-likelihood consisting of the weighted densities of the copulas; 
	
	\vspace{0.3cm}
	\vspace{0.3cm}
	
	(3) Take the first derivative of the copula function to compute conditional
		probabilities and measure mispricing degrees $MI_{X\mid Y}$ and $MI_{Y\mid X}$ for each day in the trading period using the copula and estimated parameters;
		
	\vspace{0.3cm}
	\vspace{0.3cm}
		
	(4) Build long and short positions of $Y$ and $X$ on the days that $M_{1,t}>\Delta_{1}$ and $M_{2,t}<\Delta_{2}$ if there is no positions in $X$ or $Y$. Conversely, build positions long/short of $X$ and $Y$ on the day that $M_{1,t}<\Delta_{2}$ and $M_{2,t}>\Delta_{1}$ if there is no positions in $X$ or $Y$;
	
	\vspace{0.3cm}
	\vspace{0.3cm}
		
	(5) All positions are closed if $M_{1,t}$ reaches $\Delta_{3}$ or $M_{2,t}$ reaches $\Delta_{4}$, where $\Delta_{1},\Delta_{2},\Delta_{3}$ and $\Delta_{4}$ are predetermined thresholds or are automatically closed out on the last day of the trading period if they do not reach the thresholds. Here we use $\Delta_{1}=0.2, \Delta_{2}=-0.2$ and $\Delta_{3}=\Delta_{4}=0$.
	
	
	\vspace{0.6cm}
	
	\section{Empirical Results}
	
	\vspace{0.3cm}
	
	\subsection{Profitability of the Strategies}
	
First, we analyze the sensitivity of the outcomes when the opening thresholds are changed to top 5-35 pairs we provide multiple boxplots in Figures \ref{fig:fig26} and \ref{fig:fig42}. They report annualized excess returns (Figure \ref{fig:fig26}) and annualized Sharpe ratios (Figure \ref{fig:fig42}) for each of the strategies from 1991/2-2015 on commited capital and on fully invested capital after costs (10 bps)\footnote{The numerical experiments show that the performances out-of-sample stay very similar when we consider 20 bps. Since the results are very much alike they are not presented here and are available under request.}. Pairs are formed based on the smallest sum of squared deviations. The last boxplot (from left to right) show the performance for the distance strategy, while the others report the outcomes using multiple opening trigger points for the cumulative mispriced indexes $M_{1}$ and $M_{2}$ (0.05, 0.1, 0.15, 0.2, 0.25, 0.3, 0.35, 0.4, and 0.55). Based on these outcomes we perform the subsequent analyses considering 0.2 as the opening threshold for the mixed copula strategy.

\begin{figure}[H]
	\centering
	%\includegraphics[scale=0.26]{fig26.pdf}
	\includegraphics[width=18cm,height=7.5cm]{fig26.pdf}
	\caption{\textbf{Annualized returns of pairs trading strategies after costs on committed and fully invested capital}}
	\caption*{\scriptsize These boxplots show annualized returns on committed (left) and fully invested (right) capital after transaction cost to different opening thresholds from July 1991 to December 2015 for Top 5 to Top 35 pairs. Pairs are formed based on the smallest sum of squared deviations.}
	\label{fig:fig25}
\end{figure}


\begin{figure}[H]
	\centering \tiny
	\includegraphics[width=18cm,height=7.5cm]{fig42.pdf}
	\caption{\textbf {Sharpe ratio of pairs trading strategies after costs on committed and fully invested capital}}
	\caption*{\justifying \scriptsize These boxplots show Sharpe ratios on committed (left) and fully invested (right) capital after transaction cost to different opening thresholds from July 1991 to December 2015 for Top 5 to Top 35 pairs. Pairs are formed based on the smallest sum of squared deviations.}
	\label{fig:fig42}
\end{figure}

Table \ref{tab:table101} reports annualized excess returns, annualized Sharpe and Sortino ratios, \citet*{nw87} adjusted t-statistics, share of negative observations, the maximum drawdown in terms of maximum percentage drop between two consecutive days (MDD1) and between two days within a period of maximum six months (MDD2) for both strategies from 1991/2-2015, for Top 5 (Panel A), Top 20 (Panel B), and Top 35 (panel C) pairs after costs (10 bps) \footnote{The numerical experiments show that the performances out-of-sample stay very similar when we consider 20 bps. The outcomes are also robust for the other number of pairs considered. Since the results are very much alike they are not presented here and are available under request. }. Furthermore, Section 1 shows the Return on Committed Capital and Section 2 on Fully Invested Capital. %Panel A lists the results after transaction costs and Panel B before transaction costs. 	
	
By analyzing Table \ref{tab:table101}, it is possible to observe a series of important facts. First, note that the copula-based pairs strategy outperforms the distance method for all metrics for Top 5 pairs and committed capital. The mixed copula strategy yields the highest average excess returns and reach a Sharpe ratio of 0.63, slightly more than double what we get from investing in the tradicional distance method. The Sortino ratio confirms that the mixed copula model offers better risk-adjusted returns. The statistics also indicate that the mixed copula model delivers the highest t-statistics (statistically and economically significant at 1\%) and a lower probability of a negative trade, where the share of days with negative returns (41.79\%) is consistently less than the market performance (47.45\% of negative returns over the period). Furthermore, the summary statistics also show that mixed copula method offers better hedges against losses than the distance strategy for Top 5 pairs on committed capital when considering the downside risk statistics $MDD1$ and $MDD2$. We find that the number of tradeable signals is only equiparable in this study for the different strategies for Top 5 pairs. We will explore this point further in the next subsection.

 The listed results for Top 20 and Top 35 pairs on committed capital show that the distance strategy is more profitable than the mixed copula method, although the Sharpe ratios are similar, although the sharpe ratios are similar, indicating that returns are alike when we take into account the risks taken. All profits are now economically and statistically significant at 1\%. Overall, the copula method is again a less risky strategy regarding the drawdown measures.

Section 2 of Table \ref{tab:table101} shows results on fully invested capital scheme. We can note that this approach yields a higher Sharpe and Sortino ratios for the copula-based strategy and the excess return of the portfolio averaged 11.6\% a year (almost thrice the return of the committed capital approach), with large and significant Newey-West adjusted t-statistic of 4.26 for Top 5 pairs. Apart the tail risk (drawdown) measures, the mixed copula method consistently outperforms the distance strategy for all pairs considered.

%albeit with higher volatility.





In this case, we obtain mixed results. We can note that the copula strategy is superior or very comparable to the distance strategies on committed capital, particularly when considering a period of six months (MDD2) and is usually the worst on fully invested capital, especially for the drop between two consecutive days (MDD1).

  Sharpe ratio on committed capital above 1 for Top 20 pairs (1.72 and 1.03 before and after costs, respectively), indicating that the returns on committed capital are greater than the risk taken. The Sortino ratio confirms that the copula method offers better risk-adjusted returns

The copula-based method also shows the highest average excess returns on fully invested capital (up to 17.53\%) and before costs on committed capital. Although the outcomes are not the best on committed capital after costs, the strategy still has a competitive advantage since it delivers the highest risk-adjusted statistics among the strategies after considering the costs.
	
By analyzing Tables \ref{tab:table101} and \ref{tab:table102}, it is possible to observe a series of important facts. First, note that the copula-based pairs strategy outperforms the distance strategies when considering the Sharpe and Sortino risk-return ratios for Top 5 and Top 20 pairs, particularly before costs. The copula strategy yields a Sharpe ratio on committed capital above 1 for Top 20 pairs (1.72 and 1.03 before and after costs, respectively), indicating that the returns on committed capital are greater than the risk taken. The Sortino ratio confirms that the copula method offers better risk-adjusted returns. The statistics also indicate that the copula model delivers the highest t-statistics (all statistically and economically significant at 1\%) and a lower probability of a negative trade, especially for Top 5 pairs, where the share of days with negative returns (38\%-39\%) is consistently less than the market performance (47.45\% of negative returns over the period).

Overall, the summary statistics show that copula method is a less risky strategy for Top 5 and Top 20 pairs, except when considering the drawdown measures. In this case, we obtain mixed results. We can note that the copula strategy is superior or very comparable to the distance strategies on committed capital, particularly when considering a period of six months (MDD2) and is usually the worst on fully invested capital, especially for the drop between two consecutive days (MDD1).

The copula-based method also shows the highest average excess returns on fully invested capital (up to 17.53\%) and before costs on committed capital. Although the outcomes are not the best on committed capital after costs, the strategy still has a competitive advantage since it delivers the highest risk-adjusted statistics among the strategies after considering the costs.

\begin{threeparttable}[H]
	\centering \tiny
	\caption{Excess returns of pairs trading strategies on portfolios of Top 5, 20 and 35 pairs after costs.}
	\begin{tabularx}{\textwidth}{@{\extracolsep{\fill}}llllllll@{}}
		\toprule
		Strategy & Mean  & Sharpe & Sortino & t-stat & \% of negative   & MDD1 & MDD2 \\
		& Return (\% ) & ratio &  ratio     &  &  trades     &       &  \\
		\midrule
		\multicolumn{8}{c}{\textbf{Section 1: Return on Committed Capital}} \\
		\multicolumn{8}{c}{\textit{Panel A - Top 5 pairs}} \\
		&       &       &       &       &       &       &  \\
		Distance & 2.60  & 0.31  & 0.58  & $1.86^{*}$  & 46.98 & -6.73    & -19.62 \\
		Mixed Copula &  3.98  &  0.63  & 1.08  &  $3.49^{***}$  &  41.79 &  -4.36  &  -9.29 \\
		\multicolumn{1}{r}{} & \multicolumn{1}{r}{} & \multicolumn{1}{r}{} & \multicolumn{1}{r}{} & \multicolumn{1}{r}{} & \multicolumn{1}{r}{} & \multicolumn{1}{r}{} & \multicolumn{1}{r}{} \\
		\multicolumn{8}{c}{\textit{Panel B - Top 20 pairs}} \\
		&       &       &       &       &       &       &  \\
		Distance &  3.14  &  0.65  & 1.13  & $3.32^{***}$  & 48.02 & -3.88  & -9.69 \\
		Mixed Copula  & 1.24  & 0.64  & 1.04  & $3.52^{***}$  & 41.33 &  -2.07  &  -3.43  \\
		\multicolumn{1}{r}{} & \multicolumn{1}{r}{} & \multicolumn{1}{r}{} & \multicolumn{1}{r}{} & \multicolumn{1}{r}{} & \multicolumn{1}{r}{} & \multicolumn{1}{r}{} & \multicolumn{1}{r}{} \\
		\multicolumn{8}{c}{\textit{Panel C - Top 35 pairs}} \\
		&       &       &       &       &       &       &  \\
		Distance &  3.06  &  0.76  & 1.34  & $3.87^{***}$  & 48.02 & -2.70  & -7.52 \\
		Mixed Copula & 0.82  & 0.73  & 1.19  & $3.95^{***}$  & 41.31 &  -1.18  &  -1.98  \\
		\multicolumn{1}{r}{} & \multicolumn{1}{r}{} & \multicolumn{1}{r}{} & \multicolumn{1}{r}{} & \multicolumn{1}{r}{} & \multicolumn{1}{r}{} & \multicolumn{1}{r}{} & \multicolumn{1}{r}{} \\
		\midrule
		\multicolumn{8}{c}{\textbf{Section 2: Return on Fully Invested Capital}} \\
		\multicolumn{8}{c}{\textit{Panel A - Top 5 pairs}} \\
		&       &       &       &       &       &       &  \\
		Distance & 4.01  & 0.28  & 0.57  & $1.81^{*}$  & 46.98 & -8.70    & -38.36  \\
		Mixed Copula & 11.58  & 0.78  & 1.43  & $4.26^{***}$  & 41.79 & -9.00  & -25.68 \\
		\multicolumn{1}{r}{} & \multicolumn{1}{r}{} & \multicolumn{1}{r}{} & \multicolumn{1}{r}{} & \multicolumn{1}{r}{} & \multicolumn{1}{r}{} & \multicolumn{1}{r}{} & \multicolumn{1}{r}{} \\
		\multicolumn{8}{c}{\textit{Panel B - Top 20 pairs}} \\
		&       &       &       &       &       &       &  \\
		Distance & 6.12  & 0.67  & 1.20  & $3.58^{***}$  & 48.02 & -5.43  & -20.03 \\
		Mixed Copula  & 12.30  & 0.85  & 1.54  & $4.60^{***}$  & 41.31 & -9.00  & -25.68  \\
		\multicolumn{1}{r}{} & \multicolumn{1}{r}{} & \multicolumn{1}{r}{} & \multicolumn{1}{r}{} & \multicolumn{1}{r}{} & \multicolumn{1}{r}{} & \multicolumn{1}{r}{} & \multicolumn{1}{r}{} \\
		\multicolumn{8}{c}{\textit{Panel C - Top 35 pairs}} \\
		&       &       &       &       &       &       &  \\
		Distance & 5.71  & 0.75  & 1.37  & $4.01^{***}$  & 48.02 & -4.23  & -15.07 \\
		Mixed Copula & 12.73  & 0.88  & 1.59  & $4.73^{***}$  & 41.28 & -9.00  & -25.68  \\
		\multicolumn{1}{r}{} & \multicolumn{1}{r}{} & \multicolumn{1}{r}{} & \multicolumn{1}{r}{} & \multicolumn{1}{r}{} & \multicolumn{1}{r}{} & \multicolumn{1}{r}{} & \multicolumn{1}{r}{} \\
		\bottomrule
	\end{tabularx}\%
	\begin{tablenotes}
		\item \textit{Note:} \scriptsize \tiny Summary statistics of the annualized excess returns, annualized Sharpe and Sortino ratios on portfolios of top 5, 20 and 35 pairs between July 1991 and December 2015 (6,173 observations). \textcolor{blue} {Pairs are formed based on the smallest sum of squared deviations}. The t-statistics are computed using Newey-West standard errors with a six-lag correction. The columns labeled MDD1 and MDD2 compute the largest drawdown in terms of maximum percentage drop between two consecutive days and between two days within a period of maximum six months, respectively.
		\item \scriptsize $^{\ast\ast\ast}$, $^{\ast\ast}$, $^{\ast}$  significant at 1\%, 5\% and 10\% levels, respectively.
	\end{tablenotes}
	\label{tab:table101}
\end{threeparttable}

\vspace{0.6cm}

%\begin{threeparttable}[H]
%	\centering \tiny
%	\caption{Excess returns on committed capital of pairs trading strategies on portfolios of Top 5, 20 and 35 pairs after costs  using a L1-norm minimization. }
%	\begin{tabularx}{\textwidth}{@{\extracolsep{\fill}}llllllll@{}}
%		\toprule
%		Strategy & Mean  & Sharpe & Sortino & t-stat & \% of negative   & MDD1 & MDD2 \\
%		& Return (\% ) & ratio &  ratio     &  &  trades     &       &  \\
%		\midrule
%		\multicolumn{8}{c}{\textbf{Section 1: Return on Committed Capital}} \\
%		\multicolumn{8}{c}{\textit{Panel A - Top 5 pairs}} \\
%		&       &       &       &       &       &       &  \\
%		Distance &  3.83  & 0.46  & 0.85  & $2.64^{***}$  & 46.83 & -6.12    & -18.85 \\
%		Mixed Copula & 3.51  &  0.58  & 1.00  &  $3.27^{***}$  &  40.74 &  -3.73  &  -11.12 \\
%		\multicolumn{1}{r}{} & \multicolumn{1}{r}{} & \multicolumn{1}{r}{} & \multicolumn{1}{r}{} & \multicolumn{1}{r}{} & \multicolumn{1}{r}{} & \multicolumn{1}{r}{} & \multicolumn{1}{r}{} \\
%		\multicolumn{8}{c}{\textit{Panel B - Top 20 pairs}} \\
%		&       &       &       &       &       &       &  \\
%		Distance &  2.85  & 0.57  & 1.00  & $ 2.94^{***}$  & 47.85 & -4.27  & -11.38 \\
%		Mixed Copula  & 1.42  &  0.80  & 1.35  & $4.42^{***}$  & 40.34  & -1.29  & -3.02  \\
%		\multicolumn{1}{r}{} & \multicolumn{1}{r}{} & \multicolumn{1}{r}{} & \multicolumn{1}{r}{} & \multicolumn{1}{r}{} & \multicolumn{1}{r}{} & \multicolumn{1}{r}{} & \multicolumn{1}{r}{} \\
%		\multicolumn{8}{c}{\textit{Panel C - Top 35 pairs}} \\
%		&       &       &       &       &       &       &  \\
%		Distance &  3.49  &   0.84  & 1.49  & $4.30^{***}$  & 47.76 & -2.35  & -7.90 \\
%		Mixed Copula & 0.83  & 0.80  & 1.35  & $4.40^{***}$  & 40.34 & -0.74  & -1.82  \\
%		\multicolumn{1}{r}{} & \multicolumn{1}{r}{} & \multicolumn{1}{r}{} & \multicolumn{1}{r}{} & \multicolumn{1}{r}{} & \multicolumn{1}{r}{} & \multicolumn{1}{r}{} & \multicolumn{1}{r}{} \\
%		\bottomrule
%		\multicolumn{8}{c}{\textbf{Return on Fully Invested Capital}} \\
%		\multicolumn{8}{c}{\textit{Panel A - Top 5 pairs}} \\
%		&       &       &       &       &       &       &  \\
%		Distance & 7.42  & 0.49  & 1.00  & $2.97^{***}$  & 46.83 & -7.87    & -28.40 \\
%		Mixed Copula & 10.06  & 0.66  & 1.18  & $3.69^{***}$  & 40.73 & -12.93  & -43.71 \\
%		\multicolumn{1}{r}{} & \multicolumn{1}{r}{} & \multicolumn{1}{r}{} & \multicolumn{1}{r}{} & \multicolumn{1}{r}{} & \multicolumn{1}{r}{} & \multicolumn{1}{r}{} & \multicolumn{1}{r}{} \\
%		\multicolumn{8}{c}{\textit{Panel B - Top 20 pairs}} \\
%		&       &       &       &       &       &       &  \\
%		Distance & 5.49  & 0.59  & 1.05  & $ 3.15^{***}$  & 47.85 & -5.51  & -24.74 \\
%		Mixed Copula  & 11.78  & 0.78  & 1.37  & $4.26^{***}$  & 40.32  & -12.93  & -42.81  \\
%		\multicolumn{1}{r}{} & \multicolumn{1}{r}{} & \multicolumn{1}{r}{} & \multicolumn{1}{r}{} & \multicolumn{1}{r}{} & \multicolumn{1}{r}{} & \multicolumn{1}{r}{} & \multicolumn{1}{r}{} \\
%		\multicolumn{8}{c}{\textit{Panel C - Top 35 pairs}} \\
%		&       &       &       &       &       &       &  \\
%		Distance & 6.62  & 0.85  & 1.55  & $4.49^{***}$  & 47.76 & -3.56  & -15.38 \\
%		Mixed Copula & 11.61 & 0.77  & 1.35  & $4.22^{***}$  & 40.22 & 12.93 & 42.87 \\
%		\multicolumn{1}{r}{} & \multicolumn{1}{r}{} & \multicolumn{1}{r}{} & \multicolumn{1}{r}{} & \multicolumn{1}{r}{} & \multicolumn{1}{r}{} & \multicolumn{1}{r}{} & \multicolumn{1}{r}{} \\
%		\bottomrule
%	\end{tabularx}
%	\begin{tablenotes}
%		\item \textit{Note:} \tiny Summary statistics of the annualized excess returns, annualized Sharpe and Sortino ratios on portfolios of top 5, 20 and 35 pairs between July 1991 and December 2015 (6,173 observations). \textcolor{blue} {Pairs are formed based on the smallest sum of absolute deviations}. The t-statistics are computed using Newey-West standard errors with a six-lag correction. The columns labeled MDD1 and MDD2 compute the largest drawdown in terms of maximum percentage drop between two consecutive days and between two days within a period of maximum six months, respectively.
%		\item \scriptsize $^{\ast\ast\ast}$, $^{\ast\ast}$, $^{\ast}$  significant at 1\%, 5\% and 10\% levels, respectively.
%	\end{tablenotes}
%	\label{tab:table102}\%
%\end{threeparttable}\%

	
	
All strategies have a significant alpha to make it a viable standalone trading strategy. Despite this, the strategies can be combined to improve the relative performance. (maybe at the conclusion as future recommendation)

	
	\vspace{1.0cm}
	
	Table \ref{tab:table103} displays the results for Top 101-120 pairs, i.e., 20 pairs below the Top 100 pairs. The distance methods appear to have a better out-of-sample performance than in the Top 5 and Top 20 pairs, yielding higher values for most of the performance statistics computed. Compared to the copula strategy, the performance of the distance approaches is consistently better after costs, except for MDD2 on committed capital.
	
	
	\vspace{0.3cm}
	
	Overall, we can notice that the copula strategy suffer considerably more when transaction costs are taking into account than the distance strategies. The excess returns of copula method are reduced by more than 40\% after costs, while the distance strategies profits are reduced by no more than 15\%.
	

	Figures \ref{fig:fig108} to \ref{fig:fig111} show cumulative excess returns through the full data set for every strategy. Figure \ref{fig:fig108} shows the returns with no delay and before costs for Top 5 (top), Top 20 (center) and Top 101-120 pairs (bottom) on committed (left) and fully invested capital (right) whereas Figure \ref{fig:fig109} displays the returns with no delay and after costs, Figure \ref{fig:fig110} with ``one-day rule'' after the signal and before costs, and Figure \ref{fig:fig111} with ``one-day rule'' after the signal and after costs.
	
	The patterns found in the figures bolster the average excess returns and t-statistics displayed in Tables \ref{tab:table101} to \ref{tab:table106}. Overall, we can observe that the copula strategy has a relatively stronger performance when no restrictions are imposed on trade, \emph{i.e.}, before costs and when trades can be executed quickly after the signal, in particular on fully invested capital. When only one of the constraints is attached to trade the performance measures show that the copula method still delivers a relative good out-performance over the whole data set. However, the distance strategies are clearly less affected by the restrictions for Top 5 and Top 20 pairs.
	
	It should be noted that the copula strategy displays, from Figures \ref{fig:fig108} to \ref{fig:fig110}, a very poor performance from 1998-2000 for Top 5 pairs. However, it achieves a very favorable out-of-sample performance relative to the benchmark approaches after the subprime mortgage crisis.
	
%	\begin{figure}[H]
%		\centering
%		\includegraphics[scale=0.6]{fig1_p1_rev.pdf}
%		\caption{\textbf{Cumulative excess returns of pairs trading strategies before costs and without delay}}
%		\caption*{This figure shows how an investment of \$1 evolves from July 1991 to December 2015 for each of the strategies.}
%		\label{fig:fig108}
%	\end{figure}
%	
%	\begin{figure}[H]
%		\centering
%		\includegraphics[scale=0.6]{fig2_p1_rev.pdf}
%		\caption{\textbf{Cumulative excess returns of pairs trading strategies after costs and without delay}}
%		\caption*{This figure shows how an investment of \$1 evolves from July 1991 to December 2015 for each of the strategies.}
%		\label{fig:fig109}
%	\end{figure}
%	
%	\begin{figure}[H]
%		\centering
%		\includegraphics[scale=0.6]{fig3_p1_rev.pdf}
%		\caption{\textbf{Cumulative excess returns of pairs trading strategies before costs and with one-day waiting period}}
%		\caption*{This figure shows how an investment of \$1 evolves from July 1991 to December 2015 for each of the strategies.}
%		\label{fig:fig110}
%	\end{figure}

%	\begin{figure}[H]
%	\centering
%	\includegraphics[scale=0.6]{fig4_p1_rev.pdf}
%	\caption{\textbf{Cumulative excess returns of pairs trading strategies after costs and with one-day waiting period}}
%	\caption*{This figure shows how an investment of \$1 evolves from July 1991 to December 2015 for each of the strategies.}
%	\label{fig:fig111}
%\end{figure}
	
	\vspace{0.3cm}
	
	\subsection{Trading statistics}
	
	Table \ref{tab:table107} reports trading statistics. Panel A, B and C report results for Top 5, Top 20 and Top 101-120 pairs, respectively. The average price deviation trigger for opening pairs is listed in the first row of each panel. For any panel, we can observe that for the 0.75$\sigma$ trigger the positions are initiated before, relatively. The positions are initiated when prices have diverged by 3.09\\%, 3.53\\%, and 5.02\\% for Top 5, Top 20, and Top 101-120, respectively. Similar to \citet*{ggr06}, the trigger spread increases with the number of pairs for all approaches.
	
	\begin{threeparttable}[H]
		\centering \scriptsize
		\caption{Trading statistics.}
		\begin{tabularx}{\textwidth}{@{\extracolsep{\fill}}p{7cm}p{1cm}p{1cm}p{1cm}p{1cm}@{}}
			\toprule
			\multicolumn{1}{c}{Strategy} & Distance (2.0$\sigma$) & Distance (0.75$\sigma$) & Copula-GARCH \\
			\midrule
			& \multicolumn{4}{c}{\textit{Panel A: Top 5}} \\
			& & &  \\
		Average price deviation trigger for opening pairs & 0.0622 & 0.0309 & 0.0697   \\
		Total number of pairs opened & 346   & 674   & 3288  \\
		Average number of pairs traded per six-month period & 7.0612 & 13.7551 & 67.1020   \\
		Average number of round-trip trades per pair & 1.4122 & 2.7510 & 13.4204  \\
		~~Standard Deviation & 1.0389 & 2.2560 & 7.7886    \\
		Average time pairs are open in days & 50.8382 & 33.5208 & 2.9793  \\
		~~Standard Deviation & 38.7356 & 37.3741 & 4.3142   \\
		Median time pairs are open in days & 39.5  & 16    & 2       \\
		& & & \\
		& \multicolumn{4}{c}{\textit{Panel B: Top20}} \\
		& & &\\
		Average price deviation trigger for opening pairs & 0.0692 & 0.0353 & 0.0765   \\
		Total number of pairs opened & 1302  & 2502  & 12242   \\
		Average number of pairs traded per six-month period & 26.5714 & 51.0612 & 249.8367  \\
		Average number of round-trip trades per pair & 1.3286 & 2.5531 & 12.4918  \\
		~Standard Deviation & 0.9945 & 2.2627 & 7.8108   \\
		Average time pairs are open in days & 52.7366 & 35.3373 & 3.0734  \\
		~Standard Deviation & 40.1988 & 38.9524 & 4.5812   \\
		Median time pairs are open in days & 42    & 17    & 2       \\
		& & &\\
		& \multicolumn{4}{c}{\textit{Panel C: Top 101-120}} \\
		& & &\\
		Average price deviation trigger for opening pairs & 0.0969 & 0.0502 & 0.1046   \\
		Total number of pairs opened & 1194  & 2210  & 10783  \\
		Average number of pairs traded per six-month period & 24.3673 & 45.1020 & 220.0612 \\
		Average number of round-trip trades per pair & 1.2184 & 2.2551 & 11.0031  \\
		~Standard Deviation & 1.0234 & 2.0475 & 7.3947  \\
		Average time pairs are open in days & 54.7111 & 38.4715 & 3.3447  \\
		~Standard Deviation & 40.2876 & 40.2570 & 4.5950 \\
			Median time pairs are open in days & 45    & 20    & 2      \\
			\bottomrule
		\end{tabularx}\%
		\begin{tablenotes}
			\item \textit{Note:} \scriptsize  Trading statistics for portfolio of top 5, 20 and 101-120 pairs between July 1991 and December 2015 (49 periods). Pairs are formed over a 12-month period according to a minimum-distance criterion and then traded over the subsequent 6-month period. Average price deviation trigger for opening a pair is calculated as the price difference divided by the average of the prices.
		\end{tablenotes}
		\label{tab:table107}\%
	\end{threeparttable}\%
	
	\medskip
	
	The average number of pairs traded per six-month period is about 5 times more for copula approach than for the 0.75$\sigma$ trigger, and more than 9 times the 2.0$\sigma$ trigger over the 49 trading periods\footnote{Clearly, one of the reasons for copula-based pairs strategy being more affected by trading costs is due to a much greater number of trading opportunities. Implementing a stop-loss strategy may lead to higher returns and reduce the standard deviation of returns.}. Each pair is held open, in average, by approximately 3 trading days, which indicates that it is a short-term strategy when using the copula rules. Meanwhile, the average holding period for the 0.75 and 2.0 standard deviations approaches are about 1.6-1.8 and 2.5 trading months, respectively. This indicates that the strategy is a medium-term investment when using the distance approaches.
	
	\medskip
	
	\subsection{Regression on Fama-French asset pricing factors}
	
	In one attempt to understand the economic drivers behind our data and to evaluate whether pairs trading profitability is a compensation for risk, we regress daily excess returns onto different risk factors: (i) daily \citet*{ff15}'s five research factors: the excess return on a broad market portfolio, ($R_{m} - R_{f}$), the difference between the return on a portfolio of small stocks and the return on a portfolio of large stocks ($SMB$, small minus big), the difference between the return on a portfolio of high book-to-market stocks and the return on a portfolio of low book-to-market stocks ($HML$, high minus low) and two additional factors: the difference between the return of the most profitable stocks and the return of the least profitable stocks ($RMW$, robust minus weak), and the difference between the return of stocks that invest conservatively and the return of stocks that invest aggressively ($CMA$, conservative minus aggressive),  \emph{i.e.},
	\begin{equation}
	R_{i,d}-R_{f,d}=\alpha _{i}+\beta _{i}\left( R_{m,d}-R_{f,d}\right)
	+s_{i}SMB_{d}+h_{i}HML_{d}+r_{i}RMW_{d}+c_{i}CMA_{d}+\varepsilon _{i,d},
	\label{eq:eq101}
	\end{equation}
	where $R_{i,d}$ is the return of stock $i$ on day $d$ and $R_{f,d}$ is the daily risk-free rate; (ii) similarly to \citet*{ggr06}, we use another version of daily Fama and French's research factors including the daily \citet*{ff93}'s three factors plus momentum (Mom)\footnote{Mom is the average return on two (big and small) high prior return portfolios minus the average return on the two low prior return portfolios:
		\begin{equation*}
		Mom = 1/2 (\text{Small High} + \text{Big High}) - 1/2(\text{Small Low} + \text{Big Low}).
		\end{equation*}}
	and short-term reversal (Rev)\footnote{Rev is the average return on two (big and small) low prior return portfolios minus the average return on the two high prior return portfolios:
		\begin{equation*}
		Rev = 1/2(\text{Small Low} + \text{Big Low})- 1/2(\text{Small High} + \text{Big High}).
		\end{equation*}}
	factors:
	\begin{equation}
	\begin{aligned}
	R_{i,d}-R_{f,d}&=\alpha _{i}+\beta _{i}\left( R_{m,d}-R_{f,d}\right)+s_{i}SMB_{d}\\
	&~+h_{i}HML_{d}+m_{i}Mom_{d}+v_{i}Rev_{d}+\varepsilon _{i,d}.
	\end{aligned}
	\label{eq:eq104}
	\end{equation}
	
	We selected the model in terms of an approximation to the mean squared prediction error using Bayesian Information Criterion (BIC) \citet*{Schwarz1978} considering the seven regressors contained on tables and short-term reversal. 
	
	\citet*{fama2016} acknowledge that, similar to the 3-factor model, the 5-factor model is unable to explain the momentum effect. They also mention that the focus of the model is on explaining long-term expected returns rather than short-term variation in returns. (Blitz, 2016). We also consider sorts involving prior returns (Momentum, Short and Long-Term Reversals factors).
	
	%We can also note that the Worst Case Copula-CVaR has a superior performance than the Gaussian Copula-CVaR in terms of the downside (tail) riskmetrics as VaR, CVaR and CR.
	
	%The copula-based approaches are more profitable for daily and weekly rebalancing over time, especially after 2009. We can note the copula methods present a hump-shaped pattern in 1999, while the otherbenchmarks show a sharp decline in the subperiod that corresponds to the bear market that comprises thedotcom crisis and the September 11th terrorist attack (2000-2002). All portfolios show a hump-shaped pattern during the subprime mortgage financial crisis in 2007-2008. Overall, we can observe that after 2002 the patterns are similar, but the figure indicates that the copula methods, even though the objective function is the minimization of CVaR under a constraint on expected return, preserve more wealth in thelong-term period,Meanwhile, the risk-adjusted returns of
	%the 2-dimensional copula and distance strategies are 21\% and 27\% lower than their respective
	%raw average excess returns in Table 6, indicating that a relatively larger portion of their raw
	%excess returns are explained by risk factors.
	
	%While the liquidity factor is negatively correlated to all
	%strategies’ returns, we find no evidence of their correlation to market excess returns. All strategies
	%show positive and significant alphas after accounting for various risk-factors. We also find that in
	%addition to all strategies performing better during periods of significant volatility, the cointegration
	%method is the superior strategy during turbulent market conditions.
	%Fama, Eugene F., and Kenneth R. French. 1996. “Multifactor Explanations of Asset Pricing Anomalies.” The Journal of Finance, vol. 51, no. 1 (March): 55–84.
	
	All the data used to fit the above regressions are described in and obtained from Kenneth French’s data library\footnote{\url{http://http://mba.tuck.dartmouth.edu/pages/faculty/ken.french/data_library.html}}.
	
	The main purpose of these regressions is to estimate the intercept alpha - the average excess return not explained by these factors. The standard errors have been adjusted for heteroscedasticity and autocorrelation by using Newey-West adjustment with six lags.
	
	Tables \ref{tab:table108} to \ref{tab:table119} report the coefficients and corresponding Newey-West t-statistics of regressing daily portfolio return series onto \citet*{ff93}'s three research factors plus momentum and short-term reversal (even numbered tables) and \citet*{ff15}'s five research factors (odd numbered tables) for each of the four strategies from 1990/2-2015, before and after transaction costs, for Top 5, Top 20 and Top 101-120 pairs. Tables \ref{tab:table108} to \ref{tab:table113} summarize results when positions are initiated and closed in the same day the pair diverges, and Tables \ref{tab:table114} to \ref{tab:table119} when we trade according ``wait one day'' period. For each table, Section 1 lists the Return on Committed Capital and Section 2 on Fully Invested Capital. Panel A provides results after transaction costs and Panel B before transaction costs.
	
	The momentum signal have the best ability to predict the excess returns performance. We find that the excess returns are driven by the momentum factor, ..., factor related to liquidity,...The other traditional equity risk premia uncover very low correlation or the correlation of the excess returns with the other traditional factors is nearly zero. While the...factor is negatively correlated to the excess returns, we find no evidence of their correlation to excess returns. We also find in addition that in addition to all strategies performing better during periods of significant volatility that the ... method is the superior strategy during turbulent market conditions.
	
	A large positive value implues that the factor has a positive effect on the performance. 
	
	
	
	By analyzing Tables \ref{tab:table108} to \ref{tab:table111}, we conclude that the copula approach usually provides larger alphas than the distance strategies, typically significant at 1\\%, even after accounting all the factors, for Top 5 and Top 20 pairs without delay. Thus, the results show that the variations of the profits are not fully explained by the variation of the risk factors. Tables \ref{tab:table112} and \ref{tab:table113} present results for Top 101-120 pairs. Here the alphas for copula method are still significantly different from zero at 1\\% before costs. However, the strategy does not produce significant profits on fully invested capital after costs for both multi-factors models, differently from the distance approaches. In addition, note that we usually find a positive correlation between the strategies' average profits and the market premium when a rapid execution of the trade is made (Tables \ref{tab:table108} to \ref{tab:table113}). The market premium slopes for distance strategies are typically larger and significant at 1\\%.
	
	Analyzing the other risk factors from Tables \ref{tab:table108} to \ref{tab:table111} we can observe that the SMB coefficients are significant for copula method, at least at 10\\%; the value effect premium (HML factor) slopes are significant for \citet*{ff15}'s five research factors at 1\\% for 2.0 standard deviation threshold. We also can notice that the exposures of pairs trading return strategies to the reversal factor are positive, as suggested by \citet*{j90} and \citet*{l90fads}, and significant at 1\\% for the distance strategies and usually at 1\\% for the copula-based pairs strategy for all regression models from Tables \ref{tab:table108} to \ref{tab:table119}.  In addition, we find that the distance strategies' returns are negatively correlated with momentum loadings and highly significant (at 1\\%) as in \citet*{ggr06} for all regressions. Furthermore, the return spread of firms that invest conservatively minus aggressively (CMA factor) is negatively correlated with 2.0 standard deviation strategy returns at least at 10\\% for regression in Tables \ref{tab:table109} to \ref{tab:table111}.
	
 From Tables \ref{tab:table112} and \ref{tab:table113} one could also observe that: the SMB factor is negatively correlated with profits generated by distance methods at 1\\% for \citet*{ff93}'s three research factors plus momentum and short-term reversal and at least at 10\\% for 2.0$\sigma$ threshold and \citet*{ff15}'s five research factors; the HML factor is positively correlated with the excess returns for 2.0$\sigma$ trigger point at 10\\% for \citet*{ff15}'s five research factors.

When we delay trades by one day (Tables \ref{tab:table114} to \ref{tab:table119}) one could observe that the intercepts for copula strategy are significant before costs, usually at least at 5\\% for the \citet*{ff15}'s five research factors, whereas the distance approaches typically produce economically larger and significant alphas after costs, at least at 10\\%. In addition, we can note that the market premium coefficients are not significant for the copula-based strategy. The SMB factor is negatively correlated with the excess returns from copula method at 10\\% for Top 5 pairs and at least at 5\\% with the profits generated by 2.0 standard deviation threshold. Moreover, the HML factor is positively associated with the returns for the 2.0 standard deviation trigger point at least at 5\\% for Top 5 and Top 20 pairs when regressing the excess returns onto the \citet*{ff15}'s five research factors. Finally, the profits from the 2.0$\sigma$ strategy is negatively correlated with the CMA factor at 5\\% for Top 5 and Top 20 pairs.

The other risk-factors do not explain significantly the excess returns of the pairs trading strategies. It should be noted that the results show that the exposures to the various sources of systematic profile risk provide a low explanation of the variations of the average excess returns for any strategy as measured by the R-square and adjusted R-squared, particularly for the copula-based pairs strategy, indicating that the method is nearly factor-neutral over the whole sample period.


		\begin{sidewaystable}
		\caption{Systematic risk of Top 5 pairs without delay: \citet*{ff93}'s three factors plus Momentum and Short-Term Reversal.}
		\begin{threeparttable}[H]
			\centering \scriptsize
			\begin{tabularx}{\textwidth}{@{\extracolsep{\fill}}lllllllllllllll@{}}
				\toprule
				\multicolumn{1}{c}{Strategy} & \multicolumn{1}{c}{Intercept} & \multicolumn{1}{c}{t-stat} & \multicolumn{1}{c}{Rm-Rf} & \multicolumn{1}{c}{t-stat} & \multicolumn{1}{c}{SMB} & \multicolumn{1}{c}{t-stat} & \multicolumn{1}{c}{HML} & \multicolumn{1}{c}{t-stat} & \multicolumn{1}{c}{Mom} & \multicolumn{1}{c}{t-stat} & \multicolumn{1}{c}{Rev} & \multicolumn{1}{c}{t-stat} & \multicolumn{1}{c}{$R^{2}$} & \multicolumn{1}{c}{$R^{2}_{adj}$} \\
				\midrule
				\multicolumn{15}{c}{\textbf{Section 1: Return on Committed Capital}} \\
				\multicolumn{15}{c}{\textit{Panel A: After Transaction Costs}} \\
				\multicolumn{1}{c}{} & \multicolumn{1}{c}{} & \multicolumn{1}{c}{} & \multicolumn{1}{c}{} & \multicolumn{1}{c}{} & \multicolumn{1}{c}{} & \multicolumn{1}{c}{} & \multicolumn{1}{c}{} &       &       &       &       &       &       &  \\
				\multicolumn{1}{c}{Distance (2.0$\sigma$)} & 0.0105 & $1.7130^{*}$ & 0.0359 & $3.6379^{***}$ & -0.0258 & -1.6003 & 0.0322 & 1.5886 & -0.0417 & $-4.1624^{***}$ & 0.0673 & $4.8592^{***}$ & 0.0366 & 0.0359 \\
				\multicolumn{1}{c}{Distance (0.75$\sigma$)} & 0.0137 & $2.0385^{**}$ & 0.0400 & $3.7355^{***}$ & -0.0129 & -0.7210 & 0.0270 & 1.2780 & -0.0478 & $-3.8903^{***}$ & 0.0666 & $4.2153^{***}$ & 0.0324 & 0.0317 \\
				\multicolumn{1}{c}{Copula-GARCH} & 0.0121 & $2.5461^{**}$ & 0.0233 & $3.0429^{***}$ & -0.0201 & $-1.8464^{*}$ & 0.0179 & 1.2359 & 0.0020 & 0.2223 & 0.0189 & $1.7996^{*}$ & 0.0089 & 0.0081  \\
				&       &       &       &       &       &       &       &       &       &       &       &       &       &  \\
				\multicolumn{15}{c}{\textit{Panel B: Before Transaction Costs}} \\
				&       &       &       &       &       &       &       &       &       &       &       &       &       &  \\
				\multicolumn{1}{c}{Distance (2.0$\sigma$)}& 0.0116 & $1.8925^{*}$ & 0.0359 & $3.6313^{***}$ & -0.0262 & -1.6310 & 0.0325 & 1.6044 & -0.0418 & $-4.1697^{***}$ & 0.0677 & $4.8956^{***}$ & 0.0369 & 0.0361 \\
				\multicolumn{1}{c}{Distance (0.75$\sigma$)} & 0.0159 & $2.3594^{**}$ & 0.0399 & $3.7301^{***}$ & -0.0133 & -0.7431 & 0.0276 & 1.3020 & -0.0478 & $-3.8942^{***}$ & 0.0670 & $4.2476^{***}$ & 0.0325 & 0.0318 \\
				\multicolumn{1}{c}{Copula-GARCH} & 0.0226 & $4.7154^{***}$ & 0.0235 & $3.0533^{***}$ & -0.0202 & $-1.8366^{*}$ & 0.0186 & 1.2779 & 0.0024 & 0.2600 & 0.0196 & $1.8493^{*}$ & 0.0089 & 0.0080  \\
				&       &       &       &       &       &       &       &       &       &       &       &       &       &  \\
				\midrule
				\multicolumn{15}{c}{\textbf{Section 2: Return on Fully Invested Capital}} \\
				\multicolumn{15}{c}{\textit{Panel A: After Transaction Costs}} \\
				&       &       &       &       &       &       &       &       &       &       &       &       &       &  \\
				\multicolumn{1}{c}{Distance (2.0$\sigma$)} & 0.0246 & $2.2840^{**}$ & 0.0567 & $3.6663^{***}$ & -0.0468 & $-1.9138^{*}$ & 0.0529 & 1.5816 & -0.0744 & $-4.2842^{***}$ & 0.0946 & $4.6658^{***}$ & 0.0277 & 0.0269\\
				\multicolumn{1}{c}{Distance (0.75$\sigma$)} & 0.0141 & 1.4741 & 0.0431 & $3.1493^{***}$ & -0.0164 & -0.7187 & 0.0323 & 1.1664 & -0.0588 & $-3.6745^{***}$ & 0.0823 & $4.4385^{***}$ & 0.0238 & 0.0230 \\
				\multicolumn{1}{c}{Copula-GARCH} & 0.0328 & $2.5625^{**}$ & 0.0426 & $2.5243^{**}$ & -0.0358 & -1.2551 & 0.0508 & 1.3539 & 0.0130 & 0.5355 & 0.0559 & $2.0049^{**}$ & 0.0059 & 0.0051  \\
				&       &       &       &       &       &       &       &       &       &       &       &       &       &  \\
				\multicolumn{15}{c}{\textit{Panel B: Before Transaction Costs}} \\
				&       &       &       &       &       &       &       &       &       &       &       &       &       &  \\
				\multicolumn{1}{c}{Distance (2.0$\sigma$)} & 0.0265 & $2.4462^{**}$ & 0.0566 & $3.6615^{***}$ & -0.0472 & $-1.9286^{*}$ & 0.0535 & 1.5950 & -0.0744 & $-4.2836^{***}$ & 0.0952 & $4.6979^{***}$ & 0.0278 & 0.0270 \\
				\multicolumn{1}{c}{Distance (0.75$\sigma$)} & 0.0170 & $1.7780^{*}$ & 0.0431 & $3.1453^{***}$ & -0.0166 & -0.7304 & 0.0329 & 1.1867 & -0.0589 & $-3.6760^{***}$ & 0.0828 & $4.4624^{***}$ & 0.0238 & 0.0230 \\
				\multicolumn{1}{c}{Copula-GARCH} & 0.0602 & $4.6260^{***}$ & 0.0434 & $2.5406^{**}$ & -0.0361 & -1.2512 & 0.0529 & 1.3953 & 0.0144 & 0.5881 & 0.0571 & $2.0295^{**}$ & 0.0059 & 0.0051\\
				\bottomrule
			\end{tabularx}
			\begin{tablenotes}
				\item \textit{Note:} \scriptsize  This table shows results of regressing daily portfolio return series onto \citet*{ff93}'s three research factors plus momentum and short-term reversal over July 1991 and December 2015 (6173 observations) for pairs traded in the same day the pair diverges. Section 1 shows the Return on Committed Capital and Section 2 on Fully Invested Capital. Panel A lists the results after transaction costs and Panel B before transaction costs. The t-statistics are computed using Newey-West standard errors with six lags.
				\item \scriptsize $^{\ast\ast\ast}$, $^{\ast\ast}$, $^{\ast}$  significant at 1\\%, 5\\% and 10\\% levels, respectively.
			\end{tablenotes}
		\end{threeparttable}\%
		\label{tab:table108}\%
	\end{sidewaystable}\%
	
	
	\begin{sidewaystable}
		\caption{Systematic risk of Top 5 pairs without delay: \citet*{ff15}'s five factors.}
		\begin{threeparttable}[H]
			\centering \scriptsize
			\begin{tabularx}{\textwidth}{@{\extracolsep{\fill}}lllllllllllllll@{}}
				\toprule
				\multicolumn{1}{c}{Strategy} & \multicolumn{1}{c}{Intercept} & \multicolumn{1}{c}{t-stat} & \multicolumn{1}{c}{Rm-Rf} & \multicolumn{1}{c}{t-stat} & \multicolumn{1}{c}{SMB} & \multicolumn{1}{c}{t-stat} & \multicolumn{1}{c}{HML} & \multicolumn{1}{c}{t-stat} & \multicolumn{1}{c}{RMW} & \multicolumn{1}{c}{t-stat} & \multicolumn{1}{c}{CMA} & \multicolumn{1}{c}{t-stat} & \multicolumn{1}{c}{$R^{2}$} & \multicolumn{1}{c}{$R^{2}_{adj}$} \\
				\midrule
				\multicolumn{15}{c}{\textbf{Section 1: Return on Committed Capital}} \\
				\multicolumn{15}{c}{\textit{Panel A: After Transaction Costs}} \\
				\multicolumn{1}{c}{} & \multicolumn{1}{c}{} & \multicolumn{1}{c}{} & \multicolumn{1}{c}{} & \multicolumn{1}{c}{} & \multicolumn{1}{c}{} & \multicolumn{1}{c}{} & \multicolumn{1}{c}{} &       &       &       &       &       &       &  \\
				\multicolumn{1}{c}{Distance (2.0$\sigma$)} & 0.0148 & $2.4664^{**}$ & 0.0539 & $4.9199^{***}$ & -0.0276 & -1.4376 & 0.0653 & $2.7082^{***}$ & -0.0071 & -0.2653 & -0.0634 & $-1.8191^{*}$ & 0.0217 & 0.0209 \\
				\multicolumn{1}{c}{Distance (0.75$\sigma$)} & 0.0175 & $2.6401^{***}$ & 0.0613 & $5.0021^{***}$ & -0.0141 & -0.6882 & 0.0570 & $2.0740^{**}$ & -0.0021 & -0.0788 & -0.0472 & -1.2851 & 0.0188 & 0.0180 \\
				\multicolumn{1}{c}{Copula-GARCH} & 0.0142 & $3.0649^{***}$ & 0.0236 & $3.1722^{***}$ & -0.0229 & $-1.7680^{*}$ & 0.0223 & 1.2661 & -0.0147 & -0.7745 & -0.0201 & -0.7945 & 0.0079 & 0.0071 \\
				&       &       &       &       &       &       &       &       &       &       &       &       &       &  \\
				\multicolumn{15}{c}{\textit{Panel B: Before Transaction Costs}} \\
				&       &       &       &       &       &       &       &       &       &       &       &       &       &  \\
				\multicolumn{1}{c}{Distance (2.0$\sigma$)} & 0.0160 & $2.6517^{***}$ & 0.0540 & $4.9290^{***}$ & -0.0281 & -1.4637 & 0.0655 & $2.7185^{***}$ & -0.0072 & -0.2707 & -0.0631 & $-1.8128^{*}$ & 0.0218 & 0.0210 \\
				\multicolumn{1}{c}{Distance (0.75$\sigma$)} & 0.0198 & $2.9677^{***}$ & 0.0614 & $5.0108^{***}$ & -0.0146 & -0.7107 & 0.0573 & $2.0845^{**}$ & -0.0024 & -0.0890 & -0.0467 & -1.2706 & 0.0188 & 0.0180 \\
				\multicolumn{1}{c}{Copula-GARCH} & 0.0248 & $5.2925^{***}$ & 0.0240 & $3.1965^{***}$ & -0.0230 & $-1.7602^{*}$ & 0.0227 & 1.2737 & -0.0148 & -0.7719 & -0.0197 & -0.7725 & 0.0078 & 0.0070 \\
				&       &       &       &       &       &       &       &       &       &       &       &       &       &  \\
				\midrule
				\multicolumn{15}{c}{\textbf{Section 2: Return on Fully Invested Capital}} \\
				\multicolumn{15}{c}{\textit{Panel A: After Transaction Costs}} \\
				&       &       &       &       &       &       &       &       &       &       &       &       &       &  \\
				\multicolumn{1}{c}{Distance (2.0$\sigma$)} & 0.0297 & $2.7715^{***}$ & 0.0879 & $5.4660^{***}$ & -0.0443 & -1.5056 & 0.1066 & $3.1742^{***}$ & 0.0140 & 0.2914 & -0.0927 & $-1.8621^{*}$ & 0.0168 & 0.0160 \\
				\multicolumn{1}{c}{Distance (0.75$\sigma$)} & 0.0180 & $1.9072^{*}$ & 0.0742 & $5.0434^{***}$ & -0.0120 & -0.4638 & 0.0614 & $1.8534^{*}$ & 0.0220 & 0.6128 & -0.0413 & -0.9301 & 0.0127 & 0.0119 \\
				\multicolumn{1}{c}{Copula-GARCH} & 0.0386 & $3.0671^{***}$ & 0.0477 & $2.7034^{***}$ & -0.0398 & -1.2600 & 0.0473 & 1.0261 & -0.0263 & -0.5657 & -0.0257 & -0.4672 & 0.0041 & 0.0033 \\
				&       &       &       &       &       &       &       &       &       &       &       &       &       &  \\
				\multicolumn{15}{c}{\textit{Panel B: Before Transaction Costs}} \\
				&       &       &       &       &       &       &       &       &       &       &       &       &       &  \\
				\multicolumn{1}{c}{Distance (2.0$\sigma$)} & 0.0316 & $2.9380^{***}$ & 0.0880 & $5.4738^{***}$ & -0.0446 & -1.5174 & 0.1070 & $3.1842^{***}$ & 0.0139 & 0.2890 & -0.0926 & $-1.8594^{*}$ & 0.0168 & 0.0160 \\
				\multicolumn{1}{c}{Distance (0.75$\sigma$)} & 0.0210 & $2.2155^{**}$ & 0.0744 & $5.0509^{***}$ & -0.0124 & -0.4771 & 0.0618 & $1.8626^{*}$ & 0.0217 & 0.6033 & -0.0407 & -0.9159 & 0.0127 & 0.0119 \\
				\multicolumn{1}{c}{Copula-GARCH} & 0.0661 & $5.1680^{***}$ & 0.0486 & $2.7251^{***}$ & -0.0400 & -1.2473 & 0.0481 & 1.0336 & -0.0259 & -0.5493 & -0.0245 & -0.4407 & 0.0041 & 0.0033 \\
				\bottomrule
			\end{tabularx}\%
			\begin{tablenotes}
				\item \textit{Note:} \scriptsize  This table shows results of regressing daily return series onto \citet*{ff15}'s five research factors over July 1991 and December 2015 (6173 observations) for pairs traded and closed in the same day the pair diverges. Section 1 shows the Return on Committed Capital and Section 2 on Fully Invested Capital. Panel A lists the results after transaction costs and Panel B before transaction costs. The t-statistics are computed using Newey-West standard errors with six lags.
				\item \scriptsize $^{\ast\ast\ast}$, $^{\ast\ast}$, $^{\ast}$  significant at 1\\%, 5\\% and 10\\% levels, respectively.
			\end{tablenotes}
		\end{threeparttable}\%
		\label{tab:table109}\%
	\end{sidewaystable}\%
	
	\begin{sidewaystable}
		\caption{Systematic risk of Top 20 pairs without delay: \citet*{ff93}'s three factors plus Momentum and Short-Term Reversal.}
		\begin{threeparttable}[H]
			\centering \scriptsize
			\begin{tabularx}{\textwidth}{@{\extracolsep{\fill}}lllllllllllllll@{}}
				\toprule
				\multicolumn{1}{c}{Strategy} & \multicolumn{1}{c}{Intercept} & \multicolumn{1}{c}{t-stat} & \multicolumn{1}{c}{Rm-Rf} & \multicolumn{1}{c}{t-stat} & \multicolumn{1}{c}{SMB} & \multicolumn{1}{c}{t-stat} & \multicolumn{1}{c}{HML} & \multicolumn{1}{c}{t-stat} & \multicolumn{1}{c}{Mom} & \multicolumn{1}{c}{t-stat} & \multicolumn{1}{c}{Rev} & \multicolumn{1}{c}{t-stat} & \multicolumn{1}{c}{$R^{2}$} & \multicolumn{1}{c}{$R^{2}_{adj}$} \\
				\midrule
				\multicolumn{15}{c}{\textbf{Section 1: Return on Committed Capital}} \\
				\multicolumn{15}{c}{\textit{Panel A: After Transaction Costs}} \\
				\multicolumn{1}{c}{} & \multicolumn{1}{c}{} & \multicolumn{1}{c}{} & \multicolumn{1}{c}{} & \multicolumn{1}{c}{} & \multicolumn{1}{c}{} & \multicolumn{1}{c}{} & \multicolumn{1}{c}{} &       &       &       &       &       &       &  \\
				\multicolumn{1}{c}{Distance (2.0$\sigma$)} & 0.0082 & $2.1394^{**}$ & 0.0148 & $2.7986^{***}$ & -0.0060 & -0.5732 & 0.0119 & 0.8865 & -0.0248 & $-3.2072^{***}$ & 0.0418 & $4.6814^{***}$ & 0.0296 & 0.0288\\
				\multicolumn{1}{c}{Distance (0.75$\sigma$)} & 0.0129 & $3.0451^{***}$ & 0.0145 & $2.3771^{**}$ & -0.0062 & -0.5300 & 0.0149 & 1.0594 & -0.0271 & $-3.0683^{***}$ & 0.0404 & $4.1616^{***}$ & 0.0244 & 0.0237 \\
				\multicolumn{1}{c}{Copula-GARCH} & 0.0128 & $4.4100^{***}$ & 0.0077 & $2.0105^{**}$ & -0.0126 & $-2.1659^{**}$ & 0.0025 & 0.3048 & 0.0037 & 0.8750 & 0.0167 & $3.1007^{***}$ & 0.0080 & 0.0072 \\
				&       &       &       &       &       &       &       &       &       &       &       &       &       &  \\
				\multicolumn{15}{c}{\textit{Panel B: Before Transaction Costs}} \\
				&       &       &       &       &       &       &       &       &       &       &       &       &       &  \\
				\multicolumn{1}{c}{Distance (2.0$\sigma$)} & 0.0093 & $2.4071^{**}$ & 0.0148 & $2.7897^{***}$ & -0.0064 & -0.6122 & 0.0123 & 0.9160 & -0.0248 & $-3.2115^{***}$ & 0.0423 & $4.7311^{***}$ & 0.0300 & 0.0292 \\
				\multicolumn{1}{c}{Distance (0.75$\sigma$)} & 0.0150 & $3.5326^{***}$ & 0.0145 & $2.3679^{**}$ & -0.0066 & -0.5674 & 0.0155 & 1.0952 & -0.0272 & $-3.0789^{***}$ & 0.0409 & $4.2143^{***}$ & 0.0248 & 0.0240 \\
				\multicolumn{1}{c}{Copula-GARCH} & 0.0227 & $7.7275^{***}$ & 0.0080 & $2.0608^{**}$ & -0.0127 & $-2.1669^{**}$ & 0.0031 & 0.3795 & 0.0040 & 0.9299 & 0.0173 & $3.1689^{***}$ & 0.0082 & 0.0074  \\
				&       &       &       &       &       &       &       &       &       &       &       &       &       &  \\
				\midrule
				\multicolumn{15}{c}{\textbf{Section 2: Return on Fully Invested Capital}} \\
				\multicolumn{15}{c}{\textit{Panel A: After Transaction Costs}} \\
				&       &       &       &       &       &       &       &       &       &       &       &       &       &  \\
				\multicolumn{1}{c}{Distance (2.0$\sigma$)} & 0.0166 & $2.2657^{**}$ & 0.0311 & $3.2074^{***}$ & -0.0076 & -0.3700 & 0.0475 & $2.0306^{**}$ & -0.0480 & $-2.9748^{***}$ & 0.0734 & $4.8436^{***}$ & 0.0320 & 0.0312 \\
				\multicolumn{1}{c}{Distance (0.75$\sigma$)} & 0.0146 & $2.4635^{**}$ & 0.0206 & $2.3393^{**}$ & -0.0142 & -0.7714 & 0.0365 & $1.8910^{*}$ & -0.0393 & $-3.0043^{***}$ & 0.0643 & $4.5762^{***}$ & 0.0305 & 0.0297 \\
				\multicolumn{1}{c}{Copula-GARCH} & 0.0301 & $3.2494^{***}$ & 0.0200 & $1.7209^{*}$ & -0.0652 & $-2.0722^{**}$ & 0.0062 & 0.2224 & 0.0190 & 1.1606 & 0.0465 & $2.6897^{***}$ & 0.0071 & 0.0063 \\
				&       &       &       &       &       &       &       &       &       &       &       &       &       &  \\
				\multicolumn{15}{c}{\textit{Panel B: Before Transaction Costs}} \\
				&       &       &       &       &       &       &       &       &       &       &       &       &       &  \\
				\multicolumn{1}{c}{Distance (2.0$\sigma$)} & 0.0183 & $2.5023^{**}$ & 0.0311 & $3.2084^{***}$ & -0.0080 & -0.3934 & 0.0481 & $2.0540^{**}$ & -0.0481 & $-2.9779^{***}$ & 0.0741 & $4.8838^{***}$ & 0.0323 & 0.0315  \\
				\multicolumn{1}{c}{Distance (0.75$\sigma$)} & 0.0174 & $2.9392^{***}$ & 0.0206 & $2.3362^{**}$ & -0.0148 & -0.8009 & 0.0372 & 1.9219 & -0.0394 & $-3.0161^{***}$ & 0.0649 & $4.6167^{***}$ & 0.0308 & 0.0300 \\
				\multicolumn{1}{c}{Copula-GARCH} & 0.0621 & $6.6118^{***}$ & 0.0210 & $1.7923^{*}$ & -0.0652 & $-2.0670^{**}$ & 0.0078 & 0.2777 & 0.0200 & 1.2076 & 0.0474 & $2.7118^{***}$ & 0.0072 & 0.0064 \\
				\bottomrule
			\end{tabularx}\%
			\begin{tablenotes}
				\item \textit{Note:} \scriptsize  This table shows results of regressing daily portfolio return series onto \citet*{ff93}'s three research factors plus momentum and short-term reversal from July 1991 to December 2015 (6173 observations) for pairs traded in the same day the pair diverges. Section 1 shows the Return on Committed Capital and Section 2 on Fully Invested Capital. Panel A lists the results after transaction costs and Panel B before transaction costs. The t-statistics are computed using Newey-West standard errors with six lags.
				\item \scriptsize $^{\ast}$. $^{\ast\ast}$. $^{\ast\ast\ast}$  significant at 1\\%. 5\\% and 10\\% levels, respectively.
			\end{tablenotes}
		\end{threeparttable}\%
		\label{tab:table110}\%
	\end{sidewaystable}\%
	
	
	\begin{sidewaystable}
		\caption{Systematic risk of Top 20 pairs without delay: \citet*{ff15}'s five factors.}
		\begin{threeparttable}[H]
			\centering \scriptsize
			\begin{tabularx}{\textwidth}{@{\extracolsep{\fill}}lllllllllllllll@{}}
				\toprule
				\multicolumn{1}{c}{strategy} & \multicolumn{1}{c}{Intercept} & \multicolumn{1}{c}{t-stat} & \multicolumn{1}{c}{Rm-Rf} & \multicolumn{1}{c}{t-stat} & \multicolumn{1}{c}{SMB} & \multicolumn{1}{c}{t-stat} & \multicolumn{1}{c}{HML} & \multicolumn{1}{c}{t-stat} & \multicolumn{1}{c}{RMW} & \multicolumn{1}{c}{t-stat} & \multicolumn{1}{c}{CMA} & \multicolumn{1}{c}{t-stat} & \multicolumn{1}{c}{$R^{2}$} & \multicolumn{1}{c}{$R^{2}_{adj}$} \\
				\midrule
				\multicolumn{15}{c}{\textbf{Section 1: Return on Committed Capital}} \\
				\multicolumn{15}{c}{\textit{Panel A: After Transaction Costs}} \\
				\multicolumn{1}{c}{Strategy} & \multicolumn{1}{c}{} & \multicolumn{1}{c}{} & \multicolumn{1}{c}{} & \multicolumn{1}{c}{} & \multicolumn{1}{c}{} & \multicolumn{1}{c}{} & \multicolumn{1}{c}{} &       &       &       &       &       &       &  \\
				\multicolumn{1}{c}{Distance (2.0$\sigma$)} & 0.0109 & $2.8204^{***}$ & 0.0266 & $4.7757^{***}$ & -0.0054 & -0.4478 & 0.0313 & $2.8092^{***}$ & 0.0022 & 0.1305 & -0.0390 & $-1.9798^{**}$ & 0.0144 & 0.0136 \\
				\multicolumn{1}{c}{Distance (0.75$\sigma$)} & 0.0149 & $3.5317^{***}$ & 0.0292 & $4.5334^{***}$ & -0.0020 & -0.1514 & 0.0316 & $2.4302^{**}$ & 0.0178 & 1.0036 & -0.0307 & -1.4602 & 0.0117 & 0.0109  \\
				\multicolumn{1}{c}{Copula-GARCH} & 0.0142 & $5.0810^{***}$ & 0.0114 & $2.8961^{***}$ & -0.0117 & $-1.7413^{*}$ & -0.0028 & -0.3446 & 0.0014 & 0.1144 & 0.0027 & 0.2026 & 0.0043 & 0.0035 \\
				&       &       &       &       &       &       &       &       &       &       &       &       &       &  \\
				\multicolumn{15}{c}{\textit{Panel B: Before Transaction Costs}} \\
				&       &       &       &       &       &       &       &       &       &       &       &       &       &  \\
				\multicolumn{1}{c}{Distance (2.0$\sigma$)} & 0.0120 & $3.0949^{***}$ & 0.0267 & $4.7980^{***}$ & -0.0058 & -0.4817 & 0.0316 & $2.8333^{***}$ & 0.0022 & 0.1272 & -0.0388 & $-1.9669^{**}$ & 0.0144 & 0.0137 \\
				\multicolumn{1}{c}{Distance (0.75$\sigma$)} & 0.0170 & $4.0260^{***}$ & 0.0294 & $4.5528^{***}$ & -0.0024 & -0.1840 & 0.0320 & $2.4574^{**}$ & 0.0178 & 1.0005 & -0.0304 & -1.4448 & 0.0118 & 0.0110 \\
				\multicolumn{1}{c}{Copula-GARCH} &  0.0241 & $8.5316^{***}$ & 0.0119 & $2.9749^{***}$ & -0.0118 & $-1.7349^{*}$ & -0.0026 & -0.3168 & 0.0015 & 0.1300 & 0.0033 & 0.2413 & 0.0044 & 0.0036 \\
				&       &       &       &       &       &       &       &       &       &       &       &       &       &  \\
				\midrule
				\multicolumn{15}{c}{\textbf{Section 2: Return on Fully Invested Capital}} \\
				\multicolumn{15}{c}{\textit{Panel A: After Transaction Costs}} \\
				&       &       &       &       &       &       &       &       &       &       &       &       &       &  \\
				\multicolumn{1}{c}{Distance (2.0$\sigma$)} & 0.0212 & $2.9873^{***}$ & 0.0518 & $5.0304^{***}$ & -0.0089 & -0.3850 & 0.0842 & $3.6304^{***}$ & -0.0044 & -0.1454 & -0.0682 & $-1.9603^{**}$ & 0.0173 & 0.0165  \\
				\multicolumn{1}{c}{Distance (0.75$\sigma$)} & 0.0180 & $3.1047^{***}$ & 0.0426 & $4.4030^{***}$ & -0.0094 & -0.4522 & 0.0605 & $2.8994^{***}$ & 0.0203 & 0.8022 & -0.0454 & -1.5242 & 0.0145 & 0.0137 \\
				\multicolumn{1}{c}{Copula-GARCH} & 0.0347 & $3.8296^{***}$ & 0.0268 & $2.2449^{**}$ & -0.0697 & $-1.8779^{*}$ & -0.0174 & -0.5456 & -0.0252 & -0.6931 & 0.0261 & 0.5721 & 0.0047 & 0.0039 \\
				&       &       &       &       &       &       &       &       &       &       &       &       &       &  \\
				\multicolumn{15}{c}{\textit{Panel B: Before Transaction Costs}} \\
				&       &       &       &       &       &       &       &       &       &       &       &       &       &  \\
				\multicolumn{1}{c}{Distance (2.0$\sigma$)} & 0.0230 & $3.2356^{***}$ & 0.0520 & $5.0534^{***}$ & -0.0094 & -0.4058 & 0.0846 & $3.6482^{***}$ & -0.0044 & -0.1462 & -0.0678 & $-1.9494^{*}$ & 0.0174 & 0.0166 \\
				\multicolumn{1}{c}{Distance (0.75$\sigma$)} & 0.0209 & $3.5917^{***}$ & 0.0428 & $4.4207^{***}$ & -0.0099 & -0.4790 & 0.0611 & $2.9203^{***}$ & 0.0203 & 0.7977 & -0.0450 & -1.5098 & 0.0146 & 0.0138  \\
				\multicolumn{1}{c}{Copula-GARCH} & 0.0668 & $7.2601^{***}$ & 0.0280 & $2.3198^{**}$ & -0.0697 & $-1.8723^{*}$ & -0.0169 & -0.5252 & -0.0253 & -0.6898 & 0.0275 & 0.5974 & 0.0047 & 0.0039 \\
				\bottomrule
			\end{tabularx}\%
			\begin{tablenotes}
				\item \textit{Note:} \scriptsize  This table shows results of regressing daily return series onto \citet*{ff15}'s five research factors from July 1991 to December 2015 (6173 observations) for pairs traded and closed in the same day the pair diverges. Section 1 shows the Return on Committed Capital and Section 2 on Fully Invested Capital. Panel A lists the results after transaction costs and Panel B before transaction costs. The t-statistics are computed using Newey-West standard errors with six lags.
				\item \scriptsize $^{\ast}$. $^{\ast\ast}$. $^{\ast\ast\ast}$  significant at 1\\%. 5\\% and 10\\% levels, respectively.
			\end{tablenotes}
		\end{threeparttable}\%
		\label{tab:table111}\%
	\end{sidewaystable}\%
	
	%\% Table generated by Excel2LaTeX from sheet 'd0p101_120ff3MR'
	\begin{sidewaystable}
		\caption{Systematic risk of Top 101-120 pairs without delay: \citet*{ff93}'s three factors plus Momentum and Short-Term Reversal.}
		\begin{threeparttable}[H]
			\centering \scriptsize
			\begin{tabularx}{\textwidth}{@{\extracolsep{\fill}}lllllllllllllll@{}}
				\toprule
				\multicolumn{1}{c}{Strategy} & \multicolumn{1}{c}{Intercept} & \multicolumn{1}{c}{t-stat} & \multicolumn{1}{c}{Rm-Rf} & \multicolumn{1}{c}{t-stat} & \multicolumn{1}{c}{SMB} & \multicolumn{1}{c}{t-stat} & \multicolumn{1}{c}{HML} & \multicolumn{1}{c}{t-stat} & \multicolumn{1}{c}{Mom} & \multicolumn{1}{c}{t-stat} & \multicolumn{1}{c}{Rev} & \multicolumn{1}{c}{t-stat} & \multicolumn{1}{c}{$R^{2}$} & \multicolumn{1}{c}{$R^{2}_{adj}$} \\
				\midrule
				\multicolumn{15}{c}{\textbf{Section 1: Return on Committed Capital}} \\
				\multicolumn{15}{c}{\textit{Panel A: After Transaction Costs}} \\
				\multicolumn{1}{c}{} & \multicolumn{1}{c}{} & \multicolumn{1}{c}{} & \multicolumn{1}{c}{} & \multicolumn{1}{c}{} & \multicolumn{1}{c}{} & \multicolumn{1}{c}{} & \multicolumn{1}{c}{} &       &       &       &       &       &       &  \\
				\multicolumn{1}{c}{Distance (2.0$\sigma$)} & 0.0171 & $3.9884^{***}$ & 0.0217 & $3.7659^{***}$ & -0.0341 & $-3.0828^{***}$ & -0.0098 & -0.8122 & -0.0542 & $-7.3980^{***}$ & 0.0502 & $5.5927^{***}$ & 0.0565 & 0.0557 \\
				\multicolumn{1}{c}{Distance (0.75$\sigma$)} & 0.0156 & $3.3434^{***}$ & 0.0193 & $3.0008^{***}$ & -0.0383 & $-3.0801^{***}$ & -0.0101 & -0.7101 & -0.0573 & $-6.3982^{***}$ & 0.0608 & $5.6721^{***}$ & 0.0541 & 0.0533  \\
				\multicolumn{1}{c}{Copula-GARCH} & 0.0064 & $1.9814^{**}$ & 0.0025 & 0.5443 & -0.0042 & -0.4916 & 0.0029 & 0.3208 & 0.0064 & 1.1719 & 0.0265 & $3.1875^{***}$ & 0.0083 & 0.0075  \\
				&       &       &       &       &       &       &       &       &       &       &       &       &       &  \\
				\multicolumn{15}{c}{\textit{Panel B: Before Transaction Costs}} \\
				&       &       &       &       &       &       &       &       &       &       &       &       &       &  \\
				\multicolumn{1}{c}{Distance (2.0$\sigma$)} & 0.0181 & $4.2062^{***}$ & 0.0216 & $3.7530^{***}$ & -0.0344 & $-3.1041^{***}$ & -0.0095 & -0.7839 & -0.0543 & $-7.4069^{***}$ & 0.0506 & $5.6314^{***}$ & 0.0567 & 0.0559 \\
				\multicolumn{1}{c}{Distance (0.75$\sigma$)} & 0.0175 & $3.7285^{***}$ & 0.0192 & $2.9900^{***}$ & -0.0386 & $-3.1016^{***}$ & -0.0097 & -0.6804 & -0.0574 & $-6.4019^{***}$ & 0.0613 & $5.7114^{***}$ & 0.0543 & 0.0535  \\
				\multicolumn{1}{c}{Copula-GARCH} & 0.0150 & $4.6097^{***}$ & 0.0026 & 0.5654 & -0.0042 & -0.4855 & 0.0034 & 0.3752 & 0.0064 & 1.1699 & 0.0271 & $3.2230^{***}$ & 0.0085 & 0.0077 \\
				&       &       &       &       &       &       &       &       &       &       &       &       &       &  \\
				\midrule
				\multicolumn{15}{c}{\textbf{Section 2: Return on Fully Invested Capital}} \\
				\multicolumn{15}{c}{\textit{Panel A: After Transaction Costs}} \\
				&       &       &       &       &       &       &       &       &       &       &       &       &       &  \\
				\multicolumn{1}{c}{Distance (2.0$\sigma$)} & 0.0291 & $3.4766^{***}$ & 0.0443 & $3.7270^{***}$ & -0.0536 & $-2.5058^{**}$ & 0.0114 & 0.4654 & -0.1219 & $-8.5177^{***}$ & 0.1082 & $6.5385^{***}$ & 0.0675 & 0.0667 \\
				\multicolumn{1}{c}{Distance (0.75$\sigma$)} & 0.0180 & $2.6072^{***}$ & 0.0288 & $2.7632^{***}$ & -0.0447 & $-2.3331^{**}$ & -0.0023 & -0.0997 & -0.0919 & $-6.6865^{***}$ & 0.0982 & $6.1567^{***}$ & 0.0629 & 0.0622  \\
				\multicolumn{1}{c}{Copula-GARCH} & 0.0023 & 0.2133 & 0.0012 & 0.0767 & 0.0022 & 0.0719 & 0.0072 & 0.2272 & 0.0180 & 0.8581 & 0.0790 & $2.5163^{**}$ & 0.0060 & 0.0052 \\
				&       &       &       &       &       &       &       &       &       &       &       &       &       &  \\
				\multicolumn{15}{c}{\textit{Panel B: Before Transaction Costs}} \\
				&       &       &       &       &       &       &       &       &       &       &       &       &       &  \\
				\multicolumn{1}{c}{Distance (2.0$\sigma$)} & 0.0308 & $3.6711^{***}$ & 0.0443 & $3.7255^{***}$ & -0.0540 & $-2.5244^{**}$ & 0.0119 & 0.4868 & -0.1221 & $-8.5271^{***}$ & 0.1088 & $6.5719^{***}$ & 0.0677 & 0.0669 \\
				\multicolumn{1}{c}{Distance (0.75$\sigma$)} & 0.0207 & $2.9783^{***}$ & 0.0288 & $2.7643^{***}$ & -0.0452 & $-2.3516^{**}$ & -0.0018 & -0.0783 & -0.0921 & $-6.6893^{***}$ & 0.0988 & $6.1848^{***}$ & 0.0631 & 0.0623  \\
				\multicolumn{1}{c}{Copula-GARCH} & 0.0317 & $2.9139^{***}$ & 0.0018 & 0.1127 & 0.0029 & 0.0933 & 0.0090 & 0.2818 & 0.0184 & 0.8673 & 0.0802 & $2.5317^{**}$ & 0.0060 & 0.0052 \\
				\bottomrule
			\end{tabularx}\%
			\begin{tablenotes}
				\item \textit{Note:} \scriptsize  This table shows results of regressing daily portfolio return series onto \citet*{ff93}'s three research factors plus momentum and short-term reversal from July 1991 to December 2015 (6173 observations) for pairs traded in the same day the pair diverges. Section 1 shows the Return on Committed Capital and Section 2 on Fully Invested Capital. Panel A lists the results after transaction costs and Panel B before transaction costs. The t-statistics are computed using Newey-West standard errors with six lags.
				\item \scriptsize $^{\ast}$. $^{\ast\ast}$. $^{\ast\ast\ast}$  significant at 1\\%. 5\\% and 10\\% levels, respectively.
			\end{tablenotes}
		\end{threeparttable}\%
		\label{tab:table112}\%
	\end{sidewaystable}\%
	
	%\% Table generated by Excel2LaTeX from sheet 'd0p101_120ff5'
	\begin{sidewaystable}
		\caption{Systematic risk of Top 101-120 pairs without delay: \citet*{ff15}'s five factors.}
		\begin{threeparttable}[H]
			\centering \scriptsize
			\begin{tabularx}{\textwidth}{@{\extracolsep{\fill}}lllllllllllllll@{}}
				\toprule
				\multicolumn{1}{c}{Strategy} & \multicolumn{1}{c}{Intercept} & \multicolumn{1}{c}{t-stat} & \multicolumn{1}{c}{Rm-Rf} & \multicolumn{1}{c}{t-stat} & \multicolumn{1}{c}{SMB} & \multicolumn{1}{c}{t-stat} & \multicolumn{1}{c}{HML} & \multicolumn{1}{c}{t-stat} & \multicolumn{1}{c}{RMW} & \multicolumn{1}{c}{t-stat} & \multicolumn{1}{c}{CMA} & \multicolumn{1}{c}{t-stat} & \multicolumn{1}{c}{$R^{2}$} & \multicolumn{1}{c}{$R^{2}_{adj}$} \\
				\midrule
				\multicolumn{15}{c}{\textbf{Section 1: Return on Committed Capital}} \\
				\multicolumn{15}{c}{\textit{Panel A: After Transaction Costs}} \\
				\multicolumn{1}{c}{} & \multicolumn{1}{c}{} & \multicolumn{1}{c}{} & \multicolumn{1}{c}{} & \multicolumn{1}{c}{} & \multicolumn{1}{c}{} & \multicolumn{1}{c}{} & \multicolumn{1}{c}{} &       &       &       &       &       &       &  \\
				\multicolumn{1}{c}{Distance (2.0$\sigma$)} & 0.0182 & $4.3299^{***}$ & 0.0479 & $7.3568^{***}$ & -0.0288 & $-2.4061^{**}$ & 0.0109 & 0.9393 & 0.0297 & $1.8187^{*}$ & -0.0121 & -0.6111 & 0.0261 & 0.0253 \\
				\multicolumn{1}{c}{Distance (0.75$\sigma$)} & 0.0171 & $3.7189^{***}$ & 0.0510 & $7.0725^{***}$ & -0.0295 & $-2.2498^{**}$ & 0.0079 & 0.5800 & 0.0437 & $2.3887^{**}$ & -0.0076 & -0.3676 & 0.0234 & 0.0226 \\
				\multicolumn{1}{c}{Copula-GARCH} & 0.0085 & $2.6893^{***}$ & 0.0088 & $1.8751^{*}$ & -0.0004 & -0.0388 & -0.0040 & -0.3841 & 0.0113 & 1.0017 & -0.0025 & -0.1701 & 0.0017 & 0.0009 \\
				&       &       &       &       &       &       &       &       &       &       &       &       &       &  \\
				\multicolumn{15}{c}{\textit{Panel B: Before Transaction Costs}} \\
				&       &       &       &       &       &       &       &       &       &       &       &       &       &  \\
				\multicolumn{1}{c}{Distance (2.0$\sigma$)} & 0.0192 & $4.5547^{***}$ & 0.0480 & $7.3642^{***}$ & -0.0290 & $-2.4269^{**}$ & 0.0111 & 0.9581 & 0.0297 & $1.8128^{*}$ & -0.0118 & -0.5987 & 0.0261 & 0.0253 \\
				\multicolumn{1}{c}{Distance (0.75$\sigma$)} & 0.0190 & $4.1135^{***}$ & 0.0511 & $7.0826^{***}$ & -0.0298 & $-2.2705^{**}$ & 0.0082 & 0.6003 & 0.0438 & $2.3907^{**}$ & -0.0074 & -0.3574 & 0.0234 & 0.0226 \\
				\multicolumn{1}{c}{Copula-GARCH} & 0.0171 & $5.3758^{***}$ & 0.0090 & $1.9019^{*}$ & -0.0003 & -0.0347 & -0.0035 & -0.3286 & 0.0113 & 0.9903 & -0.0029 & -0.1929 & 0.0017 & 0.0009 \\
				&       &       &       &       &       &       &       &       &       &       &       &       &       &  \\
				\midrule
				\multicolumn{15}{c}{\textbf{Section 2: Return on Fully Invested Capital}} \\
				\multicolumn{15}{c}{\textit{Panel A: After Transaction Costs}} \\
				&       &       &       &       &       &       &       &       &       &       &       &       &       &  \\
				\multicolumn{1}{c}{Distance (2.0$\sigma$)} & 0.0318 & $3.8873^{***}$ & 0.0984 & $7.3038^{***}$ & -0.0431 & $-1.8390^{*}$ & 0.0688 & $2.4445^{**}$ & 0.0590 & $1.9524^{*}$ & -0.0544 & -1.4317 & 0.0295 & 0.0287 \\
				\multicolumn{1}{c}{Distance (0.75$\sigma$)} & 0.0207 & $3.0488^{***}$ & 0.0782 & $6.7997^{***}$ & -0.0318 & -1.5287 & 0.0300 & 1.2722 & 0.0647 & $2.4664^{**}$ & -0.0209 & -0.6763 & 0.0251 & 0.0243  \\
				\multicolumn{1}{c}{Copula-GARCH} & 0.0085 & 0.7981 & 0.0206 & 1.3292 & 0.0136 & 0.4218 & -0.0143 & -0.3826 & 0.0335 & 0.9357 & -0.0035 & -0.0661 & 0.0009 & 0.0001 \\
				&       &       &       &       &       &       &       &       &       &       &       &       &       &  \\
				\multicolumn{15}{c}{\textit{Panel B: Before Transaction Costs}} \\
				&       &       &       &       &       &       &       &       &       &       &       &       &       &  \\
				\multicolumn{1}{c}{Distance (2.0$\sigma$)} & 0.0336 & $4.0887^{***}$ & 0.0985 & $7.3146^{***}$ & -0.0436 & $-1.8577^{*}$ & 0.0693 & $2.4581^{**}$ & 0.0588 & $1.9438^{*}$ & -0.0541 & -1.4213 & 0.0296 & 0.0288 \\
				\multicolumn{1}{c}{Distance (0.75$\sigma$)} & 0.0234 & $3.4291^{***}$ & 0.0784 & $6.8149^{***}$ & -0.0322 & -1.5463 & 0.0303 & 1.2871 & 0.0647 & $2.4648^{**}$ & -0.0206 & -0.6671 & 0.0251 & 0.0243 \\
				\multicolumn{1}{c}{Copula-GARCH} & 0.0380 & $3.5308^{***}$ & 0.0212 & 1.3641 & 0.0143 & 0.4406 & -0.0126 & -0.3344 & 0.0334 & 0.9227 & -0.0041 & -0.0768 & 0.0009 & 0.0000 \\
				\bottomrule
			\end{tabularx}\%
			\begin{tablenotes}
				\item \textit{Note:} \scriptsize  This table shows results of regressing daily portfolio return series onto \citet*{ff15}'s five research factors from July 1991 to December 2015 (6173 observations) for pairs traded in the same day the pair diverges. Section 1 shows the Return on Committed Capital and Section 2 on Fully Invested Capital. Panel A lists the results after transaction costs and Panel B before transaction costs. The t-statistics are computed using Newey-West standard errors with six lags.
				\item \scriptsize $^{\ast}$. $^{\ast\ast}$. $^{\ast\ast\ast}$  significant at 1\\%. 5\\% and 10\\% levels, respectively.
			\end{tablenotes}
		\end{threeparttable}\%
		\label{tab:table113}\%
	\end{sidewaystable}\%

\begin{sidewaystable}
	\caption{Systematic risk of Top 5 pairs with a one-day waiting period: \citet*{ff93}'s three factors plus Momentum and Short-Term Reversal.}
	\begin{threeparttable}[H]
		\centering \scriptsize
		\begin{tabularx}{\textwidth}{@{\extracolsep{\fill}}lllllllllllllll@{}}
			\toprule
			\multicolumn{1}{c}{Strategy} & \multicolumn{1}{c}{Intercept} & \multicolumn{1}{c}{t-stat} & \multicolumn{1}{c}{Rm-Rf} & \multicolumn{1}{c}{t-stat} & \multicolumn{1}{c}{SMB} & \multicolumn{1}{c}{t-stat} & \multicolumn{1}{c}{HML} & \multicolumn{1}{c}{t-stat} & \multicolumn{1}{c}{Mom} & \multicolumn{1}{c}{t-stat} & \multicolumn{1}{c}{Rev} & \multicolumn{1}{c}{t-stat} & \multicolumn{1}{c}{$R^{2}$} & \multicolumn{1}{c}{$R^{2}_{adj}$} \\
			\midrule
			\multicolumn{15}{c}{\textbf{Section 1: Return on Committed Capital}} \\
			\multicolumn{15}{c}{\textit{Panel A: After Transaction Costs}} \\
			\multicolumn{1}{c}{} & \multicolumn{1}{c}{} & \multicolumn{1}{c}{} & \multicolumn{1}{c}{} & \multicolumn{1}{c}{} & \multicolumn{1}{c}{} & \multicolumn{1}{c}{} & \multicolumn{1}{c}{} &       &       &       &       &       &       &  \\
			\multicolumn{1}{c}{Distance (2.0$\sigma$)} & 0.0059 & 0.9736 & 0.0136 & 1.4841 & -0.0080 & -0.5179 & 0.0220 & 1.1342 & -0.0388 & $-4.0880^{***}$ & 0.0559 & $4.5300^{***}$ & 0.0201 & 0.0193 \\
			\multicolumn{1}{c}{Distance (0.75$\sigma$)} & 0.0112 & $1.6671^{*}$ & 0.0059 & 0.5879 & 0.0072 & 0.4347 & 0.0079 & 0.3844 & -0.0465 & $-3.9023^{***}$ & 0.0540 & $3.8301^{***}$ & 0.0150 & 0.0142 \\
			\multicolumn{1}{c}{Copula-GARCH} & -0.0038 & -0.7703 & 0.0084 & 1.1051 & -0.0167 & -1.4811 & 0.0119 & 0.8491 & 0.0003 & 0.0347 & 0.0306 & $2.8326^{***}$ & 0.0066 & 0.0058 \\
			&       &       &       &       &       &       &       &       &       &       &       &       &       &  \\
			\multicolumn{15}{c}{\textit{Panel B: Before Transaction Costs}} \\
			&       &       &       &       &       &       &       &       &       &       &       &       &       &  \\
			\multicolumn{1}{c}{Distance (2.0$\sigma$)} & 0.0070 & 1.1550 & 0.0136 & 1.4787 & -0.0084 & -0.5471 & 0.0223 & 1.1516 & -0.0389 & $-4.0982^{***}$ & 0.0563 & $4.5691^{***}$ & 0.0203 & 0.0195 \\
			\multicolumn{1}{c}{Distance (0.75$\sigma$)} & 0.0135 & $1.9908^{**}$ & 0.0058 & 0.5832 & 0.0068 & 0.4137 & 0.0084 & 0.4101 & -0.0465 & $-3.9086^{***}$ & 0.0545 & $3.8647^{***}$ & 0.0151 & 0.0143 \\
			\multicolumn{1}{c}{Copula-GARCH} & 0.0068 & 1.3734 & 0.0086 & 1.1361 & -0.0168 & -1.4879 & 0.0126 & 0.8931 & 0.0007 & 0.0772 & 0.0313 & $2.8809^{***}$ & 0.0068 & 0.0060 \\
			&       &       &       &       &       &       &       &       &       &       &       &       &       &  \\
			\midrule
			\multicolumn{15}{c}{\textbf{Section 2: Return on Fully Invested Capital}} \\
			\multicolumn{15}{c}{\textit{Panel A: After Transaction Costs}} \\
			&       &       &       &       &       &       &       &       &       &       &       &       &       &  \\
			\multicolumn{1}{c}{Distance (2.0$\sigma$)} & 0.0158 & 1.4784 & 0.0242 & $1.6957^{*}$ & -0.0148 & -0.5994 & 0.0324 & 1.0041 & -0.0674 & $-3.7852^{***}$ & 0.0800 & $3.9932^{***}$ & 0.0151 & 0.0143 \\
			\multicolumn{1}{c}{Distance (0.75$\sigma$)} & 0.0178 & 1.4896 & 0.0019 & 0.1229 & 0.0009 & 0.0388 & 0.0553 & 1.1180 & -0.0592 & $-3.0177^{***}$ & 0.0586 & $2.7672^{***}$ & 0.0073 & 0.0065 \\
			\multicolumn{1}{c}{Copula-GARCH} & 0.0023 & 0.1721 & 0.0076 & 0.4490 & -0.0540 & $-1.8655^{*}$ & -0.0024 & -0.0683 & -0.0054 & -0.2364 & 0.0646 & $2.3010^{**}$ & 0.0040 & 0.0032 \\
			&       &       &       &       &       &       &       &       &       &       &       &       &       &  \\
			\multicolumn{15}{c}{\textit{Panel B: Before Transaction Costs}} \\
			&       &       &       &       &       &       &       &       &       &       &       &       &       &  \\
			\multicolumn{1}{c}{Distance (2.0$\sigma$)} & 0.0176 & 1.6436 & 0.0242 & $1.6943^{*}$ & -0.0151 & -0.6132 & 0.0329 & 1.0197 & -0.0674 & $-3.7852^{***}$ & 0.0806 & $4.0247^{***}$ & 0.0152 & 0.0144 \\
			\multicolumn{1}{c}{Distance (0.75$\sigma$)} & 0.0206 & $1.7118^{*}$ & 0.0019 & 0.1219 & 0.0007 & 0.0308 & 0.0560 & 1.1293 & -0.0591 & $-3.0086^{***}$ & 0.0590 & $2.7841^{***}$ & 0.0073 & 0.0065 \\
			\multicolumn{1}{c}{Copula-GARCH} & 0.0191 & 1.4339 & 0.0078 & 0.4618 & -0.0545 & $-1.8746^{*}$ & -0.0013 & -0.0366 & -0.0044 & -0.1929 & 0.0653 & $2.3090^{**}$ & 0.0040 & 0.0032 \\
			\bottomrule
		\end{tabularx}\%
		\begin{tablenotes}
			\item \textit{Note:} \scriptsize  This table shows results of regressing daily portfolio return series onto \citet*{ff93}'s three research factors plus momentum and short-term reversal from July 1991 to December 2015 (6173 observations) for pairs traded according to the one-day waiting period rule. Section 1 shows the return on Committed Capital and Section 2 on Fully Invested Capital. Panel A lists the results after transaction costs and Panel B before transaction costs. The t-statistics are computed using Newey-West standard errors with six lags.
			\item \scriptsize $^{\ast}$. $^{\ast\ast}$. $^{\ast\ast\ast}$  significant at 1\\%. 5\\% and 10\\% levels, respectively.
		\end{tablenotes}
	\end{threeparttable}\%
	\label{tab:table114}\%
\end{sidewaystable}\%


\begin{sidewaystable}
	\caption{Systematic risk of Top 5 pairs with a one-day waiting period: \citet*{ff15}'s five factors.}
	\begin{threeparttable}[H]
		\centering \scriptsize
		\begin{tabularx}{\textwidth}{@{\extracolsep{\fill}}lllllllllllllll@{}}
			\toprule
			\multicolumn{1}{c}{Strategy} & \multicolumn{1}{c}{Intercept} & \multicolumn{1}{c}{t-stat} & \multicolumn{1}{c}{Rm-Rf} & \multicolumn{1}{c}{t-stat} & \multicolumn{1}{c}{SMB} & \multicolumn{1}{c}{t-stat} & \multicolumn{1}{c}{HML} & \multicolumn{1}{c}{t-stat} & \multicolumn{1}{c}{RMW} & \multicolumn{1}{c}{t-stat} & \multicolumn{1}{c}{CMA} & \multicolumn{1}{c}{t-stat} & \multicolumn{1}{c}{$R^{2}$} & \multicolumn{1}{c}{$R^{2}_{adj}$} \\
			\midrule
			\multicolumn{15}{c}{\textbf{Section 1: Return on Committed Capital}} \\
			\multicolumn{15}{c}{\textit{Panel A: After Transaction Costs}} \\
			\multicolumn{1}{c}{} & \multicolumn{1}{c}{} & \multicolumn{1}{c}{} & \multicolumn{1}{c}{} & \multicolumn{1}{c}{} & \multicolumn{1}{c}{} & \multicolumn{1}{c}{} & \multicolumn{1}{c}{} &       &       &       &       &       &       &  \\
			\multicolumn{1}{c}{Distance (2.0$\sigma$)} & 0.0098 & $1.6488^{*}$ & 0.0263 & $2.7513^{***}$ & -0.0122 & -0.6606 & 0.0581 & $2.7564^{***}$ & -0.0169 & -0.6150 & -0.0691 & $-2.1641^{**}$ & 0.0096 & 0.0088 \\
			\multicolumn{1}{c}{Distance (0.75$\sigma$)} & 0.0146 & $2.1975^{**}$ & 0.0205 & $1.9370^{*}$ & 0.0037 & 0.1955 & 0.0475 & $1.9574^{*}$ & -0.0117 & -0.4316 & -0.0688 & $-2.0285^{**}$ & 0.0054 & 0.0046 \\
			\multicolumn{1}{c}{Copula-GARCH} & -0.0008 & -0.1693 & 0.0120 & 1.5803 & -0.0206 & -1.5566 & 0.0133 & 0.8493 & -0.0194 & -0.9698 & -0.0134 & -0.5776 & 0.0032 & 0.0024 \\
			&       &       &       &       &       &       &       &       &       &       &       &       &       &  \\
			\multicolumn{15}{c}{\textit{Panel B: Before Transaction Costs}} \\
			&       &       &       &       &       &       &       &       &       &       &       &       &       &  \\
			\multicolumn{1}{c}{Distance (2.0$\sigma$)} & 0.0110 & $1.8377^{*}$ & 0.0264 & $2.7610^{***}$ & -0.0127 & -0.6868 & 0.0584 & $2.7684^{***}$ & -0.0170 & -0.6200 & -0.0689 & $-2.1575^{**}$ & 0.0096 & 0.0088 \\
			\multicolumn{1}{c}{Distance (0.75$\sigma$)} & 0.0168 & $2.5295^{**}$ & 0.0206 & $1.9461^{*}$ & 0.0033 & 0.1728 & 0.0478 & $1.9699^{**}$ & -0.0120 & -0.4418 & -0.0683 & $-2.0130^{**}$ & 0.0054 & 0.0046 \\
			\multicolumn{1}{c}{Copula-GARCH} & 0.0098 & $2.0443^{**}$ & 0.0124 & 1.6260 & -0.0207 & -1.5591 & 0.0136 & 0.8616 & -0.0194 & -0.9673 & -0.0131 & -0.5593 & 0.0032 & 0.0024 \\
			&       &       &       &       &       &       &       &       &       &       &       &       &       &  \\
			\multicolumn{15}{c}{\textbf{Section 2: Return on Fully Invested Capital}} \\
			\multicolumn{15}{c}{\textit{Panel A: After Transaction Costs}} \\
			&       &       &       &       &       &       &       &       &       &       &       &       &       &  \\
			\multicolumn{1}{c}{Distance (2.0$\sigma$)} & 0.0206 & $1.9448^{*}$ & 0.0465 & $3.2011^{***}$ & -0.0162 & -0.5476 & 0.0921 & $2.9306^{***}$ & -0.0035 & -0.0752 & -0.1097 & $-2.2485^{**}$ & 0.0075 & 0.0067 \\
			\multicolumn{1}{c}{Distance (0.75$\sigma$)} & 0.0187 & $1.6963^{*}$ & 0.0339 & $2.3411^{**}$ & 0.0197 & 0.6944 & 0.0869 & $2.0753^{**}$ & 0.0811 & 1.2226 & -0.0493 & -0.8947 & 0.0038 & 0.0029 \\
			\multicolumn{1}{c}{Copula-GARCH} & 0.0083 & 0.6296 & 0.0171 & 0.9712 & -0.0635 & $-1.9433^{*}$ & 0.0004 & 0.0089 & -0.0438 & -0.8763 & -0.0191 & -0.3487 & 0.0019 & 0.0011 \\
			&       &       &       &       &       &       &       &       &       &       &       &       &       &  \\
			\multicolumn{15}{c}{\textit{Panel B: Before Transaction Costs}} \\
			&       &       &       &       &       &       &       &       &       &       &       &       &       &  \\
			\multicolumn{1}{c}{Distance (2.0$\sigma$)}& 0.0224 & $2.1154^{**}$ & 0.0466 & $3.2108^{***}$ & -0.0166 & -0.5597 & 0.0925 & $2.9427^{***}$ & -0.0037 & -0.0781 & -0.1095 & $-2.2434^{**}$ & 0.0075 & 0.0067 \\
			\multicolumn{1}{c}{Distance (0.75$\sigma$)} & 0.0215 & $1.9389^{*}$ & 0.0341 & $2.3490^{**}$ & 0.0195 & 0.6871 & 0.0871 & $2.0776^{**}$ & 0.0811 & 1.2202 & -0.0484 & -0.8782 & 0.0038 & 0.0030 \\
			\multicolumn{1}{c}{Copula-GARCH} & 0.0252 & $1.9030^{*}$ & 0.0176 & 0.9902 & -0.0637 & $-1.9359^{*}$ & 0.0006 & 0.0143 & -0.0424 & -0.8428 & -0.0184 & -0.3354 & 0.0019 & 0.0011  \\
			\bottomrule
		\end{tabularx}\%
		\begin{tablenotes}
			\item \textit{Note:} \scriptsize  This table shows results of regressing daily portfolio return series onto \citet*{ff15}'s five research factors over July 1991 and December 2015 (6173 observations) for pairs traded according to the one-day waiting period rule. Section 1 shows the Return on Committed Capital and Section 2 on Fully Invested Capital. Panel A lists the results after transaction costs and Panel B before transaction costs. The t-statistics are computed using Newey-West standard errors with six lags.
			\item \scriptsize $^{\ast}$. $^{\ast\ast}$. $^{\ast\ast\ast}$  significant at 1\\%. 5\\% and 10\\% levels, respectively.
		\end{tablenotes}
	\end{threeparttable}\%
	\label{tab:table115}\%
\end{sidewaystable}\%

\begin{sidewaystable}
	\caption{Systematic risk of Top 20 pairs with a one-day waiting period: \citet*{ff93}'s three factors plus Momentum and Short-Term Reversal.}
	\begin{threeparttable}[H]
		\centering \scriptsize
		\begin{tabularx}{\textwidth}{@{\extracolsep{\fill}}lllllllllllllll@{}}
			\toprule
			\multicolumn{1}{c}{Strategy} & \multicolumn{1}{c}{Intercept} & \multicolumn{1}{c}{t-stat} & \multicolumn{1}{c}{Rm-Rf} & \multicolumn{1}{c}{t-stat} & \multicolumn{1}{c}{SMB} & \multicolumn{1}{c}{t-stat} & \multicolumn{1}{c}{HML} & \multicolumn{1}{c}{t-stat} & \multicolumn{1}{c}{Mom} & \multicolumn{1}{c}{t-stat} & \multicolumn{1}{c}{Rev} & \multicolumn{1}{c}{t-stat} & \multicolumn{1}{c}{$R^{2}$} & \multicolumn{1}{c}{$R^{2}_{adj}$} \\
			\midrule
			\multicolumn{15}{c}{\textbf{Section 1: Return on Committed Capital}} \\
			\multicolumn{15}{c}{\textit{Panel A: After Transaction Costs}} \\
			\multicolumn{1}{c}{} & \multicolumn{1}{c}{} & \multicolumn{1}{c}{} & \multicolumn{1}{c}{} & \multicolumn{1}{c}{} & \multicolumn{1}{c}{} & \multicolumn{1}{c}{} & \multicolumn{1}{c}{} &       &       &       &       &       &       &  \\
			\multicolumn{1}{c}{Distance (2.0$\sigma$)} & 0.0057 & 1.5428 & -0.0037 & -0.7360 & 0.0021 & 0.2061 & 0.0020 & 0.1641 & -0.0231 & $-3.2594^{***}$ & 0.0322 & $4.0170^{***}$ & 0.0133 & 0.0125 \\
			\multicolumn{1}{c}{Distance (0.75$\sigma$)} & 0.0093 & $2.2875^{**}$ & -0.0119 & $-2.0263^{**}$ & 0.0082 & 0.7101 & -0.0011 & -0.0800 & -0.0283 & $-3.4252^{***}$ & 0.0319 & $3.4998^{***}$ & 0.0115 & 0.0107 \\
			\multicolumn{1}{c}{Copula-GARCH} & 0.0003 & 0.1005 & -0.0015 & -0.4167 & -0.0041 & -0.6578 & 0.0008 & 0.0981 & 0.0044 & 1.0882 & 0.0189 & $3.4078^{***}$ & 0.0051 & 0.0042 \\
			&       &       &       &       &       &       &       &       &       &       &       &       &       &  \\
			\multicolumn{15}{c}{\textit{Panel B: Before Transaction Costs}} \\
			&       &       &       &       &       &       &       &       &       &       &       &       &       &  \\
			\multicolumn{1}{c}{Distance (2.0$\sigma$)} & 0.0068 & $1.8198^{*}$ & -0.0038 & -0.7372 & 0.0017 & 0.1677 & 0.0024 & 0.1983 & -0.0232 & $-3.2639^{***}$ & 0.0327 & $4.0702^{***}$ & 0.0136 & 0.0128 \\
			\multicolumn{1}{c}{Distance (0.75$\sigma$)} & 0.0114 & $2.7915^{***}$ & -0.0120 & $-2.0240^{**}$ & 0.0078 & 0.6762 & -0.0005 & -0.0406 & -0.0284 & $-3.4366^{***}$ & 0.0324 & $3.5540^{***}$ & 0.0118 & 0.0110 \\
			\multicolumn{1}{c}{Copula-GARCH} & 0.0101 & $3.4592^{***}$ & -0.0013 & -0.3392 & -0.0043 & -0.6772 & 0.0014 & 0.1671 & 0.0048 & 1.1588 & 0.0195 & $3.4798^{***}$ & 0.0053 & 0.0045 \\
			&       &       &       &       &       &       &       &       &       &       &       &       &       &  \\
			\midrule
			\multicolumn{15}{c}{\textbf{Section 2: Return on Fully Invested Capital}} \\
			\multicolumn{15}{c}{\textit{Panel A: After Transaction Costs}} \\
			&       &       &       &       &       &       &       &       &       &       &       &       &       &  \\
			\multicolumn{1}{c}{Distance (2.0$\sigma$)} & 0.0147 & $2.0908^{**}$ & -0.0058 & -0.6469 & 0.0100 & 0.4988 & 0.0122 & 0.6036 & -0.0510 & $-3.8345^{***}$ & 0.0494 & $3.6289^{***}$ & 0.0133 & 0.0125 \\
			\multicolumn{1}{c}{Distance (0.75$\sigma$)} & 0.0094 & 1.5588 & -0.0192 & $-2.2643^{**}$ & 0.0074 & 0.4067 & 0.0151 & 0.7886 & -0.0399 & $-3.3912^{***}$ & 0.0422 & $3.1154^{***}$ & 0.0112 & 0.0104 \\
			\multicolumn{1}{c}{Copula-GARCH} & 0.0037 & 0.3554 & -0.0136 & -1.0617 & -0.0341 & -1.2119 & -0.0050 & -0.1722 & 0.0219 & 1.3918 & 0.0683 & $3.8087^{***}$ & 0.0052 & 0.0044 \\
			&       &       &       &       &       &       &       &       &       &       &       &       &       &  \\
			\multicolumn{15}{c}{\textit{Panel B: Before Transaction Costs}} \\
			&       &       &       &       &       &       &       &       &       &       &       &       &       &  \\
			\multicolumn{1}{c}{Distance (2.0$\sigma$)} & 0.0164 & $2.3369^{**}$ & -0.0058 & -0.6415 & 0.0095 & 0.4750 & 0.0128 & 0.6342 & -0.0510 & $-3.8367^{***}$ & 0.0501 & $3.6721^{***}$ & 0.0135 & 0.0127 \\
			\multicolumn{1}{c}{Distance (0.75$\sigma$)} & 0.0120 & $1.9974^{**}$ & -0.0192 & $-2.2593^{**}$ & 0.0070 & 0.3838 & 0.0158 & 0.8213 & -0.0401 & $-3.3994^{***}$ & 0.0428 & $3.1525^{***}$ & 0.0114 & 0.0106 \\
			\multicolumn{1}{c}{Copula-GARCH} & 0.0236 & $2.2209^{**}$ & -0.0133 & -1.0247 & -0.0344 & -1.2143 & -0.0040 & -0.1355 & 0.0228 & 1.4331 & 0.0694 & $3.8327^{***}$ & 0.0053 & 0.0045  \\
			\bottomrule
		\end{tabularx}\%
		\begin{tablenotes}
			\item \textit{Note:} \scriptsize  This table shows results of regressing daily portfolio return series onto \citet*{ff93}'s three research factors plus momentum and short-term reversal from July 1991 to December 2015 (6173 observations) for pairs traded according to the one-day waiting period rule. Section 1 shows the Return on Committed Capital and Section 2 on Fully Invested Capital. Panel A lists the results after transaction costs and Panel B before transaction costs. The t-statistics are computed using Newey-West standard errors with six lags.
			\item \scriptsize $^{\ast}$. $^{\ast\ast}$. $^{\ast\ast\ast}$  significant at 1\\%. 5\\% and 10\\% levels, respectively.
		\end{tablenotes}
	\end{threeparttable}\%
	\label{tab:table116}\%
\end{sidewaystable}\%


\begin{sidewaystable}
	\caption{Systematic risk of Top 20 pairs with a one-day waiting period: \citet*{ff15}'s five factors.}
	\begin{threeparttable}[H]
		\centering \scriptsize
		\begin{tabularx}{\textwidth}{@{\extracolsep{\fill}}lllllllllllllll@{}}
			\toprule
			\multicolumn{1}{c}{Strategy} & \multicolumn{1}{c}{Intercept} & \multicolumn{1}{c}{t-stat} & \multicolumn{1}{c}{Rm-Rf} & \multicolumn{1}{c}{t-stat} & \multicolumn{1}{c}{SMB} & \multicolumn{1}{c}{t-stat} & \multicolumn{1}{c}{HML} & \multicolumn{1}{c}{t-stat} & \multicolumn{1}{c}{RMW} & \multicolumn{1}{c}{t-stat} & \multicolumn{1}{c}{CMA} & \multicolumn{1}{c}{t-stat} & \multicolumn{1}{c}{$R^{2}$} & \multicolumn{1}{c}{$R^{2}_{adj}$} \\
			\midrule
			\multicolumn{15}{c}{\textbf{Section 1: Return on Committed Capital}} \\
			\multicolumn{15}{c}{\textit{Panel A: After Transaction Costs}} \\
			\multicolumn{1}{c}{} & \multicolumn{1}{c}{} & \multicolumn{1}{c}{} & \multicolumn{1}{c}{} & \multicolumn{1}{c}{} & \multicolumn{1}{c}{} & \multicolumn{1}{c}{} & \multicolumn{1}{c}{} &       &       &       &       &       &       &  \\
			\multicolumn{1}{c}{Distance (2.0$\sigma$)} & 0.0079 & $2.1230^{**}$ & 0.0042 & 0.8184 & 0.0012 & 0.0976 & 0.0240 & $2.4015^{**}$ & -0.0039 & -0.2387 & -0.0432 & $-2.4212^{**}$ & 0.0036 & 0.0028 \\
			\multicolumn{1}{c}{Distance (0.75$\sigma$)} & 0.0110 & $2.6847^{***}$ & -0.0016 & -0.2724 & 0.0107 & 0.8204 & 0.0240 & $1.9361^{*}$ & 0.0109 & 0.6303 & -0.0475 & $-2.3591^{**}$ & 0.0026 & 0.0018 \\
			\multicolumn{1}{c}{Copula-GARCH} & 0.0020 & 0.7021 & 0.0021 & 0.5489 & -0.0037 & -0.4993 & -0.0044 & -0.5557 & -0.0011 & -0.0841 & 0.0011 & 0.0735 & 0.0004 & -0.0004 \\
			&       &       &       &       &       &       &       &       &       &       &       &       &       &  \\
			\multicolumn{15}{c}{\textit{Panel B: Before Transaction Costs}} \\
			&       &       &       &       &       &       &       &       &       &       &       &       &       &  \\
			\multicolumn{1}{c}{Distance (2.0$\sigma$)} & 0.0089 & $2.4084^{**}$ & 0.0043 & 0.8427 & 0.0007 & 0.0628 & 0.0243 & $2.4284^{**}$ & -0.0039 & -0.2424 & -0.0430 & $-2.4068^{**}$ & 0.0036 & 0.0028 \\
			\multicolumn{1}{c}{Distance (0.75$\sigma$)} & 0.0131 & $3.1947^{***}$ & -0.0014 & -0.2450 & 0.0102 & 0.7891 & 0.0244 & $1.9659^{**}$ & 0.0109 & 0.6268 & -0.0473 & $-2.3426^{**}$ & 0.0026 & 0.0018 \\
			\multicolumn{1}{c}{Copula-GARCH} & 0.0119 & $4.2145^{***}$ & 0.0026 & 0.6552 & -0.0038 & -0.5043 & -0.0043 & -0.5341 & -0.0008 & -0.0636 & 0.0016 & 0.1087 & 0.0004 & -0.0004 \\
			&       &       &       &       &       &       &       &       &       &       &       &       &       &  \\
			\midrule
			\multicolumn{15}{c}{\textbf{Section 2: Return on Fully Invested Capital}} \\
			\multicolumn{15}{c}{\textit{Panel A: After Transaction Costs}} \\
			&       &       &       &       &       &       &       &       &       &       &       &       &       &  \\
			\multicolumn{1}{c}{Distance (2.0$\sigma$)} & 0.0172 & $2.5038^{**}$ & 0.0101 & 1.0910 & 0.0081 & 0.3533 & 0.0523 & $2.5516^{**}$ & -0.0036 & -0.1176 & -0.0638 & $-2.0675^{**}$ & 0.0034 & 0.0026 \\
			\multicolumn{1}{c}{Distance (0.75$\sigma$)} & 0.0110 & $1.8547^{*}$ & -0.0025 & -0.2896 & 0.0131 & 0.6339 & 0.0447 & $2.0690^{**}$ & 0.0252 & 0.9577 & -0.0519 & $-1.7201^{*}$ & 0.0028 & 0.0020 \\
			\multicolumn{1}{c}{Copula-GARCH} & 0.0101 & 1.0000 & -0.0022 & -0.1591 & -0.0347 & -1.0716 & -0.0283 & -0.8546 & -0.0132 & -0.3020 & 0.0098 & 0.2115 & 0.0009 & 0.0001 \\
			&       &       &       &       &       &       &       &       &       &       &       &       &       &  \\
			\multicolumn{15}{c}{\textit{Panel B: Before Transaction Costs}} \\
			&       &       &       &       &       &       &       &       &       &       &       &       &       &  \\
			\multicolumn{1}{c}{Distance (2.0$\sigma$)} & 0.0190 & $2.7598^{***}$ & 0.0103 & 1.1155 & 0.0076 & 0.3319 & 0.0527 & $2.5719^{**}$ & -0.0036 & -0.1189 & -0.0635 & $-2.0552^{**}$ & 0.0035 & 0.0026 \\
			\multicolumn{1}{c}{Distance (0.75$\sigma$)} & 0.0137 & $2.3014^{**}$ & -0.0023 & -0.2647 & 0.0126 & 0.6132 & 0.0452 & $2.0860^{**}$ & 0.0252 & 0.9554 & -0.0514 & $-1.7032^{*}$ & 0.0029 & 0.0020 \\
			\multicolumn{1}{c}{Copula-GARCH} & 0.0301 & $2.9442^{***}$ & -0.0016 & -0.1164 & -0.0346 & -1.0636 & -0.0278 & -0.8339 & -0.0122 & -0.2779 & 0.0095 & 0.2040 & 0.0008 & 0.0000 \\
			\bottomrule
		\end{tabularx}\%
		\begin{tablenotes}
			\item \textit{Note:} \scriptsize  This table shows results of regressing daily portfolio return series onto \citet*{ff15}'s five research factors from July 1991 to December 2015 (6173 observations) for pairs traded according to the one-day waiting period rule. Section 1 shows the Return on Committed Capital and Section 2 on Fully Invested Capital. Panel A lists the results after transaction costs and Panel B before transaction costs. The t-statistics are computed using Newey-West standard errors with six lags.
			\item \scriptsize $^{\ast}$. $^{\ast\ast}$. $^{\ast\ast\ast}$  significant at 1\\%. 5\\% and 10\\% levels, respectively.
		\end{tablenotes}
	\end{threeparttable}\%
	\label{tab:table117}\%
\end{sidewaystable}

\vspace{0.3cm}


\begin{sidewaystable}
	\caption{Systematic risk of Top 101-120 pairs with a one-day waiting period: \citet*{ff93}'s three factors plus Momentum and Short-Term Reversal.}
	\begin{threeparttable}[H]
		\centering \scriptsize
		\begin{tabularx}{\textwidth}{@{\extracolsep{\fill}}lllllllllllllll@{}}
			\toprule
			\multicolumn{1}{c}{Strategy} & \multicolumn{1}{c}{Intercept} & \multicolumn{1}{c}{t-stat} & \multicolumn{1}{c}{Rm-Rf} & \multicolumn{1}{c}{t-stat} & \multicolumn{1}{c}{SMB} & \multicolumn{1}{c}{t-stat} & \multicolumn{1}{c}{HML} & \multicolumn{1}{c}{t-stat} & \multicolumn{1}{c}{Mom} & \multicolumn{1}{c}{t-stat} & \multicolumn{1}{c}{Rev} & \multicolumn{1}{c}{t-stat} & \multicolumn{1}{c}{$R^{2}$} & \multicolumn{1}{c}{$R^{2}_{adj}$} \\
			\midrule
			\multicolumn{15}{c}{\textbf{Section 1: Return on Committed Capital}} \\
			\multicolumn{15}{c}{\textit{Panel A: After Transaction Costs}} \\
			\multicolumn{1}{c}{} & \multicolumn{1}{c}{} & \multicolumn{1}{c}{} & \multicolumn{1}{c}{} & \multicolumn{1}{c}{} & \multicolumn{1}{c}{} & \multicolumn{1}{c}{} & \multicolumn{1}{c}{} &       &       &       &       &       &       &  \\
			\multicolumn{1}{c}{Distance (2.0$\sigma$)} & 0.0124 & $2.9357^{***}$ & -0.0004 & -0.0776 & -0.0281 & $-2.6594^{***}$ & -0.0211 & $-1.8800^{*}$ & -0.0557 & $-7.8134^{***}$ & 0.0383 & $4.5480^{***}$ & 0.0337 & 0.0329 \\
			\multicolumn{1}{c}{Distance (0.75$\sigma$)} & 0.0126 & $2.7120^{***}$ & -0.0104 & $-1.6871^{*}$ & -0.0309 & $-2.6284^{***}$ & -0.0239 & $-1.8332^{*}$ & -0.0586 & $-7.0545^{***}$ & 0.0499 & $5.1599^{***}$ & 0.0324 & 0.0316 \\
			\multicolumn{1}{c}{Copula-GARCH} & 0.0035 & 1.1647 & -0.0039 & -0.9075 & 0.0029 & 0.3457 & -0.0078 & -0.9201 & 0.0054 & 1.0918 & 0.0298 & $3.7997^{***}$ & 0.0111 & 0.0103 \\
			&       &       &       &       &       &       &       &       &       &       &       &       &       &  \\
			\multicolumn{15}{c}{\textit{Panel B: Before Transaction Costs}} \\
			&       &       &       &       &       &       &       &       &       &       &       &       &       &  \\
			\multicolumn{1}{c}{Distance (2.0$\sigma$)} & 0.0134 & $3.1582^{***}$ & -0.0005 & -0.0878 & -0.0284 & $-2.6849^{***}$ & -0.0208 & $-1.8462^{*}$ & -0.0558 & $-7.8249^{***}$ & 0.0387 & $4.5893^{***}$ & 0.0339 & 0.0331 \\
			\multicolumn{1}{c}{Distance (0.75$\sigma$)} & 0.0145 & $3.1004^{***}$ & -0.0104 & $-1.6946^{*}$ & -0.0312 & $-2.6542^{***}$ & -0.0235 & $-1.7969^{*}$ & -0.0587 & $-7.0596^{***}$ & 0.0504 & $5.2049^{***}$ & 0.0326 & 0.0319 \\
			\multicolumn{1}{c}{Copula-GARCH} & 0.0122 & $3.9696^{***}$ & -0.0038 & -0.8746 & 0.0029 & 0.3486 & -0.0074 & -0.8552 & 0.0055 & 1.0975 & 0.0305 & $3.8410^{***}$ & 0.0113 & 0.0105 \\
			&       &       &       &       &       &       &       &       &       &       &       &       &       &  \\
			\midrule
			\multicolumn{15}{c}{\textbf{Section 2: Return on Fully Invested Capital}} \\
			\multicolumn{15}{c}{\textit{Panel A: After Transaction Costs}} \\
			&       &       &       &       &       &       &       &       &       &       &       &       &       &  \\
			\multicolumn{1}{c}{Distance (2.0$\sigma$)} & 0.0198 & $2.4371^{**}$ & -0.0021 & -0.1846 & -0.0413 & $-2.0226^{**}$ & -0.0259 & -1.1090 & -0.1097 & $-7.6334^{***}$ & 0.0785 & $4.9291^{***}$ & 0.0352 & 0.0344 \\
			\multicolumn{1}{c}{Distance (0.75$\sigma$)} & 0.0182 & $2.5956^{***}$ & -0.0300 & $-2.2044^{**}$ & -0.0172 & -0.9706 & -0.0375 & $-1.7328^{*}$ & -0.0821 & $-6.2230^{***}$ & 0.0649 & $3.9850^{***}$ & 0.0241 & 0.0233 \\
			\multicolumn{1}{c}{Copula-GARCH} & 0.0087 & 0.7312 & -0.0275 & -1.5376 & 0.0467 & 1.3980 & -0.0306 & -0.9349 & 0.0116 & 0.5610 & 0.0924 & $2.7083^{***}$ & 0.0076 & 0.0068 \\
			&       &       &       &       &       &       &       &       &       &       &       &       &       &  \\
			\multicolumn{15}{c}{\textit{Panel B: Before Transaction Costs}} \\
			&       &       &       &       &       &       &       &       &       &       &       &       &       &  \\
			\multicolumn{1}{c}{Distance (2.0$\sigma$)} & 0.0215 & $2.6404^{***}$ & -0.0021 & -0.1851 & -0.0417 & $-2.0421^{**}$ & -0.0254 & -1.0849 & -0.1098 & $-7.6438^{***}$ & 0.0790 & $4.9589^{***}$ & 0.0354 & 0.0346 \\
			\multicolumn{1}{c}{Distance (0.75$\sigma$)} & 0.0207 & $2.9467^{***}$ & -0.0300 & $-2.2014^{**}$ & -0.0176 & -0.9915 & -0.0369 & $-1.7052^{*}$ & -0.0822 & $-6.2318^{***}$ & 0.0654 & $4.0119^{***}$ & 0.0242 & 0.0234 \\
			\multicolumn{1}{c}{Copula-GARCH} & 0.0279 & $2.3272^{**}$ & -0.0272 & -1.5079 & 0.0474 & 1.4099 & -0.0295 & -0.8926 & 0.0116 & 0.5584 & 0.0936 & $2.7261^{***}$ & 0.0076 & 0.0068  \\
			\bottomrule
		\end{tabularx}\%
		\begin{tablenotes}
			\item \textit{Note:} \scriptsize  This table shows results of regressing daily portfolio return series onto \citet*{ff93}'s three research factors plus momentum and short-term reversal from July 1991 to December 2015 (6173 observations) for pairs traded according to the one-day waiting period rule. Section 1 shows the Return on Committed Capital and Section 2 on Fully Invested Capital. Panel A lists the results after transaction costs and Panel B before transaction costs. The t-statistics are computed using Newey-West standard errors with six lags.
			\item \scriptsize $^{\ast}$. $^{\ast\ast}$. $^{\ast\ast\ast}$  significant at 1\\%. 5\\% and 10\\% levels, respectively.
		\end{tablenotes}
	\end{threeparttable}\%
	\label{tab:table118}\%
\end{sidewaystable}\%


\begin{sidewaystable}
	\caption{Systematic risk of Top 101-120 pairs with a one-day waiting period: \citet*{ff15}'s five factors.}
	\begin{threeparttable}[H]
		\centering \scriptsize
		\begin{tabularx}{\textwidth}{@{\extracolsep{\fill}}lllllllllllllll@{}}
			\toprule
			\multicolumn{1}{c}{Strategy} & \multicolumn{1}{c}{Intercept} & \multicolumn{1}{c}{t-stat} & \multicolumn{1}{c}{Rm-Rf} & \multicolumn{1}{c}{t-stat} & \multicolumn{1}{c}{SMB} & \multicolumn{1}{c}{t-stat} & \multicolumn{1}{c}{HML} & \multicolumn{1}{c}{t-stat} & \multicolumn{1}{c}{RMW} & \multicolumn{1}{c}{t-stat} & \multicolumn{1}{c}{CMA} & \multicolumn{1}{c}{t-stat} & \multicolumn{1}{c}{$R^{2}$} & \multicolumn{1}{c}{$R^{2}_{adj}$} \\
			\midrule
			\multicolumn{15}{c}{\textbf{Section 1: Return on Committed Capital}} \\
			\multicolumn{15}{c}{\textit{Panel A: After Transaction Costs}} \\
			\multicolumn{1}{c}{} & \multicolumn{1}{c}{} & \multicolumn{1}{c}{} & \multicolumn{1}{c}{} & \multicolumn{1}{c}{} & \multicolumn{1}{c}{} & \multicolumn{1}{c}{} & \multicolumn{1}{c}{} &       &       &       &       &       &       &  \\
			\multicolumn{1}{c}{Distance (2.0$\sigma$)} & 0.0127 & $3.0608^{***}$ & 0.0225 & $3.7573^{***}$ & -0.0239 & $-2.1129^{**}$ & 0.0032 & 0.2951 & 0.0264 & $1.6477^{*}$ & -0.0146 & -0.7831 & 0.0081 & 0.0073 \\
			\multicolumn{1}{c}{Distance (0.75$\sigma$)} & 0.0135 & $2.9542^{***}$ & 0.0169 & $2.5822^{***}$ & -0.0253 & $-2.0558^{**}$ & -0.0010 & -0.0777 & 0.0317 & $1.7620^{*}$ & -0.0127 & -0.6328 & 0.0054 & 0.0046 \\
			\multicolumn{1}{c}{Copula-GARCH} & 0.0060 & $1.9538^{*}$ & 0.0024 & 0.5389 & 0.0080 & 0.9147 & -0.0098 & -0.9848 & 0.0152 & 1.3619 & -0.0171 & -1.1294 & 0.0027 & 0.0019  \\
			&       &       &       &       &       &       &       &       &       &       &       &       &       &  \\
			\multicolumn{15}{c}{\textit{Panel B: Before Transaction Costs}} \\
			&       &       &       &       &       &       &       &       &       &       &       &       &       &  \\
			\multicolumn{1}{c}{Distance (2.0$\sigma$)} & 0.0137 & $3.2926^{***}$ & 0.0226 & $3.7662^{***}$ & -0.0242 & $-2.1378^{**}$ & 0.0034 & 0.3162 & 0.0263 & 1.6422 & -0.0144 & -0.7691 & 0.0081 & 0.0073 \\
			\multicolumn{1}{c}{Distance (0.75$\sigma$)} & 0.0154 & $3.3537^{***}$ & 0.0170 & $2.5981^{***}$ & -0.0256 & $-2.0807^{**}$ & -0.0007 & -0.0555 & 0.0318 & $1.7642^{*}$ & -0.0125 & -0.6208 & 0.0054 & 0.0046 \\
			\multicolumn{1}{c}{Copula-GARCH} & 0.0147 & $4.7639^{***}$ & 0.0026 & 0.5799 & 0.0081 & 0.9165 & -0.0093 & -0.9262 & 0.0152 & 1.3533 & -0.0175 & -1.1488 & 0.0027 & 0.0019 \\
			&       &       &       &       &       &       &       &       &       &       &       &       &       &  \\
			\midrule
			\multicolumn{15}{c}{\textbf{Section 2: Return on Fully Invested Capital}} \\
			\multicolumn{15}{c}{\textit{Panel A: After Transaction Costs}} \\
			&       &       &       &       &       &       &       &       &       &       &       &       &       &  \\
			\multicolumn{1}{c}{Distance (2.0$\sigma$)} & 0.0205 & $2.5948^{***}$ & 0.0441 & $3.4928^{***}$ & -0.0300 & -1.3662 & 0.0248 & 0.9820 & 0.0631 & $2.1787^{**}$ & -0.0404 & -1.1586 & 0.0079 & 0.0071 \\
			\multicolumn{1}{c}{Distance (0.75$\sigma$)} & 0.0190 & $2.7718^{***}$ & 0.0083 & 0.5366 & -0.0114 & -0.6281 & -0.0120 & -0.4778 & 0.0382 & 1.4724 & 0.0038 & 0.1143 & 0.0012 & 0.0004 \\
			\multicolumn{1}{c}{Copula-GARCH} & 0.0167 & 1.4054 & -0.0122 & -0.6551 & 0.0590 & $1.7106^{*}$ & -0.0208 & -0.5287 & 0.0319 & 0.7742 & -0.0874 & -1.4980 & 0.0029 & 0.0020 \\
			&       &       &       &       &       &       &       &       &       &       &       &       &       &  \\
			\multicolumn{15}{c}{\textit{Panel B: Before Transaction Costs}} \\
			&       &       &       &       &       &       &       &       &       &       &       &       &       &  \\
			\multicolumn{1}{c}{Distance (2.0$\sigma$)} & 0.0222 & $2.8078^{***}$ & 0.0442 & $3.5044^{***}$ & -0.0305 & -1.3866 & 0.0252 & 0.9968 & 0.0629 & $2.1700^{**}$ & -0.0399 & -1.1453 & 0.0079 & 0.0071 \\
			\multicolumn{1}{c}{Distance (0.75$\sigma$)} & 0.0216 & $3.1346^{***}$ & 0.0085 & 0.5509 & -0.0118 & -0.6497 & -0.0116 & -0.4625 & 0.0381 & 1.4701 & 0.0042 & 0.1259 & 0.0012 & 0.0004 \\
			\multicolumn{1}{c}{Copula-GARCH} & 0.0360 & $3.0136^{***}$ & -0.0117 & -0.6260 & 0.0597 & $1.7204^{*}$ & -0.0193 & -0.4860 & 0.0318 & 0.7660 & -0.0891 & -1.5160 & 0.0028 & 0.0020 \\
			\bottomrule
		\end{tabularx}\%
		\begin{tablenotes}
			\item \textit{Note:} \scriptsize  This table shows results of regressing daily portfolio return series onto \citet*{ff15}'s five research factors from July 1991 to December 2015 (6173 observations) for pairs traded according to the one-day waiting period rule. Section 1 shows the Return on Committed Capital and Section 2 on Fully Invested Capital. Panel A lists the results after transaction costs and Panel B before transaction costs. The t-statistics are computed using Newey-West standard errors with six lags.
			\item \scriptsize $^{\ast}$. $^{\ast\ast}$. $^{\ast\ast\ast}$  significant at 1\\%. 5\\% and 10\\% levels, respectively.
		\end{tablenotes}
	\end{threeparttable}\%
	\label{tab:table119}\%
\end{sidewaystable}\%

\newpage
	
	\subsection{Robustness checks of the performance of Excess Returns and Sharpe Ratios}
	
	One possible criticism might be that the conclusions are based on only one realization of the stochastic process of asset returns computed from the observed series of prices, since among thousands of different strategies is very likely that we find some that show superior performance in terms of excess return or Sharpe Ratio. In order to mitigate data-snooping criticisms, we use the stationary bootstrap\footnote{It allows for weakly dependent correlation over time.} of \citet*{pr94} to compute the bootstrap p-values using the methodology proposed by \citet*{lw08}.
	
	The bootstrap method is used to obtain the distribution of a null hypothesis. Here we want to investigate if the average excess return and the Sharpe Ratio of the Copula methods beat the reference (distance) strategies. To construct the distributions, we bootstrapped the original time series B=10,000 times. Our bootstrapped null distributions result from Theorem 2 of \citet*{pr94}.
	
	We select the optimal block length for the stationary bootstrap following \citet*{pw04}. As the optimal bootstrap block-length is different for each strategy, we average\footnote{We also use the optimal block size for each strategy. We find that the results are robust to the optimal block size, and therefore, we do not report them here.} the block-lengths found to proceed the comparisons between the Copula and the benchmark strategies.
	
	To test the hypotheses that the average excess returns and Sharpe Ratios of copula strategy are equal to that of distance methods, that is,
	\begin{equation}
	H_{0}:\mu_{c}-\mu_{d}=0  \ \ \textrm{and}
	\ \  H_{0}:\frac{\mu_{c}}{\sigma_{c}}-\frac{\mu_{d}}{\sigma_{d}}=0,
	\label{eq:eq153}
	\end{equation}
	we compute, following \citet*{davison1997}, a two-sized $p$-value using $B=10000$ (stationary) bootstrap re-samples as follows:
	\begin{equation}
	p_{sboot}=
	\begin{cases}
	2\frac{\sum_{b=1}^{B}\mathbb{I}\{0< t^{\ast(b)}\}+1}{B+1}, &\text{if} ~~median\left\{ t^{\ast \left( 1\right) },...,t^{\ast \left( B\right)}\right\} > 0, \\
	2\frac{\sum_{b=1}^{B}\mathbb{I}\{0\geq t^{\ast(b)}\}+1}{B+1}, &\text{otherwise},
	\end{cases}
	\label{eq:eq152}
	\end{equation}
	where $\mathbb{I}$ is the indicator function, $t^{\ast(b)}$ are the values in each block stationary bootstrap replication, and B denotes the number of bootstrap replications.
	
	Table \ref{tab:table120} report the bootstrap $p$-values for testing the null hypotheses represented by (\ref{eq:eq153}). We compare the copula method with each of the distance approaches (0.75 and 2.0 standard deviations) from 1991/2-2015 for all investment scenarios, \emph{i.e.}, without delay and waiting one-day period for the Top 5, Top 20, and Top 101-120 pairs, respectively, and before and after costs.
	
	Overall, these results reinforce the ones previously obtained. As can be observed, the copula approach significantly outperforms the distance strategies when ignoring the costs and when a rapid execution of the trade is made for Top 5 and Top 20 pairs, in particular in terms of risk-adjusted returns for both weighting structures.

	

	\medskip
	
	\begin{threeparttable}[H]
		\centering \scriptsize
		\caption{Bootstrap p-values computed from B=10,000 replications for testing the null hypotheses of equality of the average excess returns and Sharpe Ratios over the period between July 1991 and December 2015.}
		\begin{tabularx}{\textwidth}{@{\extracolsep{\fill}}lllllll@{}}
			\toprule
			& & \multicolumn{2}{c}{Copula vs Distance (0.75$\sigma$)} & \multicolumn{1}{c}{} & \multicolumn{2}{c}{Copula vs Distance (2.0$\sigma$)} \\
			\cmidrule{3-4}  \cmidrule{6-7}
			\multicolumn{1}{c}{Scenario} & & \multicolumn{1}{c}{Return} & Sharpe Ratio &       & \multicolumn{1}{c}{Return}& Sharpe Ratio \\
			\midrule
			& \multicolumn{6}{c}{Section 1: Committed Capital} \\
			\midrule
					d0p5c0 & & 0.6260 & $0.0812^{*}(>)$ &       & 0.2490 & $0.0452^{**}(>)$ \\
			d0p5c1 & & 0.4962 & 0.7944 &       & 0.7962 & 0.6960 \\
			d1p5c0 & & 0.3648 & 0.7426 &       & 0.8464 & 0.8692 \\
			d1p5c1 & & $0.0472^{**}(<)$ & 0.0778 &       & 0.1468 & 0.1844 \\
			d0p20c0 & & 0.1792 & $0.0004^{***}(>)$ &       & $0.0080^{***}(>)$ & $0.0000^{***}(>)$ \\
			d0p20c1 & & 0.7358 & 0.2472 &       & 0.5154 & $0.0752^{*}(>)$ \\
			d1p20c0 & & 0.8222 & 0.3640 &       & 0.4792 & 0.1384 \\
			d1p20c1 & & $0.0580^{*}(<)$ & 0.1552 &       & 0.1848 & 0.3062 \\
			d0p101c0 & & 0.5286 & 0.4368 &       & 0.5380 & 0.6622 \\
			d0p101c1 & & $0.0746^{*}(<)$ & 0.3864 &       & $0.0464^{**}(<)$ & 0.2112 \\
			d1p101c0 & & 0.8030 & 0.3062 &       & 0.9836 & 0.3122 \\
			d1p101c1 & & 0.1330 & 0.4608 &       & 0.1594 & 0.4148 \\
			\midrule
			& \multicolumn{6}{c}{Section 2: Fully Invested Capital} \\
			\midrule
			d0p5c0 & & $0.0032^{***}(>)$ & $0.0322^{**}(>)$ &       & $0.0462^{**}(>)$ & $0.0922^{*}(>)$ \\
			d0p5c1 & & 0.2062 & 0.4498 &       & 0.6622 & 0.8478 \\
			d1p5c0 & & 0.9902 & 0.9472 &       & 0.9602 & 0.8672 \\
			d1p5c1 & & 0.3884 & 0.3498 &       & 0.3564 & 0.2894 \\
			d0p20c0 & & $0.0000^{***}(>)$ & $0.0080^{***}(>)$ &       & $0.0004^{***}(>)$ & $0.0086^{***}(>)$ \\
			d0p20c1 & & 0.1580 & 0.7352 &       & 0.2828 & 0.6538 \\
			d1p20c0 & & 0.1932 & 0.7160 &       & 0.3970 & 0.9732 \\
			d1p20c1 & & 0.8636 & 0.4710 &       & 0.5218 & 0.2406 \\
			d0p101c0 & & 0.3106 & 0.8752 &       & 0.8958 & 0.5714 \\
			d0p101c1 & & 0.2292 & $0.0544^{***}(<)$ &       & $0.0452^{**}(<)$ & $0.0148^{**}(<)$ \\
			d1p101c0 & & 0.3380 & 0.8660 &       & 0.4404 & 0.9988 \\
			d1p101c1 & & 0.7880 & 0.2986 &       & 0.6362 & 0.3088 \\
			\bottomrule
		\end{tabularx}\%
		\begin{tablenotes}
			\item \textit{Note:} \scriptsize This table reports the bootstrap p-values for testing the null hypothesis of equality of the average excess returns and the Sharpe Ratios of Copula and distance strategies for Top 5, 20 and 101-120 pairs over the period between July 1991 and December 2015 (6173 observations). The column labeled Scenario contains symbol labels for trading with no delay or a one-day waiting period (d0 and d1, respectively), for Top 5, 20 and 101-120 pairs (p5, p20 and p101, respectively) and before or after costs (c0 and c1, respectively). The symbol labels (>) and (<) indicate that the respective null hypothesis is rejected in favor of the alternative and that the average excess returns and/or the Sharpe Ratio of the Copula strategy are found greater or less than the the average excess returns and/or the Sharpe Ratio of the distance strategies, respectively.
			\item \scriptsize $^{\ast\ast\ast}$, $^{\ast\ast}$, $^{\ast}$  significant at 1\\%, 5\\% and 10\\% levels, respectively.
		\end{tablenotes}
		\label{tab:table120}\%
	\end{threeparttable}\%
	
	\medskip
	
	\subsection{Sub-period analysis}
	
	We split the full sample period into five sub-periods: (1) July 1991 to December 1999, (2) January 2000 to December 2002, (3) January 2003 to June 2007, (4) July 2007 to June 2009, and (5) July 2009 to December 2015. The second sub-period corresponds to the bear market that comprises the dotcom crisis and the September 11th terrorist attack. The fourth sub-period corresponds to the subprime mortgage financial crisis.
	
	Figure \ref{fig:fig101} shows strong signs of the superiority of the copula approach regarding the average number of pairs traded per six-month in each sub-period - since every trading signal is an opportunity to profit - strengthening the results displayed in Table \ref{tab:table107}. There is a noticeable increase of tradeable signals in the last two sub-periods.
	
%	\begin{figure}[H]
%		\centering
%		\includegraphics[scale=0.6]{fig5_p1_rev.pdf}
%		\caption{Average number of pairs traded per six-month for Top 5, 20 and 101-120 pairs for each sub-period}
%		\%\caption*{comments}
%		\label{fig:fig101}
%	\end{figure}
	

	Figures \ref{fig:fig102} to \ref{fig:fig105} show the profitability patterns of the three strategies for Top 5, 20 an 101-120 pairs, respectively, for each sub-period. Figure \ref{fig:fig102} displays the average excess returns when a fast trade can be executed and before costs for Top 5 (top), Top 20 (center) and Top 101-120 pairs (bottom) on committed (left) and fully invested capital (right). Figure \ref{fig:fig103} shows the excess returns with no delay and after costs, while Figure \ref{fig:fig104} and Figure \ref{fig:fig105} consider returns with “one-day rule”, before and after costs, respectively.
	
	The patterns found in the figures bolster the average and cumulative excess returns displayed in Tables \ref{tab:table101} to \ref{tab:table106} and Figures \ref{fig:fig108} to \ref{fig:fig111}, respectively. Overall, we can note that the copula strategy exhibit a relative-to-benchmark impressive out-performance in the last subperiod, \emph{i.e.}, after the subprime financial crisis, even after costs and waiting one day after the signal to start and close positions. However, it delivers a very poor performance from 2000-2002. We had previously pointed out the pattern from 1998-2000 when analyzing Figures \ref{fig:fig108} to \ref{fig:fig111}.
	
	It should also be observed that the copula method also delivers a very strong relative out-of-sample performance during the first subperiod (91/2-99), even after imposing both constraints for fully invested capital, and a very good performance from 2003-2007/1, at least for Top 20 pairs, and during the subprime mortgage crisis when a quick trade is executed.
	
	Figures \ref{fig:fig112} to \ref{fig:fig115} representing the Sharpe ratios patterns for each of the strategies are shown in the appendix. We can detect similar patterns to Figures \ref{fig:fig108} to \ref{fig:fig111}.
	
	
%		\begin{figure}[H]
%		\centering
%		\includegraphics[scale=0.6]{fig6_p1_rev.pdf}
%		\caption{Average excess returns of pairs trading strategies before costs and with no delay for each sub-period}
%		\%\caption*{comments}
%		\label{fig:fig102}
%	\end{figure}
%	
%	\begin{figure}[H]
%		\centering
%		\includegraphics[scale=0.6]{fig7_p1_rev.pdf}
%		\caption{Average excess returns of pairs trading strategies after costs and with no delay for each sub-period}
%		\%\caption*{comments}
%		\label{fig:fig103}
%	\end{figure}
%	
%	\begin{figure}[H]
%		\centering
%		\includegraphics[scale=0.6]{fig8_p1_rev.pdf}
%		\caption{Average excess returns of pairs trading strategies before costs and with one-day waiting period for each sub-period}
%		\%\caption*{comments}
%		\label{fig:fig104}
%	\end{figure}
%
%\begin{figure}[H]
%	\centering
%	\includegraphics[scale=0.6]{fig9_p1_rev.pdf}
%	\caption{Average excess returns of pairs trading strategies after costs and with one-day waiting period for each sub-period}
%	\%\caption*{comments}
%	\label{fig:fig105}
%\end{figure}
	
	
	Finally, Tables \ref{tab:table121} to \ref{tab:table125} present the bootstrap $p$-values for testing the null hypotheses represented by (\ref{eq:eq153}) over the sub-periods.
	
	As we can observe, the results in the first sub-period (July 1991 to December 1999) reinforce some patterns seen in Figures \ref{fig:fig102} to \ref{fig:fig105}, eespecially when both constraints are not active on committed capital and before delay (significant at least at 5\\%) on fully invested capital. It is also clear the significant relative poor performance in the second subperiod (January 2000 to December 2002), eespecially when compared to the 2.0$\sigma$ trigger point, at least at 10\\%.
	
	For the next subperiod (January 2003 to June 2007) the copula strategy shows a significant better performance for Top 20 pairs, before costs and with no delay to start the positions. During the global financial crisis (July 2007 to June 2009) we do not find any significative superior performance of any of the strategies. Lastly, we find some evidences that the copula method outperforms the distance strategies in the last subperiod (July 2009 to December 2015) at least before costs.
	
	\medskip
	
	\begin{threeparttable}[H]
		\centering \scriptsize
		\caption{Bootstrap p-values computed from B=10,000 replications for testing the null hypotheses of equality of the average excess returns and Sharpe Ratios over the period between July 1991 and December 1999.}
		\begin{tabularx}{\textwidth}{@{\extracolsep{\fill}}lllllll@{}}
			\toprule
			& & \multicolumn{2}{c}{Copula vs Distance (0.75$\sigma$)} & \multicolumn{1}{c}{} & \multicolumn{2}{c}{Copula vs Distance (2.0$\sigma$)} \\
			\cmidrule{3-4}  \cmidrule{6-7}
			\multicolumn{1}{c}{Scenario} & & \multicolumn{1}{c}{Return} & Sharpe Ratio &       & \multicolumn{1}{c}{Return}& Sharpe Ratio \\
			\midrule
			& \multicolumn{6}{c}{Section 1: Committed Capital} \\
			\midrule
			d0p5c0 & & 0.2862 & $0.0174^{**}(>)$ &       & $0.0138^{**}(>)$ & $0.0016^{***}(>)$ \\
			d0p5c1 & & 0.7096 & 0.1466 &       & 0.1244 & $0.0224^{**}(>)$ \\
			d1p5c0 & & 0.5002 & 0.9110 &       & 0.7814 & 0.5614 \\
			d1p5c1 & & 0.1574 & 0.2648 &       & 0.5350 & 0.6320 \\
			d0p20c0 & & 0.1134 & $0.0000^{***}(>)$ &       & $0.0008^{***}(>)$ & $0.0000^{***}(>)$ \\
			d0p20c1 & & 0.5824 & $0.0136^{**}(>)$ &       & $0.0410^{**}(>)$ & $0.0004^{***}(>)$ \\
			d1p20c0 & & 0.7586 & $0.0830^{*}(>)$ &       & 0.1308 & $0.0138^{**}(>)$ \\
			d1p20c1 & & 0.4178 & 0.9110 &       & 0.8390 & 0.4142 \\
			d0p101c0 & & 0.7660 & 0.3432 &       & 0.9880 & 0.3018 \\
			d0p101c1 & & 0.3328 & 0.9242 &       & 0.4166 & 0.9380 \\
			d1p101c0 & & 0.8996 & 0.1396 &       & 0.5912 & $0.0912^{*}(>)$ \\
			d1p101c1 & & 0.5790 & 0.6530 &       & 0.7846 & 0.5520 \\
			\midrule
			& \multicolumn{6}{c}{Section 2:  Fully Invested Capital } \\
			\midrule
			d0p5c0 & & $0.0030^{***}(>)$ & $0.0400^{**}(>)$ &       & $0.0004^{***}(>)$ & $0.0036^{***}(>)$ \\
			d0p5c1 & & $0.0430^{**}(>)$ & 0.2024 &       & $0.0176^{**}(>)$ & $0.0412^{**}(>)$ \\
			d1p5c0 & & 0.8012 & 0.9528 &       & 0.8346 & 0.9774 \\
			d1p5c1 & & 0.7674 & 0.6314 &       & 0.7062 & 0.5830 \\
			d0p20c0 & & $0.0000^{***}(>)$ & $0.0000^{***}(>)$ &       & $0.0000^{***}(>)$ & $0.0000^{***}(>)$ \\
			d0p20c1 & & $0.0014^{***}(>)$ & $0.0230^{**}(>)$ &       & $0.0010^{***}(>)$ & $0.0042^{***}(>)$ \\
			d1p20c0 & & $0.0324^{**}(>)$ & 0.2132 &       & $0.0890^{*}(>)$ & 0.3692 \\
			d1p20c1 & & 0.2262 & 0.5686 &       & 0.4640 & 0.8886 \\
			d0p101c0 & & $0.0434^{**}(>)$ & 0.5282 &       & 0.1272 & 0.4340 \\
			d0p101c1 & & 0.4176 & 0.7758 &       & 0.6910 & 0.8386 \\
			d1p101c0 & & $0.0378^{**}(>)$ & 0.4352 &       & $0.0470^{**}(>)$ & 0.2574 \\
			d1p101c1 & & 0.1708 & 0.7622 &       & 0.1924 & 0.5740 \\
			\bottomrule
		\end{tabularx}\%
		\begin{tablenotes}
			\item \textit{Note:} \scriptsize This table reports the bootstrap p-values for testing the null hypothesis of equality of the average excess returns and the Sharpe Ratios of Copula and distance strategies for the Top 5, 20 and 101-120 pairs over the period between July 1991 and December 1999 (2149 observations). The column labeled Scenario contains symbol labels for trading with no delay or one day waiting period (d0 and d1, respectively), for Top 5, 20 and 101-120 pairs (p5, p20 and p101, respectively) and before or after costs (c0 and c1, respectively). The symbol labels (>) and (<) indicate that the respective null hypothesis is rejected in favor of the alternative and that the average excess returns and/or the Sharpe Ratio of the Copula strategy are found greater or less than the the average excess returns and/or the Sharpe Ratio of the distance strategies, respectively.
			\item \scriptsize $^{\ast\ast\ast}$, $^{\ast\ast}$, $^{\ast}$  significant at 1\\%, 5\\% and 10\\% levels, respectively.
		\end{tablenotes}
		\label{tab:table121}\%
	\end{threeparttable}\%
	
	\medskip
	
	\begin{threeparttable}[H]
		\centering \scriptsize
		\caption{Bootstrap p-values computed from B=10,000 replications for testing the null hypotheses of equality of the average excess returns and Sharpe Ratios over the period between January 2000 and December 2002.}
		\begin{tabularx}{\textwidth}{@{\extracolsep{\fill}}lllllll@{}}
			\toprule
			& & \multicolumn{2}{c}{Copula vs Distance (0.75$\sigma$)} & \multicolumn{1}{c}{} & \multicolumn{2}{c}{Copula vs Distance (2.0$\sigma$)} \\
			\cmidrule{3-4}  \cmidrule{6-7}
			\multicolumn{1}{c}{Scenario} & & \multicolumn{1}{c}{Return} & Sharpe Ratio &       & \multicolumn{1}{c}{Return}& Sharpe Ratio \\
			\midrule
			& \multicolumn{6}{c}{Section 1: Committed Capital} \\
			\midrule
			d0p5c0 & & 0.3914 & 0.4984 &       & $0.0806^{*}(<)$ & 0.1396 \\
			d0p5c1 & & 0.2764 & 0.3224 &       & $0.0452^{**}(<)$ & $0.0740^{*}(<)$ \\
			d1p5c0 & & 0.3354 & 0.3612 &       & 0.1222 & 0.1504 \\
			d1p5c1 & & 0.2354 & 0.2236 &       & $0.0740^{*}(<)$ & $0.0862^{*}(<)$ \\
			d0p20c0 & & 0.1446 & 0.3032 &       & 0.1396 & 0.2470 \\
			d0p20c1 & & $0.0720^{*}(<)$ & 0.1186 &       & $0.0588^{*}(<)$ & $0.0916^{*}(<)$ \\
			d1p20c0 & & $0.0932^{*}(<)$ & $0.0986^{*}(<)$ &       & $0.0584^{*}(<)$ & $0.0660^{*}(<)$ \\
			d1p20c1 & & $0.0392^{**}(<)$ & $0.0262^{**}(<)$ &       & $0.0146^{**}(<)$ & $0.0140^{**}(<)$ \\
			d0p101c0 & & $0.0610^{*}(<)$ & 0.2566 &       & $0.0454^{**}(<)$ & 0.1388 \\
			d0p101c1 & & $0.0388^{**}(<)$ & 0.1282 &       & $0.0222^{**}(<)$ & $0.0616^{*}(<)$ \\
			d1p101c0 & & $0.0902^{*}(<)$ & 0.2620 &       & 0.1224 & 0.2444 \\
			d1p101c1 & & $0.0458^{**}(<)$ & 0.1188 &       & $0.0582^{*}(<)$ & 0.1132 \\
			\midrule
			& \multicolumn{6}{c}{Section 2:  Fully Invested Capital } \\
			\midrule
			d0p5c0 & & 0.6840 & 0.5996 &       & $0.0606^{*}(<)$ & $0.0572^{*}(<)$ \\
			d0p5c1 & & 0.4476 & 0.3980 &       & $0.0226^{**}(<)$ & $0.0206^{**}(<)$ \\
			d1p5c0 & & 0.2320 & 0.2652 &       & $0.0986^{*}(<)$ & $0.0882^{*}(<)$ \\
			d1p5c1 & & 0.1588 & 0.1826 &       & $0.0614^{*}(<)$ & $0.0540^{*}(<)$ \\
			d0p20c0 & & 0.3580 & 0.1770 &       & 0.2706 & 0.1848 \\
			d0p20c1 & & $0.0994^{*}(<)$ & $0.0544^{*}(<)$ &       & $0.0750^{*}(<)$ & $0.0500^{**}(<)$ \\
			d1p20c0 & & 0.2360 & 0.1310 &       & 0.1980 & 0.1156 \\
			d1p20c1 & & 0.1212 & $0.0792^{*}(<)$ &       & $0.0898^{*}(<)$ & $0.0588^{*}(<)$ \\
			d0p101c0 & & 0.2328 & $0.0480^{**}(<)$ &       & $0.0438^{**}(<)$ & $0.0118^{**}(<)$ \\
			d0p101c1 & & $0.0786^{*}(<)$ & $0.0168^{**}(<)$ &       & $0.0106^{**}(<)$ & $0.0020^{***}(<)$ \\
			d1p101c0 & & 0.3024 & $0.0914^{*}(<)$ &       & 0.2072 & $0.0864^{*}(<)$ \\
			d1p101c1 & & 0.1916 & $0.0538^{*}(<)$ &       & 0.1254 & $0.0524^{*}(<)$ \\
			\bottomrule
		\end{tabularx}\%
		\begin{tablenotes}
			\item \textit{Note:} \scriptsize This table reports the bootstrap p-values for testing the null hypothesis of equality of the average excess returns and the Sharpe Ratios of Copula and distance strategies for the Top 5, 20 and 101-120 pairs over the period between January 2000 and December 2002 (752 observations). The column labeled Scenario contains symbol labels for trading with no delay or one day waiting period (d0 and d1, respectively), for Top 5, 20 and 101-120 pairs (p5, p20 and p101, respectively) and before or after costs (c0 and c1, respectively). The symbol labels (>) and (<) indicate that the respective null hypothesis is rejected in favor of the alternative and that the average excess returns and/or the Sharpe Ratio of the Copula strategy are found greater or less than the the average excess returns and/or the Sharpe Ratio of the distance strategies, respectively.
			\item \scriptsize $^{\ast\ast\ast}$, $^{\ast\ast}$, $^{\ast}$  significant at 1\\%, 5\\% and 10\\% levels, respectively.
		\end{tablenotes}
		\label{tab:table122}\%
	\end{threeparttable}\%
	
	\medskip
	
	\begin{threeparttable}[H]
		\centering \scriptsize
		\caption{Bootstrap p-values computed from B=10,000 replications for testing the null hypotheses of equality of the average excess returns and Sharpe Ratios over the period between January 2003 and June 2007.}
		\begin{tabularx}{\textwidth}{@{\extracolsep{\fill}}lllllll@{}}
			\toprule
			& & \multicolumn{2}{c}{Copula vs Distance (0.75$\sigma$)} & \multicolumn{1}{c}{} & \multicolumn{2}{c}{Copula vs Distance (2.0$\sigma$)} \\
			\cmidrule{3-4}  \cmidrule{6-7}
			\multicolumn{1}{c}{Scenario} & & \multicolumn{1}{c}{Return} & Sharpe Ratio &       & \multicolumn{1}{c}{Return}& Sharpe Ratio \\
			\midrule
			& \multicolumn{6}{c}{Section 1: Committed Capital} \\
			\midrule
			d0p5c0 & & 0.9748 & 0.7072 &       & 0.8790 & 0.9404 \\
			d0p5c1 & & 0.5484 & 0.6316 &       & 0.3502 & 0.4134 \\
			d1p5c0 & & 0.5212 & 0.5762 &       & 0.3666 & 0.4186 \\
			d1p5c1 & & 0.1936 & 0.1368 &       & $0.0924^{*}(<)$ & $0.0764^{*}(<)$ \\
			d0p20c0 & & $0.0480^{**}(>)$ & $0.0092^{***}(>)$ &       & $0.0394^{**}(>)$ & $0.0110^{**}(>)$ \\
			d0p20c1 & & 0.3402 & 0.1932 &       & 0.3954 & 0.2632 \\
			d1p20c0 & & 0.2902 & 0.1692 &       & 0.3224 & 0.2310 \\
			d1p20c1 & & 0.9366 & 0.9652 &       & 0.8420 & 0.8446 \\
			d0p101c0 & & 0.7730 & 0.4542 &       & 0.9202 & 0.8138 \\
			d0p101c1 & & 0.6774 & 0.7870 &       & 0.3060 & 0.3956 \\
			d1p101c0 & & 0.6842 & 0.4028 &       & 0.9522 & 0.6618 \\
			d1p101c1 & & 0.7424 & 0.8132 &       & 0.4266 & 0.5018 \\
			\midrule
			& \multicolumn{6}{c}{Section 2:  Fully Invested Capital } \\
			\midrule
			d0p5c0 & & 0.5566 & 0.7082 &       & 0.8444 & 0.7436 \\
			d0p5c1 & & 0.7146 & 0.6776 &       & 0.2672 & 0.2368 \\
			d1p5c0 & & 0.9618 & 0.9348 &       & 0.4078 & 0.3650 \\
			d1p5c1 & & 0.5844 & 0.6054 &       & 0.1768 & 0.1672 \\
			d0p20c0 & & $0.0070^{***}(>)$ & $0.0326^{**}(>)$ &       & $0.0236^{**}(>)$ & $0.0702^{*}(>)$ \\
			d0p20c1 & & 0.2512 & 0.3274 &       & 0.4532 & 0.5844 \\
			d1p20c0 & & 0.3770 & 0.4910 &       & 0.5956 & 0.7804 \\
			d1p20c1 & & 0.9174 & 0.9072 &       & 0.7832 & 0.7242 \\
			d0p101c0 & & 0.2632 & 0.4464 &       & 0.6036 & 0.8858 \\
			d0p101c1 & & 0.9454 & 0.8394 &       & 0.4932 & 0.3762 \\
			d1p101c0 & & 0.9310 & 0.8746 &       & 0.8500 & 0.6382 \\
			d1p101c1 & & 0.5476 & 0.4854 &       & 0.3678 & 0.2976 \\
			\bottomrule
		\end{tabularx}\%
		\begin{tablenotes}
			\item \textit{Note:} \scriptsize This table reports the bootstrap p-values for testing the null hypothesis of equality of the average excess returns and the Sharpe Ratios of Copula and distance strategies for the Top 5, 20 and 101-120 pairs over the period between January 2003 and June 2007 (1131 observations). The column labeled Scenario contains symbol labels for trading with no delay or one day waiting period (d0 and d1, respectively), for Top 5, 20 and 101-120 pairs (p5, p20 and p101, respectively) and before or after costs (c0 and c1, respectively). The symbol labels (>) and (<) indicate that the respective null hypothesis is rejected in favor of the alternative and that the average excess returns and/or the Sharpe Ratio of the Copula strategy are found greater or less than the the average excess returns and/or the Sharpe Ratio of the distance strategies, respectively.
			\item \scriptsize $^{\ast\ast\ast}$, $^{\ast\ast}$, $^{\ast}$  significant at 1\\%, 5\\% and 10\\% levels, respectively.
		\end{tablenotes}
		\label{tab:table123}\%
	\end{threeparttable}\%
	
	\medskip
	
	\begin{threeparttable}[H]
		\centering \scriptsize
		\caption{Bootstrap p-values computed from B=10,000 replications for testing the null hypotheses of equality of the average excess returns and Sharpe Ratios over the period between July 2007 and June 2009.}
		\begin{tabularx}{\textwidth}{@{\extracolsep{\fill}}lllllll@{}}
			\toprule
			& & \multicolumn{2}{c}{Copula vs Distance (0.75$\sigma$)} & \multicolumn{1}{c}{} & \multicolumn{2}{c}{Copula vs Distance (2.0$\sigma$)} \\
			\cmidrule{3-4}  \cmidrule{6-7}
			\multicolumn{1}{c}{Scenario} & & \multicolumn{1}{c}{Return} & Sharpe Ratio &       & \multicolumn{1}{c}{Return}& Sharpe Ratio \\
			\midrule
			& \multicolumn{6}{c}{Section 1: Committed Capital} \\
			\midrule
			d0p5c0 & & 0.5652 & 0.8164 &       & 0.9840 & 0.9146 \\
			d0p5c1 & & 0.4254 & 0.6034 &       & 0.8110 & 0.8654 \\
			d1p5c0 & & 0.3444 & 0.4184 &       & 0.9582 & 0.9562 \\
			d1p5c1 & & 0.2780 & 0.3120 &       & 0.7614 & 0.7518 \\
			d0p20c0 & & 0.5274 & 0.2944 &       & 0.3358 & 0.2324 \\
			d0p20c1 & & 0.7578 & 0.5466 &       & 0.5816 & 0.4778 \\
			d1p20c0 & & 0.3778 & 0.3880 &       & 0.6648 & 0.6528 \\
			d1p20c1 & & 0.2318 & 0.2058 &       & 0.4030 & 0.3716 \\
			d0p101c0 & & 0.5994 & 0.5164 &       & 0.5686 & 0.5222 \\
			d0p101c1 & & 0.7750 & 0.7358 &       & 0.7444 & 0.7314 \\
			d1p101c0 & & 0.3762 & 0.3534 &       & 0.5846 & 0.5588 \\
			d1p101c1 & & 0.2728 & 0.2280 &       & 0.3586 & 0.3208 \\
			\midrule
			& \multicolumn{6}{c}{Section 2:  Fully Invested Capital } \\
			\midrule
			d0p5c0 & & 0.4966 & 0.7866 &       & 0.3286 & 0.4472 \\
			d0p5c1 & & 0.7314 & 0.9780 &       & 0.5076 & 0.6158 \\
			d1p5c0 & & 0.7524 & 0.6304 &       & 0.4776 & 0.4980 \\
			d1p5c1 & & 0.6208 & 0.5276 &       & 0.6150 & 0.6194 \\
			d0p20c0 & & 0.7252 & 0.9064 &       & 0.8984 & 0.9934 \\
			d0p20c1 & & 0.8658 & 0.7756 &       & 0.7650 & 0.7102 \\
			d1p20c0 & & 0.6632 & 0.5838 &       & 0.7268 & 0.6896 \\
			d1p20c1 & & 0.4630 & 0.4450 &       & 0.5192 & 0.5156 \\
			d0p101c0 & & 0.8368 & 0.8646 &       & 0.8282 & 0.8428 \\
			d0p101c1 & & 0.8612 & 0.8820 &       & 0.8558 & 0.8566 \\
			d1p101c0 & & 0.7516 & 0.7000 &       & 0.9924 & 0.9802 \\
			d1p101c1 & & 0.5740 & 0.5582 &       & 0.7836 & 0.8258 \\
			\bottomrule
		\end{tabularx}\%
		\begin{tablenotes}
			\item \textit{Note:} \scriptsize This table reports the bootstrap p-values for testing the null hypothesis of equality of the average excess returns and the Sharpe Ratios of Copula and distance strategies for the Top 5, 20 and 101-120 pairs over the period between July 2007 and June 2009 (504 observations). The column labeled Scenario contains symbol labels for trading with no delay or one day waiting period (d0 and d1, respectively), for Top 5, 20 and 101-120 pairs (p5, p20 and p101, respectively) and before or after costs (c0 and c1, respectively). The symbol labels (>) and (<) indicate that the respective null hypothesis is rejected in favor of the alternative and that the average excess returns and/or the Sharpe Ratio of the Copula strategy are found greater or less than the the average excess returns and/or the Sharpe Ratio of the distance strategies, respectively.
			\item \scriptsize $^{\ast\ast\ast}$, $^{\ast\ast}$, $^{\ast}$  significant at 1\\%, 5\\% and 10\\% levels, respectively.
		\end{tablenotes}
		\label{tab:table124}\%
	\end{threeparttable}\%
	
	\medskip
	
	\begin{threeparttable}[H]
		\centering \scriptsize
		\caption{Bootstrap p-values computed from B=10,000 replications for testing the null hypotheses of equality of the average excess returns and Sharpe Ratios over the period between July 2009 and December 2015.}
		\begin{tabularx}{\textwidth}{@{\extracolsep{\fill}}lllllll@{}}
			\toprule
			& & \multicolumn{2}{c}{Copula vs Distance (0.75$\sigma$)} & \multicolumn{1}{c}{} & \multicolumn{2}{c}{Copula vs Distance (2.0$\sigma$)} \\
			\cmidrule{3-4}  \cmidrule{6-7}
			\multicolumn{1}{c}{Scenario} & & \multicolumn{1}{c}{Return} & Sharpe Ratio &       & \multicolumn{1}{c}{Return}& Sharpe Ratio \\
			\midrule
			& \multicolumn{6}{c}{Section 1: Committed Capital} \\
			\midrule
			d0p5c0 & & 0.2060 & $0.0994^{*}(>)$ &       & $0.0710^{*}(>)$ & $0.0406^{**}(>)$ \\
			d0p5c1 & & 0.5712 & 0.4710 &       & 0.3468 & 0.3094 \\
			d1p5c0 & & $0.0830^{*}(>)$ & $0.0276^{**}(>)$ &       & $0.0182^{**}(>)$ & $0.0060^{***}(>)$ \\
			d1p5c1 & & 0.3084 & 0.1866 &       & 0.1264 & $0.0928^{*}(>)$ \\
			d0p20c0 & & 0.5594 & 0.2600 &       & 0.4474 & 0.2568 \\
			d0p20c1 & & 0.5856 & 0.6648 &       & 0.6202 & 0.6932 \\
			d1p20c0 & & 0.2160 & $0.0446^{**}(>)$ &       & 0.1094 & $0.0388^{**}(>)$ \\
			d1p20c1 & & 0.9232 & 0.6896 &       & 0.7646 & 0.6118 \\
			d0p101c0 & & 0.4992 & 0.5374 &       & 0.5090 & 0.5412 \\
			d0p101c1 & & 0.7700 & 0.4760 &       & 0.6410 & 0.4494 \\
			d1p101c0 & & $0.0680^{*}(>)$ & $0.0540^{*}(>)$ &       & $0.0686^{*}(>)$ & $0.0552^{*}(>)$ \\
			d1p101c1 & & 0.3662 & 0.4770 &       & 0.4906 & 0.5746 \\
			\midrule
			& \multicolumn{6}{c}{Section 2:  Fully Invested Capital } \\
			\midrule
			d0p5c0 & & $0.0378^{**}(>)$ & $0.0504^{*}(>)$ &       & $0.0678^{*}(>)$ & $0.0890^{*}(>)$ \\
			d0p5c1 & & 0.2752 & 0.2888 &       & 0.4030 & 0.4274 \\
			d1p5c0 & & 0.1300 & 0.1436 &       & $0.0890^{*}(>)$ & 0.1072 \\
			d1p5c1 & & 0.3210 & 0.3274 &       & 0.2392 & 0.2546 \\
			d0p20c0 & & $0.0360^{**}(>)$ & 0.1400 &       & 0.1152 & 0.2344 \\
			d0p20c1 & & 0.7692 & 0.9662 &       & 0.9024 & 0.7640 \\
			d1p20c0 & & 0.1242 & 0.2768 &       & 0.1948 & 0.3362 \\
			d1p20c1 & & 0.5508 & 0.6892 &       & 0.7484 & 0.8818 \\
			d0p101c0 & & 0.8060 & 0.9630 &       & 0.8984 & 0.9774 \\
			d0p101c1 & & $0.0850^{*}(<)$ & 0.2150 &       & 0.1242 & 0.1908 \\
			d1p101c0 & & 0.5292 & 0.4558 &       & 0.5538 & 0.5148 \\
			d1p101c1 & & 0.8936 & 0.8688 &       & 0.8632 & 0.9894 \\
			\bottomrule
		\end{tabularx}\%
		\begin{tablenotes}
			\item \textit{Note:} \scriptsize This table reports the bootstrap p-values for testing the null hypothesis of equality of the average excess returns and the Sharpe Ratios of Copula and distance strategies for the Top 5, 20 and 101-120 pairs over the period between July 2009 and December 2015 (1637 observations). The column labeled Scenario contains symbol labels for trading with no delay or one day waiting period (d0 and d1, respectively), for Top 5, 20 and 101-120 pairs (p5, p20 and p101, respectively) and before or after costs (c0 and c1, respectively). The symbol labels (>) and (<) indicate that the respective null hypothesis is rejected in favor of the alternative and that the average excess returns and/or the Sharpe Ratio of the Copula strategy are found greater or less than the the average excess returns and/or the Sharpe Ratio of the distance strategies, respectively.
			\item \scriptsize $^{\ast\ast\ast}$, $^{\ast\ast}$, $^{\ast}$  significant at 1\\%, 5\\% and 10\\% levels, respectively.
		\end{tablenotes}
		\label{tab:table125}\%
	\end{threeparttable}\%
	
	\vspace{1.0cm}
	
	\section{Concluding Remarks}
	
The main objective of the paper is to compare the copula and distance estimation procedures to understand better the factors that affect the profitability of the strategies and to find if the approaches can produce sustainable alpha. The comparison made by means of an empirical investigation suggests that the copula strategy has a superior performance than the distance approaches when less restrictions are imposed.
	
The main findings are summarized below.
	
	\begin{enumerate}
		\item The copula strategy consistently outperforms the distance strategies in the long term in terms of risk-adjusted returns when a fast trade can be executed for Top 5 and Top 20 pairs and both weighting structures considered. However, the copula method is more sensitive than the distance approaches to timing and transaction costs.
		\item The copula approach delivers economically large and significant alphas after accounting for various asset pricing factors, usually at 1\\%, showing that our results are robust to \citet*{ff93} and \citet*{ff15}'s risk factors. In addition, we find very low R-square and adjusted R-squared for any strategy, particularly for the copula strategy which indicates that the approach is nearly factor-neutral over the whole sample period.
		\item Using the stationary bootstrap of \citet*{pr94} in order to mitigate data-snooping criticisms, we test the statistical significance of the returns and Sharpe Ratios among our strategies. We find that copula strategy outperforms the distance strategies when fewer restrictions are imposed on trade for Top 5 and Top 20 pairs.
		\item The copula approach is able to identify more trading opportunities than the distance strategies. Moreover, when using the copula method, the pairs stay open for a shorter period. This is a very important finding, since every trading signal is an opportunity to profit. Conversely, one of the reasons for copula strategy being more affected by trading costs is due to a much greater number of trading signals. Implementing a stop-loss strategy may lead to higher returns and reduce the standard
		deviation of returns.
		\item The sub-period analysis reveals that the copula strategy exhibit a very strong relative out-of-sample performance during the first sub-period (91/2-99) and last sub-period (2009/2-2015) and that distance strategies outperform the copula method during 2000-2002.
		
	\end{enumerate}
	
	Finally, we provide some suggestions for future research to improve our findings. First, we could improve the method of pairs selection before using the copula approach. Since copulas capture better nonlinear dependence structures, we suspect that if we use a nonlinear association measurement such as the randomized dependency coefficient \citep{lopez2013randomized}, or a procedure based on data mining tools as random forest \citep{dlrz10} to select the pairs, the performance of the copula method may be enhanced. Second, since daily returns only capture close-to-close volatility, leaving much to be said regarding the actual volatility of the stocks intra-day, the inclusion of realized measures of volatility using higher frequency data may indeed prove beneficial.
	
	We can use a training set
	
	Model validation is a stronger concept than model selection: we make positive attempts to validate the model rather than merely looking for signs of inconsistency (which is what residual plots, etc., do). The ideal situation is to be able to test the model in practice, This means that, after selecting a model by whatever criterion has been adopted, we check its predictive power directly on new data that become available. In the absence of new data, we can create some for ourselves by splitting data into two subsets. The idea is to select and fit the model based entirely on one part of the data, and then test it on the remaining part
	
	\addcontentsline{toc}{section}{References}
	\bibliographystyle{rfs}
	\bibliography{mypapers}
	
	\newpage
	
	\section*{Appendix}
	\addcontentsline{toc}{section}{Appendix}
	\vspace{0.6cm}
	
	
	This appendix contains additional figures representing Sharpe ratios patterns for each of the three strategies.
	
%		\begin{figure}[H]
%		\centering
%		\includegraphics[scale=0.6]{fig10_p1_rev.pdf}
%		\caption{Sharpe Ratio (annualized) of pairs trading strategies before costs and with no delay for each subperiod}
%		\%\caption*{comments}
%		\label{fig:fig112}
%	\end{figure}
%	
%	\begin{figure}[H]
%		\centering
%		\includegraphics[scale=0.6]{fig11_p1_rev.pdf}
%		\caption{Sharpe Ratio (annualized) of pairs trading strategies after costs and with no delay for each subperiod}
%		\%\caption*{comments}
%		\label{fig:fig113}
%	\end{figure}
	
%	\begin{figure}[H]
%		\centering
%		\includegraphics[scale=0.6]{fig12_p1_rev.pdf}
%		\caption{Sharpe Ratio (annualized) of pairs trading strategies before costs and with one-day waiting period for each subperiod}
%		\%\caption*{comments}
%		\label{fig:fig114}
%	\end{figure}
%
%	
%	\begin{figure}[H]
%		\centering
%		\includegraphics[scale=0.6]{fig13_p1_rev.pdf}
%		\caption{Sharpe Ratio (annualized) of pairs trading strategies after costs and with one-day waiting period for each subperiod}
%		\%\caption*{comments}
%		\label{fig:fig115}
%	\end{figure}
	
	
	\end{document} 